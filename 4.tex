\section{Fernando Lorencius S}
\subsection{Teori}
\begin{enumerate}

\item Apa itu fungsi file csv, jelaskan sejarah dan contoh\\
Jawaban :

\begin{itemize}
\item Fungsi File CSV
\end{itemize}

CSV (Comma Separated Value) adalah format basis data sederhana yang berisikan data record , dipisahkan dengan tanda koma (,) atau titik koma (;).Dalam bahasa pemrograman Python telah disediakan modul  csv  yang khusus untuk mengolah data berformat csv.

\begin{itemize}
\item Sejarah 
File csv muncul pertama kali sekitar 10 tahun sebelum Personal Computer (PC) pertama  didunia yaitu sejak sekitar tahun 1972, akan tetapi sebutan file csv digunakan pertama kali pada tahun 1983.
					\item Contoh : 
\end{itemize}

\begin{itemize}
\item Contoh file csv
\begin{verbatim}
Nama,Umur, Alamat
Fernando Lorencius S, 19 Tahun, Bandung.
\end{verbatim}
\end{itemize}

\item Aplikasi yang dapat membuat file csv\\
Jawaban :

\begin{itemize}
\item Microsoft Excel
\item Google Spreadsheet
\item Notepad ++
\item Text Editor
\end{itemize}

\item  Cara menulis dan membaca file csv di excel atau spreadsheet\\
Jawaban :

Membuka file CSV dalam program spreadsheet, dapat membuat lebih mudah dibaca. Misalnya, jika di komputer sudah ter-instal Microsoft Excel -- CSV file to excel, dengan klik dua kali file .csv untuk membukanya di Excel secara default. Jika tidak terbuka di Excel, dapat mengklik kanan file CSV dan pilih Open With > Excel.

Jika tidak memiliki Excel, untuk melihatnya dapat mengunggah file ke layanan seperti Google Spreadsheet atau menggunakan aplikasi gratis membuka CSV download dan install LibreOffice Calc.

Excel dan program spreadsheet lainnya menyajikan isi dari CSV dengan menyortirnya ke dalam kolom, dan baris.

\begin{itemize}
\item Cara Menulis file csv
\end{itemize}
Berikut adalah kode untuk menulis file CSV dengan menggunakan built-in module csv yang dimiliki Python

\begin{verbatim}
import csv

siswa = [
    ('Udin', 'A', 90),
    ('cecep', 'B', 85),
    ('Agus', 'A', 80),
    ('asep', 'B', 90),
    ('Solihin', 'C', 70)
]

# tentukan lokasi file, nama file, dan inisialisasi csv
f = open('siswa.csv', 'w')
w = csv.writer(f)
w.writerow(('Nama','Kelas','Nilai'))

# menulis file csv
for s in siswa:
    w.writerow(s)

# menutup file csv
f.close()
\end{verbatim}

\begin{itemize}
\item Cara membaca file csv
\end{itemize}

Berikut adalah contoh kode untuk membaca file CSV 

\begin{verbatim}
import csv

# tentukan lokasi file, nama file, dan inisialisasi csv
f = open('siswa.csv', 'r')
reader = csv.reader(f)

# membaca baris per baris
for row in reader:
    print row

# menutup file csv
f.close()
\end{verbatim}

\item Sejarah library csv\\
Jawaban :\\
library csv dibuat untuk permudah mengolah data. Dan mempermudah untuk melakukan export dan import file csv itu sendiri.


\item Sejarah library pandas\\
Jawaban :\\
Pandas (Python Data Analysis Library) merupakan distribusi/modul opensource dari python untuk memporese kegiatan data manajemen dan data analysis. Pandas diciptakan oleh Wes McKinney pada sekitaran tahun 2008 dan dikembangkan oleh salah satu koleganya Sien Chang pada tahun 2010. Pandas muncul karena kebutuhan komunitas pengguna opensource khususnya python akan packages khusus yang berguna untuk melakukan analisa data, seperti import dan export data, melakukan manipulasi data dsb. Untuk proses import data menggunakan distribusi pandas dalam python itu hal yang tidak mudah, karena dalam proses membutuhkan ketelitian mengenai struktur data.


\item Jelaskan  fungsi-fungsi yang terdapat di library csv\\
Jawaban :\\
Terdapat 2 fungsi dari library csv, yaitu :

\begin{itemize}
\item membaca file CSV
\end{itemize}

\begin{itemize}
\item menulis file CSV
\end{itemize}


\item Jelaskan  fungsi-fungsi yang terdapat di library pandas\\
Jawaban :\\
 tidak jauh beda dengan library csv,namun dalam library pandas penulisannya lebih sederhana dan terlihat lebih rapih dari pada library csv.

\end{enumerate}