\section{D. Irga B. Naufal Fakhri}
\subsection{Pemahaman Teori}
\begin{enumerate}
\item CSV

CSV (Comma Separated Values file) adalah sebuah tipe file text biasa yang memiliki penataan khusus yang biasanya berfungsi untuk mengelola data. sesuai dengan namanya file csv memisahkan setiap data menggunakan koma (,).

Format data CSV pertama kali digunakan pada tahun 1978 pada complier FORTRAN 77, kemudian nama CSV baru muncul dan mulai digunakan pada tahun 1983 

Contoh data pada csv:
\lstinputlisting[firstline=0, lastline=2]{src/1174066.csv}

\item Aplikasi yang bisa menciptakan CSV

Semua aplikasi teks editor seperti notepad++, vscode, sublime ataupun notepad dapat menciptakan CSV termasuk aplikasi spreadsheet seperti Microsoft Excel, Libre Office 

\item Jelaskan bagaimana cara menulis dan membaca file csv di Excel atau spreadsheet

\begin{itemize}
	\item Buka Microsoft Excel 2019-nya lalu buat dokumen baru
	\item Isikan data sesuai dengan kebutuhan, yang paling atas akan menjadi header dari file csv
	\item Setelah memasukkan data, klik file lalu klik Save As
	\item Pilih Browse dan pilih tempat menyimpannya akan dimana
	\item Masukkan nama file pada File Name
	\item Lalu pada Save As Type pilih CSV (comma delimited) (*.csv)
	\item Maka hasil file akan seperti ini
	\lstinputlisting[firstline=0, lastline=2]{src/1174066.csv}
\end{itemize}

\item Jelaskan sejarah library csv

Module csv mengimplementasikan kelas untuk membaca dan menulis data kedalam format CSV. Hal ini memungkinkan programmer untuk "tulis data ini dalam format yang disukai oleh Excel," atau "baca data dari file yang dihasilkan oleh Excel," tanpa mengetahui detail yang tepat dari format CSV yang digunakan oleh Excel. Pemrogram juga dapat menggambarkan format CSV yang dipahami oleh aplikasi lain atau menentukan format CSV tujuan khusus untuk mereka sendiri.

\item Jelaskan sejarah library pandas

pandas adalah sebuah library open source dan berlisensi BSD yang menyediakan performa yang tinggi, mudah digunakan struktur data dan data analisis untuk python.

\item Jelaskan fungsi-fungsi yang terdapat di library csv
\begin{itemize}
	\item csv.reader
	
	Berfungsi untuk membaca dan mengembalikan data kedalam variable dari file csv.
	Fungsi reader dirancang untuk mengambil data pada setiap baris didalam file dan membuat daftar semua kolom. Kemudian, tinggal dipilih kolom mana yang diinginkan untuk data variabel.
	\lstinputlisting[firstline=10, lastline=22]{src/1174066_csv.py}
	
	\item csv.writer
	
	Berfungsi untuk menuliskan data dari variable kedalam file csv.
	Fungsi writer akan membuat objek yang cocok untuk menulis. Untuk mengulang data yang ada di atas baris, gunakan fungsi writerow.
	\lstinputlisting[firstline=37, lastline=43]{src/1174066_csv.py}
	
	\item csv.register\textunderscore dialect
	
	Mendaftarkan dialect pada csv
	\item csv.unregister\textunderscore dialect
	
	Menghapus dialect yang diasosiasi dengan nama dari registry dialect
	
	\item csv.list\textunderscore dialects
	
	Mengembalikan dialect yang diasosiasi dengan nama
	
	\item csv.field\textunderscore size\textunderscore limit
	
	Mengembalikan ukuran field maksimum yang diizinkan oleh parser.
	
	\item csv.DictReader
	
	Berfungsi untuk membaca dan mengembalikan data kedalam variable dictionary dari file csv.
	\lstinputlisting[firstline=24, lastline=35]{src/1174066_csv.py}
	
\end{itemize}

\item Jelaskan fungsi-fungsi yang terdapat di library pandas
\begin{itemize}
	\item pandas.read\textunderscore csv
	
	Berfungsi untuk membaca dan mengembalikan data kedalam format DataFrame dari file csv.
	\lstinputlisting[firstline=45, lastline=48]{src/1174066_csv.py}
	
	\item to\textunderscore csv
	
	Berfungsi untuk mengedit data didalam csv dan menulisnya kedalam file csv
	\lstinputlisting[firstline=50, lastline=57]{src/1174066_csv.py}
\end{itemize}
\end{enumerate}
%%%%%%%%%%%%%%%%%%%%%%%%%%%%%%%%%%%%%%%%%%%%%%%%%%%%%%%%%%%%%%%%%%%%%%%%%%%%%%%%%%%%%%%%%%%%%%%%%%%%%%%
\section{Difa Al Fansha}
\subsection{Teori}
\begin{enumerate}

\item Apa itu fungsi file csv, jelaskan sejarah dan contoh\\
Jawaban :

\begin{itemize}
\item Fungsi File CSV
\end{itemize}

Comma Separated Value atau CSV adalah format data yang memudahkan penggunanya melakukan penginputan data ke database secara sederhana. CSV bisa digunakan dalam standar file ASCII, di mana setiap record dipisahkan dengan tanda koma atau titik koma.

\begin{itemize}
\item Sejarah 
\end{itemize}
File csv muncul pertama kali sekitar 10 tahun sebelum Personal Computer (PC) pertama didunia yaitu sejak sekitar tahun 1972, akan tetapi sebutan file csv digunakan pertama kali pada tahun 1983.

\begin{itemize}
\item Contoh file csv
\begin{verbatim}
Nama,Umur, Alamat
Difa Al Fansha, 19 Tahun, Jl. Setapak.
\end{verbatim}
\end{itemize}

\item Aplikasi yang dapat membuat file csv\\
Jawaban :

\begin{itemize}
\item Microsoft Excel
\item Google Spreadsheet
\item Notepad ++
\item Text Editor lainnya
\end{itemize}

\item  Cara menulis dan membaca file csv di excel atau spreadsheet\\
Jawaban :

\begin{itemize}
\item Cara Menulis file csv
\end{itemize}
Berikut adalah kode untuk menulis file CSV dengan menggunakan built-in module csv yang dimiliki Python

\begin{verbatim}
import csv

siswa = [
    ('arslan', 'A', 90),
    ('bayu', 'B', 85),
    ('niko', 'A', 80),
    ('abdul', 'B', 90),
    ('dahlan', 'C', 70)
]

# tentukan lokasi file, nama file, dan inisialisasi csv
f = open('siswa.csv', 'w')
w = csv.writer(f)
w.writerow(('Nama','Kelas','Nilai'))

# menulis file csv
for s in siswa:
    w.writerow(s)

# menutup file csv
f.close()
\end{verbatim}

\begin{itemize}
\item Cara membaca file csv
\end{itemize}

Berikut adalah contoh kode untuk membaca file CSV 

\begin{verbatim}
import csv

# tentukan lokasi file, nama file, dan inisialisasi csv
f = open('siswa.csv', 'r')
reader = csv.reader(f)

# membaca baris per baris
for row in reader:
    print row

# menutup file csv
f.close()
\end{verbatim}

\item Sejarah library pandas\\
Jawaban :\\
library csv dibuat untuk permudah mengolah data. Dan mempermudah untuk melakukan export dan import file csv itu sendiri


\item Sejarah library csv\\
Jawaban :\\
 library pandas dibuat agar bahasa pemograman python bisa bersaing R dan matlab, yang digunakan untuk mengolah banyak data , keperluan big data, data mining data science dan sebagainya.

\item Jelaskan  fungsi-fungsi yang terdapat di library csv\\
Jawaban :\\
Terdapat 2 fungsi dari library csv, yaitu :

\begin{itemize}
\item Cara membaca file
\end{itemize}

\begin{itemize}
\item Cara menulis file
\end{itemize}
Di Python, hasil pembacaan setiap baris pada file CSV akan dikonversi menjadi list Python.

\item Jelaskan  fungsi-fungsi yang terdapat di library pandas\\
Jawaban :\\
library pandas penulisannya lebih sederhana dan terlihat lebih rapih dari pada library csv.

\end{enumerate}

%%%%%%%%%%%%%%%%%%%%%%%%%%%%%%%%%%%%%%%%%%%%%%%%%%%%%%%%%%%%%%%%%%%%%%%%%%%%%%%%%%%%%%%%%%%
\section{Alvan Alvanzah|1174077}
\subsection{Pemahaman Teori}

\begin{enumerate}
    \item Apa itu fungsi file csv, jelaskan sejarah dan contoh
    \par Comma Separated Values atau CSV adalah suatu format data dalam basis data di mana setiap record dipisahkan dengan tanda koma (,) atau titik koma (;). Selain sederhana, format ini dapat dibuka dengan berbagai text-editor seperti Notepad, Wordpad, bahkan MS Excel. File CSV menyimpan informasi yang dipisahkan oleh koma, tidak menyimpan informasi dalam kolom. Ketika teks dan angka disimpan dalam file CSV, mudah untuk memindahkannya dari satu program ke program lainnya.
    \par Dari rilis pertama, Excel menggunakan format file biner yang disebut Binary Interchange File Format (BIFF) sebagai format file utamanya. Ini berubah ketika Microsoft merilis Office System 2007 yang memperkenalkan Office Open XML sebagai format file utamanya. Office Open XML adalah file kontainer berbasis XML yang mirip dengan XML Spreadsheets (XMLSS), yang diperkenalkan di Excel 2002. File versi XML tidak bisa menyimpan makro VBA. Meskipun mendukung format XML baru, Excel 2007 masih mendukung format lama yang masih berbasis BIFF tradisional. Selain itu Microsoft Excel juga mendukung format Comma Separated Values (CSV).
    \par Contoh penulisan :
    \par “Alvan”,”Sorong”,”19”
    \par “Bambang”,”Bekasi”,”20”
    \par “Sukirman”,”Bandung”,”20”
    \par “Suci”,”Depok”,”22”

    \item Aplikasi-aplikasi apa saja yang bisa menciptakan file csv
    \begin{itemize}
        \item Texteditor
        \par Seperti Notepad++, Visual studio code, Atom, dan Sublime
        \item Program Spreadsheet
        \par Seperti Excel, Google Spreadsheet, dan LibreOffice Calc
    \end{itemize}
    
    \item Jelaskan bagaimana cara menulis dan membaca file csv di excel atau spreadsheet
        \par Cara menulis
        \begin{itemize}
        \item Buat dokumen baru di Excel.
        \item Tambahkan judul kolom untuk setiap potongan informasi yang ingin dicatat (misalnya nama depan, nama belakang, alamat email, nomor telepon, dan ulang tahun), lalu ketikkan informasi dalam kolom yang sesuai.
        \item Setelah selesai, Pilih File lalu Simpan Sebagai.
        \item Gunakan kotak menurun untuk memilih CSV (Berbatas koma) (.csv), beri nama pada file, lalu pilih Simpan.
    \end{itemize}
    \par Cara membaca
    \begin{itemize}
        \item Buka MS Excel Anda.
        \item Klik Data lalu Get External Data lalu From Text.
        \item Akan muncul Text Import Wizard, arahkan pada file csv yang ingin anda buka lalu Open.
        \item Setelah File terbuka, akan muncul Text Import Wizard Step 1 lalu Pilih Delimited, Kemudian Next (Di sini, bisa juga menentukan baris awal yang akan di import), Step 2 lalu Centang pada Tab dan Comma (Atau sesuai pengaturan File Anda) lalu Next, Step 3 lalu Atur Format data pada tiap kolom yang tampil dan klik Finish.
        \item File anda sudah tertata rapi dan dapat di baca dengan mudah melalui MS Excel 2007
    \end{itemize}
    
    \item Jelaskan sejarah library csv
    \par library csv dibuat untuk permudah mengolah data. Dan mempermudah untuk melakukan export dan import file csv itu sendiri.
    
    \item Jelaskan sejarah library pandas
    \par library pandas dibuat agar bahasa pemograman python bisa bersaing R dan matlab, yang digunakan untuk mengolah banyak data , keperluan big data, data mining data science dan sebagainya.
    
    \item Jelaskan fungsi-fungsi yang terdapat di library csv
    \par Library csv mempunyai keunggulan dibandingkan format data lainnya adalah soal kompatibilitas. File csv dapat digunakan, diolah, diekspor atau impor, dan dimodifikasi menggunakan berbagai macam perangkat lunak dan bahasa pemrograman. Pada library csv mempunyai fungsi import dan eksport data yang baik dan bisa digunakan dalam jumlah besar.
    
    \item Jelaskan fungsi-fungsi yang terdapat di library pandas
    \par pandas menyediakan beragam fungsi operasi untuk mengolah data. Contoh jika menggunakan series bisa mencari nilai max, min, dan mean secara langsung, bahkan juga bisa melakukan operasi perpangkatan pada nilai Series secara langsung. Pandas dapat mengolah suatu data dan mengolahnya seperti join, distinct, group by, agregasi, dan teknik seperti pada SQL. Hanya saja dilakukan pada tabel yang dimuat dari file ke RAM.
\end{enumerate}
%%%%%%%%%%%%%%%%%%%%%%%%%%%%%%%%%%%%%%%%%%%%%%%%%%%%%%%%%%%%%%%%%%%%%%%%%%%%%%%%%%%%%%%%%%%%%%%%%%%%%%%%%%%%%%%%%%%%%%%%

\section{Arrizal Furqona Gifary}
\begin{enumerate}
    \item Apa itu fungsi file csv, jelaskan sejarah dan contoh
    File CSV (Nilai Terbatas Koma) adalah jenis file khusus yang dapat Anda buat atau edit di Excel. File CSV menyimpan informasi yang dipisahkan oleh koma, tidak menyimpan informasi dalam kolom. Ketika teks dan angka disimpan dalam file CSV, mudah untuk memindahkannya dari satu program ke program lainnya.
    Dari rilis pertama, Excel menggunakan format file biner yang disebut Binary Interchange File Format (BIFF) sebagai format file utamanya. Ini berubah ketika Microsoft merilis Office System 2007 yang memperkenalkan Office Open XML sebagai format file utamanya. Office Open XML adalah file kontainer berbasis XML yang mirip dengan XML Spreadsheets (XMLSS), yang diperkenalkan di Excel 2002. File versi XML tidak bisa menyimpan makro VBA.
    Meskipun mendukung format XML baru, Excel 2007 masih mendukung format lama yang masih berbasis BIFF tradisional. Selain itu Microsoft Excel juga mendukung format Comma Separated Values (CSV), DBase File (DBF), SYMbolic LinK (SYLK), Format Interchange Data (DIF) dan banyak format lainnya, termasuk format lembar kerja 1-2 Lotus - 3 (WKS, WK1, WK2, dll.) Dan Quattro Pro.
    \item Aplikasi-aplikasi apa saja yang bisa menciptakan file csv
    \begin{itemize}
        \item Texteditor
        Seperti notepad++,visual studio code,atom,sublime dan lain sebagainya
        \item Program Spreadsheet
        Seperti excell,google spreadshare,LibreOfficecalc
    \end{itemize}
    \item Jelaskan bagaimana cara menulis dan membaca file csv di excel atau spreadsheet
    Untuk menulisnya untuk yang paling atas itu kita buat headernya,untuk mepermudah membedakan datanya,dan untuk baris kedua dan seterusnya itu untuk data itu sendiri.
    dan setelah di buat kalian save as kemudian pilih format CSV.
    dan untuk membukan cukup di double clik file tersebut
    \item Jelaskan sejarah library csv
    library csv dibuat untuk permudah mengolah data. Dan mempermudah untuk melakukan export dan import file csv itu sendiri
    \item Jelaskan sejarah library pandas
    library pandas dibuat agar bahasa pemograman python bisa bersaing R dan matlab, yang digunakan untuk mengolah banyak data , keperluan big data, data mining data science dan sebagainya.
    \item Jelaskan fungsi-fungsi yang terdapat di library csv
    Terdapat 2 fungsi yang bisa digunakan oleh library csv
    Pertama,fungsi membaca file csv.
    fungsi ini bisa menggunakan list dan dictionary
    Dengan list :
    \lstinputlisting[firstline=11, lastline=21]{src/1174070/1174070_csv.py}
    Dengan dictionary :
    \lstinputlisting[firstline=24, lastline=33]{src/1174070/1174070_csv.py}
    Kedua,fungsi menulis file csv.
    \lstinputlisting[firstline=36, lastline=40]{src/1174070/1174070_csv.py}
    \item Jelaskan fungsi-fungsi yang terdapat di library pandas
    Hampir sama dengan library csv,tp library pandas penulisannya lebih sederhana dan terlihat lebih rapih dari pada library csv.
    \lstinputlisting[firstline=43, lastline=44]{src/1174070/1174070_csv.py}
\end{enumerate}
\section{Dini Permata Putri}
1.apa itu fungsi file csv, jelaskan sejarah dan contoh\\
jawab : file CSV atau Comma Separated Value seperti namanya berisi teks data yang tiap datanya dipisahkan dengan tanda koma. Sebagai gambaran, sebuah file CSV bisa berisi data berikut ini :\\
HeaderA, HeaderB, HeaderC\\
RowA1, RowB1, RowC1\\
RowA2, RowB2, RowC2\\
Jika kita membuat sebuah file di Excel dan menyimpannya dalam format CSV, maka file tersebut dibuka di Notepad maka akan terlihat isi file yang kurang lebih formatnya sama seperti di atas.\\

2. aplikasi-aplikasi apa saja yang bisa menciptakan file csv?\\
jawab : microsoft office, dll.\\

3. jelaskan bagaimana cara menulis dan membaca file csv di excel atau spreadsheet\\
jawab : 1. Buka MS Excel Anda\\
2. Klik Data > Get External Data > From Text\\ 
3. Akan muncul Text Import Wizard, arahkan pada file csv yang ingin anda buka > Open.\\
4. Setelah File terbuka, akan muncul Text Import Wizard\\
Step 1 –> Pilih Delimited, Kemudian Next (Di sini, bisa juga menentukan baris awal yang akan di import)\\
Step 2 –> Centrang pada Tab dan Comma (Atau sesuai pengaturan File Anda) > Next\\
Step 3 –> Atur Format data pada tiap kolom yang tampil dan klik Finish\\

4. jelaskan sejarah library csv\\
jawab : Jaringan perpustakaan digital pertama di Indonesia mulai beroperasi pada bulan Juni 2001.  Jaringan Perpustakaan Digital tersebut itu bernama IndonesiaDLN (Digital Library Network).  IndonesiaDLN diprakarsai oleh Knowledge Management Research Group (KMRG) Institut Teknologi Bandung (ITB) yang merintis pembuatan jaringan perpustakaan digital (digital library network) antar lembaga pendidikan tinggi.  Jaringan pustaka digital bertujuan mempermudah kalangan akademik dan masyarakat umum untuk mengakses hasil penelitian, tugas akhir mahasiswa, tesis maupun disertasi. Dana awal pengembangan jaringan berasal dari Singapura sebanyak 60.000 dolar Kanada, dan dari Yayasan Litbang Telekomunikasi dan Teknologi Informasi (YLTI) sebanyak Rp 150 juta. \\

Pada awal berdirinya, lembaga yang bergabung dalam jaringan pustaka digital IndonesiaDLN antara lain Proyek Pengembangan Universitas Indonesia Timur, LIPI Jakarta, Universitas Brawijaya Malang, Universitas Muhammadiyah Malang, Lembaga Penelitian ITB, Pasca Sarjana ITB, serta Computer Network Research Group (CNRG).\\

Ketua KMRG saat itu sekaligus sebagai penggagas IndonesiaDLN Ismail Fahmi menjelaskan bahwa ide dasar pengembangan pustaka digital bahwa hasil pemikiran dan penelitian harus bisa dipertukarkan (share) dan diakses secara cepat dan mudah. Copyright untuk tugas akhir maupun penelitian pada dasarnya termasuk public domain kecuali yang terikat pada perjanjian dengan industri atau dalam persiapan untuk mendapatkan hak paten. IndonesiaDLN bertujuan agar hasil-hasil penelitian dari perguruan tinggi maupun lembaga penelitian bisa diakes dari manapun di seluruh penjuru dunia dapat diakses secara mudah dan murah dalam bentuk digital, tanpa memerlukan biaya transportasi maupun fotokopi yang biasanya harus dengan mengeluarkan biaya cukup tinggi.\\

Gagasan pembentukan jaringan perpustakaan nasional ini bermula dari peluncuran situs Ganesha Digital Library/GDL (perpustakaan digital milik ITB) Oktober 2000. Sekitar 20 institusi kemudian terlibat dalam proyek jaringan perpustakaan ini. Beberapa server individu juga ikut menyebarkan informasinya melalui GDL, seperti Onno W. Purbo, Budi Rahardjo, dan Ismail Fahmi.\\

Jaringan pustaka digital ini merupakan satu dari beberapa produk KMRG. Produk lainnya adalah Ganesha digital library, software untuk otomatisasi perpustakaan (GNU-Lib) serta software untuk katalog database perpustakaan\\
(http://isisnetwork.lib.itb.ac.id).\\

Menurut Sekjen IndonesiaDLN,  Ismail Fahmi, jaringan perpustakaan digital ini berfungsi sebagai terminal dari berbagai server di Indonesia yang menyediakan informasi ilmu pengetahuan. Misi jaringan ini adalah mengelola ilmu pengetahuan yang dimiliki bangsa Indonesia, dalam satu jaringan yang terdistribusi dan terbuka.\\

5. jelaskan sejarah library pandas\\
jawab : engembang Wes McKinney mulai mengerjakan pandas pada 2008 ketika di AQR Capital Management karena kebutuhan akan alat kinerja tinggi yang fleksibel untuk melakukan analisis kuantitatif pada data keuangan. Sebelum meninggalkan AQR, dia bisa meyakinkan manajemen untuk mengizinkannya membuka sumber perpustakaan.\\

Pegawai AQR lainnya, Chang She, bergabung dengan upaya ini pada 2012 sebagai kontributor utama kedua ke perpustakaan.\\

Pada 2015, panda ditandatangani sebagai proyek NumFOCUS yang disponsori secara fiskal, sebuah badan amal nirlaba 501 (c) (3) di Amerika Serikat.\\

6. jelaskan fungsi-fungsi yang terdapat di library csv\\
jawab : Jika kita membuat sebuah file di Excel dan menyimpannya dalam format CSV, maka file tersebut dibuka di Notepad maka akan terlihat isi file yang kurang lebih formatnya sama seperti di atas.\\

7. jelaskan fungsi-fungsi yang terdapat di library pandas\\
jawab : dapat mengolah suatu data dan mengolahnya seperti join, distinct, group by, agregasi, dan teknik seperti pada SQL. Hanya saja dilakukan pada tabel yang dimuat dari file ke RAM.\\

Pandas juga dapat membaca file dari berbagai format seperti .txt, .csv, .tsv, dan lainnya. Anggap saja Pandas adalah spreadsheet namun tidak memiliki GUI dan punya fitur seperti SQL.\\