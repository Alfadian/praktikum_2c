\documentclass[11pt]{Article}

\begin{document}

\section{Resume} 
\label{Resume}
\subsubsection{Resume Sejarah Python}

Python merupakan bahasa pemrograman yang bersifat interpretative. Python dikembangkan oleh Guido Van Rossum pada tahun 1990 di CWI, Amsterdam sebagai kelanjutan dari Bahasa pemrograman ABC. Nama Python dipilih oleh Guido sebagai nama bahasa ciptaanya karena kecintaannya pada televisi Monthy Python’s Flying Circus. Sekarang,distribusi python sudah mencapai versi 2.6.1 sampai dengan 3.0.\\

Pada tahun 1995 Guido pindah ke CNRI di Virginia Amerika sambil terus melanjutkan pengembangan Python. Versi terakhir yang dikeluarkan adalah 1.6. Tahun 2000, setelah itu Guido dan timnya berpindah lagi ke BeOpen.com dan dari sini mereka mengeluarkan Python versi 2.0.\\

Sampai sekarang pengembangan python terus dilakukan oleh pada pemrogram yang diambil alih oleh  Guido dan juga Python Softwa
re Foundation. 

\subsubsection{Perbedaan Python 2 dan 3}

Python versi 2 merupakan versi yang dikembangkan pada tahun 2000 dan yang paling banyak digunakan saat ini, baik dilingkungan produksi dan pengembangan, dan untuk membuka python 2 ini tinggal ketik python.\\
 
Sementara Python versi 3 adalah pengembangan lanjutan dari versi 2, yang terakhir rilis pada tahun 2008. Python 3 memiliki lebih banyak fitur dibandingkan Python 2, versi 3 ini ketika akan dibuka maka akan menggunakan perintah python3. Perubahan terbesar pada Python 3 termasuk memasukkan statemen print ke dalam built-in function.\\

Adapun perbedaan dari segi print:

Untuk Pyton 2 ketikkan menginput tidak menggunakan kurung biasa, namun pakek kurung juga bias dan dia menghasilkan atau mencetak satu baris. Sedangkan untuk python 3 harus menggunakan tanda kurung dan juga akan menghasilkan atau mencetak satu baris.\\

\subsubsection{Implementasi Python Di Perusahaan Dunia}

Ada beberapa perusahan terkenal dunia yang menggunakan bahasa Python, yaitu :\\
1. Google adalah perusahaan besar yang menggunakan banyak kode Python di dalam mesin pencarinya.\\
2. Youtube, situs video terbesar dan terpopuler di dunia, sebagian besar kodenya ditulis dalam bahasa Python.\\
3. Dropbox, menggunakan python baik di sisi server maupun di sisi pengguna layanannya.\\
4. ESRI, produsen terkenal pembuat software pemetaan GIS banyak menggunakan Python di produknya.\\
5. NASA, badan antariksa Amerika ini menggunakan Python untuk bidang sainsnya.
\section{Instalasi}
\subsubsection{Instalasi Python}

Cara Install Python di Windows\\
1. Unduh Python versi 3.7,\\
2. Buka file Python yang sudah di unduh,\\
3. Sebelum install pilih atau centang 'Add Python to PATH' di ujung kiri bawah
4. Pilih user, ada baiknya pilih 'Install For All Users' agak dapat di gunakan oleh semua user komputer,\\
5. Pilih Lokasi untuk menyimpan aplikasi Python,\\
6. Kemudian Next untuk melanjutkan,\\
7. Install untuk aplikasi Python Finish.
\subsubsection{Instalasi Anaconda}
Cara Install Python di Windows\\
1.Download aplikasi Anaconda di Windows, \\
2. Buka file Anaconda yang sudah di unduh,\\
3. Klik 'I Agree' pada perjanjian Lisensi,\\
4.Pilih user, ada baiknya pilih 'All Users' agak dapat di gunakan oleh semua user komputer,\\
5. Pilih Lokasi untuk menyimpan aplikasi Anaconda,\\
6. Ketika muncul dua pilihan, cukup centang 'Register Anaconda',\\
7.Install untuk Anaconda Complete.
\subsubsection{Script dan Intepreter Phyton}
Menggunakan Interpreter Python: \\
1. Jalankan interpreter. Buka Command Prompt, ketik python pada prompt dan tekan Enter. jika Python telah berhasil makan akan muncul tanda seperti ini (>>>).\\
2. Untuk menampilkan bantuan informasi kita dapat menggunakan perintah help()dan dapat dilakukan dengan 2 cara, yaitu dengan   help(int). Kedua adalah dengan mengetikan perintah help() didalam interpreter yang akan merubah mode interpreter ‘>>>’ menjadi mode ‘help>’.

\subsubsection{Pemakaian Spyder Variable explorer}


\section{Identitas}
\subsubsection{Identitas dan Cara Menanganinya}

\end{document}