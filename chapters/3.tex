\section{D. Irga B. Naufal Fakhri}
\subsection{Pemahaman Teori}
\begin{enumerate}
\item Fungsi

Fungsi adalah blok blok kode yang teroorganisir yang dapat digunakan kembali didalam program yang digunakan untuk melakukan suatu perintah yang telah diberikan.
untuk membuat fungsi kita harus menggunakan def kemudian nama fungsinya dan (variable)nya diakhiri oleh tanda :
\lstinputlisting[caption=Contoh kode fungsi inputan ke fungsi., firstline=296, lastline=301]{src/1174066.py}
Fungsi juga berguna untuk melemparkan variable contohnya
\lstinputlisting[caption=Contoh kode fungsi outputan ke fungsi., firstline=303, lastline=308]{src/1174066.py}

\item Paket(Package) atau Libary

Paket atau yang biasa disebut dengan library adalah kumpulan kode-kode fungsi atau method pada python yang dapat dipanggil kedalam program python yang kita buat. Package berada di file terpisah dari main program
cara memanggil package: Pastikan file package ada didalam folder yang sama lalu ditambah import dengan nama filenya tanpa extensi (.py)
\lstinputlisting[caption=Contoh import package atau library., firstline=311, lastline=314]{src/1174066.py}

\item Kelas (Class), Objek (Object), Atribut (Attribute), dan Method

Kelas(Class) adalah sebuah blueprint(cetakan) dari sebuah objek.
Objek(Object) adalah hasil cetakan dari sebuah kelas(class).
Atribut(Attribute) adalah nilai data yang ada didalam sebuah object.
Method adalah sesuatu yang bisa dilakukan oleh object.

\lstinputlisting[caption=Contoh import package atau library., firstline=316, lastline=328]{src/1174066.py}

\item Cara memanggil library dari instansiasi

Cara memanggilnya:
\begin{itemize}
	\item Pertama kita import filenya
	\item kemudian buat variablenya jika menggunakan variable untuk menampung data
	\item Kemudian panggil nama classnya(file) dan panggil fungsinya
	\item Kemudian menggunakan perintah print untuk menampilkan data
\end{itemize}
\lstinputlisting[caption=Contoh package atau library., firstline=6, lastline=9]{src/fungsi_1174066.py} 
\lstinputlisting[caption=Contoh import package atau library., firstline=331, lastline=336]{src/1174066.py}

\item  Contoh pemakaian paket dengan perintah from kalkulator import Penambahan 

Pemakaian package(paket) dengan perintah from namafilenya import berfungsi untuk memanggil fungsi dari nama filenya
\lstinputlisting[caption=Contoh import package atau library., firstline=339, lastline=344]{src/1174066.py}

\item Jelaskan dengan contoh kode, pemakaian paket fungsi didalam folder

Jika file paket ada didalam folder maka kita harus menambahkan lokasi filenya ada didalam folder apa dengan cara menggunakan namafolder.namafile
\lstinputlisting[caption=Contoh import package atau library didalam folder., firstline=346, lastline=351]{src/1174066.py}

\item Jelaskan dengan contoh kode, pemakaian paket fungsi didalam folder

Jika file paket ada didalam folder maka kita harus menambahkan lokasi filenya ada didalam folder apa dengan cara menggunakan namafolder.namafile
\lstinputlisting[caption=Contoh import package atau library didalam folder., firstline=346, lastline=351]{src/1174066.py}
\end{enumerate}

\subsection{Keterampilan Pemograman}
\begin{enumerate}
\item Jawaban nomor 1
\lstinputlisting[firstline=355, lastline=373]{src/1174066.py}

\item Jawaban nomor 2
\lstinputlisting[firstline=377, lastline=384]{src/1174066.py}

\item Jawaban nomor 3
\lstinputlisting[firstline=386, lastline=394]{src/1174066.py}

\item Jawaban nomor 4
\lstinputlisting[firstline=396, lastline=400]{src/1174066.py}

\item Jawaban nomor 5
\lstinputlisting[firstline=402, lastline=405]{src/1174066.py}

\item Jawaban nomor 6
\lstinputlisting[firstline=410, lastline=418]{src/1174066.py}

\item Jawaban nomor 7
\lstinputlisting[firstline=420, lastline=428]{src/1174066.py}

\item Jawaban nomor 8
\lstinputlisting[firstline=431, lastline=439]{src/1174066.py}

\item Jawaban nomor 9
\lstinputlisting[firstline=441, lastline=448]{src/1174066.py}

\item Jawaban nomor 10
\lstinputlisting[firstline=451, lastline=468]{src/1174066.py}

\item Jawaban nomor 11
\lstinputlisting[firstline=8, lastline=20]{src/main_1174066.py}

\item Jawaban nomor 12
\lstinputlisting[firstline=23, lastline=37]{src/main_1174066.py}
\end{enumerate}

\subsection{Ketrampilan Penanganan Error}
\begin{itemize}
\item Syntax Errors

Syntax Errors adalah kesalahan pada penulisan syntax atau kode. Solusinya adalah memperbaiki penulisan syntax atau kode

\item Zero Division Error

ZeroDivisonError adalah exceptions yang terjadi saat eksekusi program menghasilkan perhitungan matematika pembagian dengan angka nol (0). Solusinya adalah tidak membagi suatu yang hasilnya nol.

\item Name Error

NameError adalah exception saat kode melakukan eksekusi terhadap local name atau global name yang tidak terdefinisi atau tidak ada. Solusinya adalah memastikan variabel atau function yang akan dipanggil ada didalam program atau tidak salah mengetikannya.

\item Type Error

TypeError adalah exception saat melakukan eksekusi terhadap suatu operasi atau fungsi dengan type object yang tidak sesuai. Solusinya adalah mengkoversi varibelnya sesuai dengan tipe data sesuai dengan yang akan digunakan.
\end{itemize}
\lstinputlisting[firstline=23, lastline=37]{src/main_1174066.py}


%%%%%%%%%%%%%%%%%%%%%%%%%%%%%%%%%%%%%%%%%%%%%%%%%%%%%%%%%%%%%%%%%%%%%%%


\section{Nurul Izza Hamka | 1174062 | Teori}
\begin{enumerate}

\item Fungsi pada python menggunakan kata kunci def. setelah menulis kata def, 
kita tulis lagi nama fungsi kemudian diikuti dengan parameter yang diberi tanda kurung dan diakhirnya diberi tanda titi dua (:). 
Setelah itu kita menulis lagi fungsi yang akan di panggil untuk di jalankan.
Inputan fungsi adalah untuk memanggil fungsi dengan menuliskan nama fungsi.
Kembalian fungsi adalah keluaran fungsi dan untuk kembali ke baris selanjutnya untuk memanggil fungsi tadi.
\lstinputlisting[firstline=9, lastline=11]{src/1174062.py}

\item Paket adalah sebuah file yang berisi kode program python yang bisa digunakan berulang ketika sebuah paket itu dipanggil.
\lstinputlisting[firstline=17, lastline=18]{src/kelas3lib.py}

\item Apa itu kelas, apa itu objek, apa itu atribut, apa itu method  :
\begin{itemize}
\item Kelas adalah sebuah objek yang di dalamnya terdapat sebuah metode atau seperangkat atribut.\\
\item\ Objek adalah struktur data yang di definisikan dalam kelas, objek ini memiliki atribut dan juga aksi /behaviour.\\
\item Atribut adalah data dari variable kelas dan juga method.\\
\item Method adalah sebuah kode yang di gunakan untuk melakukan perintah.
\end{itemize}
\lstinputlisting[firstline=13, lastline=35]{src/1174062.py}

\item Pemanggikan library kelas dari instansiasi 
\begin{itemize}
\item Pertama, lakukan import file.\\
\item Membuat variable untuk menyimpan data.\\
\item Lakukan pemanggilan data dan class.\\
\item Ketikkan print untuk melihat hasilnya.\\
\end{itemize}
\lstinputlisting[firstline=38, lastline=43]{src/1174062.py}

\item Penggunaan paket from kalkulator import untuk penambahan berfungsi untuk memanggil file yang di masukkan dan juga fungsinya.
\lstinputlisting[firstline=38, lastline=43]{src/1174062.py}

\item Pemakaian paket fungsi adalah sebuah kumpulan fungsi-fungsi.

\item Pemakaian paket kelas apabila file library ada di dalam folder.
\lstinputlisting[firstline=20, lastline=20]{src/1174062.py}

%%%Pemrograman%%%
\item No1
\lstinputlisting[firstline=54, lastline=102]{src/1174062.py}

\item no2
\lstinputlisting[firstline=105, lastline=113]{src/1174062.py}

\item no3
\lstinputlisting[firstline=114, lastline=125]{src/1174062.py}

\item no4
\lstinputlisting[firstline=126, lastline=133]{src/1174062.py}

item no5
\lstinputlisting[firstline=134, lastline=140]{src/1174062.py}

\item no6
\lstinputlisting[firstline=141, lastline=149]{src/1174062.py}

\item no7
\lstinputlisting[firstline=150, lastline=158]{src/1174062.py}

\item no8
\lstinputlisting[firstline=159, lastline=165]{src/1174062.py}

\item no9
\lstinputlisting[firstline=167, lastline=173]{src/1174062.py}

\item no10
\lstinputlisting[firstline=175, lastline=189]{src/1174062.py}


\end{enumerate}

%%%%%%%%%%%%%%%%%%%%%%%%%%%%%%%%%%%%%%%%%%%%%%%%%%%%%%%%%%%%%
\section{Fanny Shafira Damayanti | 1174069}
\section{Teori}
\begin{enumerate}

\item fungsi adalah sebuah program untuk melakukan tugas tertentu secara berulang. Fungsi di dalam Python di tandai dengan def, yang artinya definition.
Inputan fungsi berarti memanggil fungsi yang ditulis oleh user dan dikembalikannya dalam bentuk string.
\lstinputlisting[firstline=8, lastline=11]{src/1174069/1174069.py}

\item paket adalah kumpulan fungsi yang siap untuk di pakai. Paket digunakan untuk memudahkan programmer agar tidak menuliskan kembali kode-kodenya.
\lstinputlisting[firstline=25, lastline=26]{src/1174069/1174069.py}

\item objek adalah instansi dari kelas. Kelas adalah cetakan biru yang berisikan variable dan method. Method adalah fungsi dari suatu objek. Atribut adalah nilai dari suatu objek.
\lstinputlisting[firstline=29, lastline=51]{src/1174069/1174069.py}

\item cara pemanggilan libarary yaitu :

\begin{itemize}

\item Import file yang akan di panggil
\item Buat variable nya
\item Panggil nama class dan methodnya
\item Print untuk menampilkan outputnya

\lstinputlisting[firstline=54, lastline=56]{src/1174069/1174069.py}
\end{itemize}

\item import berfungsi untuk memanngil fungsi dari kelas lain.
\lstinputlisting[firstline=30, lastline=31]{src/1174069/1174069.py}

\item pemakaian paket fungsi apabila ke library ada di dalam folder
\lstinputlisting[firstline=8, lastline=8]{src/1174069/main_1174069.py}

\item pemakaian paket kelas apabila ke library ada di dalam folder
\lstinputlisting[firstline=9, lastline=9]{src/1174069/main_1174069.py}
\end{enumerate}

\section{keterampilan pemrograman}
\begin{enumerate}

\item No 1
\lstinputlisting[firstline=62, lastline=110]{src/1174069/1174069.py}

\item N0 2
\lstinputlisting[firstline=113, lastline=117]{src/1174069/1174069.py}

\item N0 3
\lstinputlisting[firstline=122, lastline=131]{src/1174069/1174069.py}

\item N0 4
\lstinputlisting[firstline=134, lastline=139]{src/1174069/1174069.py}

\item N0 5
\lstinputlisting[firstline=142, lastline=144]{src/1174069/1174069.py}

\item N0 6
\lstinputlisting[firstline=149, lastline=153]{src/1174069/1174069.py}

\item N0 7
\lstinputlisting[firstline=158, lastline=164]{src/1174069/1174069.py}

\item N0 8
\lstinputlisting[firstline=167, lastline=172]{src/1174069/1174069.py}

\item N0 9
\lstinputlisting[firstline=175, lastline=180]{src/1174069/1174069.py}

\item N0 10
\lstinputlisting[firstline=183, lastline=196]{src/1174069/1174069.py}

\item N0 11
\lstinputlisting[firstline=8, lastline=8]{src/1174069/main_1174069.py}

\item N0 12
\lstinputlisting[firstline=9, lastline=9]{src/1174069/main_1174069.py}
\end{enumerate}

\section{penanganan Error}
\lstinputlisting[firstline=199, lastline=204]{src/1174069/1174069.py}

%%%%%%%%%%%%%%%%%%%%%%%%%%%%%%%%%%%%%%%%%%%%%%%%%%%%%%%%%%
\section{1174054 | Aulyardha Anindita}
\section{Pemahaman Teori}
\subsection{Fungsi}
Fungsi adalah bagian dari suatu sub program yang terdiri dari nama fungsi itu sendiri dan variabel yang dapat digunakan ulang dan nama tersebut dapat dipanggil dimanapun dalam suatu program. Fungsi dalam python menggunakan kata kunci 'def'. Dan setelah 'def' biasanya terdapat nama pengenal fungsi yang diikuti oleh parameter yang diapit oleh tanda kurung dan diakhiri dengan tanda titik dua (:). Dan baris berikutnya adalah blok fungsi yang akan dijalankan jika fungsi dipanggil.\\
Contoh fungsi :
\lstinputlisting[firstline=8, lastline=13]{src/1174054/1174054.py}

Inputan fungsi adalah memanggil fungsi dengan fungsi yang telah dibuat dari inputan user dan mengembalikannya dalam bentuk string. \\
Contoh :
\lstinputlisting[firstline=14, lastline=19]{src/1174054/1174054.py}

Kembalian fungsi adalah keluar dari suatu fungsi dan kembali ke baris selanjutnya dimana suatu fungsi dipanggil.\\
Contoh :
\lstinputlisting[firstline=20, lastline=29]{src/1174054/1174054.py}

\subsection{Paket}
\paragraph{}
Paket adalah suatu teknik pengumpulan dari beberapa file-file modul. Paket memudahkan programmer dalam mengelompokkan dan mengorganisasikan modul yang telah dibuat. Ringkasnya, kita tidak perlu membuat script untuk beberapa kasus, namun kita bisa mengelompokkannya dalam 1 file tiap kasus dan memanggilnya dalam satu program.

Cara memanggil paket adalah menggunakan kata kunci ‘import’ untuk mengimport file yang telah dibuat.\\
Contoh paket :
\lstinputlisting[firstline=30, lastline=33]{src/1174054/1174054.py}

\subsection{Kelas, Objek, Atribut, dan Method}
\paragraph{}
Kelas adalah suatu entitas atau struktur data yang biasa digunakan yang terdiri dari objek, atribut, dan method didalamnya. Dengan kata lain kelas adalah sebuah cetak biru atau blueprint dari sebuah objek (instans)

Objek adalah suatu entitas yang biasanya memiliki variabel dan method didalamnya dengan kata lain memiliki keadaan (state) dan kelakukan (behavior).

Atribut adalah suatu entitas atau berupa fungsi-fungsi yang dimiliki oleh kelas atau objek. Biasanya atribut berisi variabel-variabel yang telah dideklarasikan.

Method adalah suatu fungsi yang melekat pada sebuah objek atau instan kelas untuk merepresentasikan suatu behavior (kelakuan).\\
Contoh :
\lstinputlisting[firstline=34, lastline=58]{src/1174054/1174054.py}

\subsection{Cara Pemanggilan Library Kelas dari Instansiansi}
Instansiansi adalah suatu pembuatan instance atau objek dari suatu kelas. Untuk memanggil nama kelas yaitu dengan menggunakan fungsi init() pada saat kita mendefinisikannya.\\
Contoh :
\lstinputlisting[firstline=59, lastline=66]{src/1174054/1174054.py}

\subsection{Pemakaian Paket dengan Perintah from dan import}
\lstinputlisting[firstline=30, lastline=33]{src/1174054/1174054.py}

\subsection{Pemakaian Paket Fungsi}
\lstinputlisting[firstline=8, lastline=8]{src/1174054/main_1174054.py}

\subsection{Pemakaian Paket Kelas}
\lstinputlisting[firstline=9, lastline=9]{src/1174054/main_1174054.py}

\section{Keterampilan Pemrograman}
\begin{enumerate}
	\item Jawaban Soal No.1
	\lstinputlisting[firstline=71, lastline=121]{src/1174054/1174054.py}
	
	\item Jawaban Soal No.2
	\lstinputlisting[firstline=122, lastline=130]{src/1174054/1174054.py}
	
	\item Jawaban Soal No.3
	\lstinputlisting[firstline=131, lastline=142]{src/1174054/1174054.py}
	
	\item Jawaban Soal No.4
	\lstinputlisting[firstline=143, lastline=150]{src/1174054/1174054.py}
	
	\item Jawaban Soal No.5
	\lstinputlisting[firstline=151, lastline=157]{src/1174054/1174054.py}
	
	\item Jawaban Soal No.6
	\lstinputlisting[firstline=158, lastline=166]{src/1174054/1174054.py}
	
	\item Jawaban Soal No.7
	\lstinputlisting[firstline=167, lastline=175]{src/1174054/1174054.py}
	
	\item Jawaban Soal No.8
	\lstinputlisting[firstline=176, lastline=183]{src/1174054/1174054.py}
	
	\item Jawaban Soal No.9
	\lstinputlisting[firstline=184, lastline=191]{src/1174054/1174054.py}
	
	\item Jawaban Soal No.10
	\lstinputlisting[firstline=192, lastline=207]{src/1174054/1174054.py}
	
	\item Jawaban Soal No.11
	\lstinputlisting[firstline=8, lastline=8]{src/1174054/main_1174054.py}
	
	\item Jawaban Soal No.12
	\lstinputlisting[firstline=9, lastline=9]{src/1174054/main_1174054.py}
\end{enumerate}

\section{Penanganan Error}
\lstinputlisting[firstline=14, lastline=19]{src/1174054/1174054.py}

\section{Dini Permata Putri}


1. Apa itu fungsi, inputan fugsi dan kembalian fungsi dengan contoh kode program lainnya.\\
fungsi / function adalah satu blok kode yang melakukan tugas tertentu atau satu blok kode yang melakukan tugas tertentu atau satu blok instruksi yang eksekusi ketika dipanggil dari bagian lain dalam seuatu program. tujuan pembuatan fungsi adalah : memudahkan dalam pembuatan program.\\
contoh kodenya :\\
def function\_name(parameters):\\
	"""function\_docstring"""\\
	statement(s)\\
	return [expression]\\
	
2. Apa itu paket dan cara pemanggilan paket atau library dengan contoh kode program lainnya.\\
paket digunakan untuk mengelompokkan kelas-kelas yang mempunyai kemiripan fungsi (related class). kelas-kelas java yang akan digunakan didalam program, terlebih dahulu harus diimpor beserta dengan nama paket dimana kelas tersebut berada.\\
cara memanggilnya :\\
penampung = namaClass()\\
penampung.namaMetode()\\

3. Jelaskan apa itu kelas, apa itu objek, apa itu atribut, apa itu method dan contoh kode program lainnya masing-masing.\\
- class adalah salah satu cara bagaimana kita membuuat sebuah kode yang mempunyai behaviour tertentu dan lebih mudah dalam mengprganisasi berbagai fungsi dan state-nya. dalam sebuah class kamu dapat menyimpan sebuah state tanpa harus membuat banyak state bila tidak menggunakan class.\\
contohnya : \\
Class Product:\\
	vendor.message = "Ini adalah rahasia"\\
	name = ""\\
	price = ""\\
	size = ""\\
	unit = ""\\
	
	def.init\_(self, name):\\
		print "ini adalah consuctor"\\
		self.name = name\\
		self.unit = "ml"\\
		self.size = 250\\
		
	def get.vendor.message(self):\\
		print self.vendor.message\\
		
	def set\_price(Self, price):\\
		self.price = price\\
		
- objek adalah instansi atau perwujudan dari sebuah kelas. bila kelas adalah prototypenya, dan objek adalah barang jadinya.\\
contohnya :\\
obj=Karyawan("K001", "Dini", "Teknisi")\\
obj.infoKaryawan()\\

\#tambah karyawan baru\\
obj2=Karyawan("K002", "Ayu", "Akunting")\\
obj2.infoKaryawan()

\#tampilkan total karyawan\\
print "Total Karyawan " %d " % Karyawan.jml_karyawan\\

- atribut adalah instance spesifik untuk setiap objek, atribut class sama untuk semua contoh -- yang dalam hal ini adalah semua dog.\\
contohnya :\\
class Dog:\\
\#class Attribute\\
	species = 'mammal'\\
\#Initializer / Instance Attributes\\
	def.init(self, name, age):\\
		self.name = name\\
		self.age = age\\
jadi, sementara setiap Dog memiliki nama dan umur yang unik, setiap Dog akan menjadi mamalia.\\

\-Method digunakan untuk melakukan operasi dengan atribut objek, seperti init metodenya, argumen pertama selalu self:\\
contohnya :\\
class dog:\\
\# instance method
	def description(self):\\
		return "() is () years old".format(self.name, self.age)\\
\# instance method\\
	def speak(self, sound):\\
		return "() says ()".format(self.name, sound)\\
\#instantiate the Dog object\\
mikey = Dog("Mikey", 6)\\
\#call our instance methods\\
print(mikey.description())\\
print(mikey.speak("Gruff Gruff"))\\

4. Jelaskan cara pemanggilan library kelas dari instansiasi dan pemakaiannya dengan contoh program lainnya.\\
library adaalh salah satu cara yang paling efisien untu menghemat waktu ketika membangun aplikasi. untuk memanggil file library digunakan perintah include() atau require(), selain include() dan require() ada juga include.once() dan require.once(), perbedaaannya adalah jika suatu include ke suatu file dilakukan selama lebih dari 1 kali dalam suatu file, maka akan menghasilkan eror karena dianggapnya ada pendklarasian ulang(redeclare), tetapi jika menggunakan include.once() atau require.once() maka kejadian tersebut dihindaari.\\
contohnya :\\
\$aoutoload\['libraries'] = array('form_validation','database','session');\\
\$this load library ('nama.library')\\
$$

5.Jelaskan dengan contoh pemakaian paket denga  perintah from kalkulator import penambahan disertai dengan contoh kode lainnya.\\
paket dengan perintah from kalkuator import penambahan pertama yaitu tentukan nama fungsi, variabel, dan inputannya apa saja, setiap penuliskan harus menggunakaan () dan : dan identasi (jaraknya harus sama)\\
contoh :\\
def penambahan (a+b):\\
r=(a+b)\\
return\\
a=5\\
b=6\\
anu=penambahan(a,b)

6. Jelaskan dengan contoh kodenya, pemakaian paket fungsi apabila file library ada di dalam folder.\\
def tambah(bil1,bil2):\\
	total = bil1+bil2\\
	return total\\

def kurang(bil1,bil2):\\
	total = bil1-bil2\\
	return total\\
	
def kali(bil1,bil2):\\
	total = bil1*bil2\\
	return total\\
	
def bagi(bil1,bil2):\\
	total = bil1/bil2\\
	return total\\
	
def nilai(n1,n2):\\
	hasil = n1+n2\\
	if 80 <= hasil <= 100:\\
		print("nilai anda adalah A")\\
	elif 70 <= hasil <80:\\
		print("nilai anda adalah B")\\
	elif 60 <= hasil <70:\\
		print("nilai anda adalah C")\\
	elif 50 <= hasil <60:\\
		print("nilai anda adalah D, SIlahkan Mengulang kembali")\\
	else:
		print("anda GAGAL")\\
		
7.Jelaskan dengan contoh kodenya, pemakaian paket kelas apabila file library ada di dalam folder\\
mendefinisikan sebuah class dengan menggunakan kata kunci class diikuti oleh nama class tersebut.\\
class ClassName:
	'''class docstring'''\\
	class.body\\
class memiliki docstring atau string dokumentasi yang bersifat opsional artinya bisa atau tidak. Docstring bisa diakses menggunakan format\\
ClassName.doc

