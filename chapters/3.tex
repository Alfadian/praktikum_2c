\section{D. Irga B. Naufal Fakhri}
\subsection{Pemahaman Teori}
\begin{enumerate}
\item Fungsi

Fungsi adalah blok blok kode yang teroorganisir yang dapat digunakan kembali didalam program yang digunakan untuk melakukan suatu perintah yang telah diberikan.
untuk membuat fungsi kita harus menggunakan def kemudian nama fungsinya dan (variable)nya diakhiri oleh tanda :
\lstinputlisting[caption=Contoh kode fungsi inputan ke fungsi., firstline=296, lastline=301]{src/1174066.py}
Fungsi juga berguna untuk melemparkan variable contohnya
\lstinputlisting[caption=Contoh kode fungsi outputan ke fungsi., firstline=303, lastline=308]{src/1174066.py}

\item Paket(Package) atau Libary

Paket atau yang biasa disebut dengan library adalah kumpulan kode-kode fungsi atau method pada python yang dapat dipanggil kedalam program python yang kita buat. Package berada di file terpisah dari main program
cara memanggil package: Pastikan file package ada didalam folder yang sama lalu ditambah import dengan nama filenya tanpa extensi (.py)
\lstinputlisting[caption=Contoh import package atau library., firstline=311, lastline=314]{src/1174066.py}

\item Kelas (Class), Objek (Object), Atribut (Attribute), dan Method

Kelas(Class) adalah sebuah blueprint(cetakan) dari sebuah objek.
Objek(Object) adalah hasil cetakan dari sebuah kelas(class).
Atribut(Attribute) adalah nilai data yang ada didalam sebuah object.
Method adalah sesuatu yang bisa dilakukan oleh object.

\lstinputlisting[caption=Contoh import package atau library., firstline=316, lastline=328]{src/1174066.py}

\item Cara memanggil library dari instansiasi

Cara memanggilnya:
\begin{itemize}
	\item Pertama kita import filenya
	\item kemudian buat variablenya jika menggunakan variable untuk menampung data
	\item Kemudian panggil nama classnya(file) dan panggil fungsinya
	\item Kemudian menggunakan perintah print untuk menampilkan data
\end{itemize}
\lstinputlisting[caption=Contoh package atau library., firstline=6, lastline=9]{src/fungsi_1174066.py} 
\lstinputlisting[caption=Contoh import package atau library., firstline=331, lastline=336]{src/1174066.py}

\item  Contoh pemakaian paket dengan perintah from kalkulator import Penambahan 

Pemakaian package(paket) dengan perintah from namafilenya import berfungsi untuk memanggil fungsi dari nama filenya
\lstinputlisting[caption=Contoh import package atau library., firstline=339, lastline=344]{src/1174066.py}

\item Jelaskan dengan contoh kode, pemakaian paket fungsi didalam folder

Jika file paket ada didalam folder maka kita harus menambahkan lokasi filenya ada didalam folder apa dengan cara menggunakan namafolder.namafile
\lstinputlisting[caption=Contoh import package atau library didalam folder., firstline=346, lastline=351]{src/1174066.py}

\item Jelaskan dengan contoh kode, pemakaian paket fungsi didalam folder

Jika file paket ada didalam folder maka kita harus menambahkan lokasi filenya ada didalam folder apa dengan cara menggunakan namafolder.namafile
\lstinputlisting[caption=Contoh import package atau library didalam folder., firstline=346, lastline=351]{src/1174066.py}
\end{enumerate}

\subsection{Keterampilan Pemograman}
\begin{enumerate}
\item Jawaban nomor 1
\lstinputlisting[firstline=355, lastline=373]{src/1174066.py}

\item Jawaban nomor 2
\lstinputlisting[firstline=377, lastline=384]{src/1174066.py}

\item Jawaban nomor 3
\lstinputlisting[firstline=386, lastline=394]{src/1174066.py}

\item Jawaban nomor 4
\lstinputlisting[firstline=396, lastline=400]{src/1174066.py}

\item Jawaban nomor 5
\lstinputlisting[firstline=402, lastline=405]{src/1174066.py}

\item Jawaban nomor 6
\lstinputlisting[firstline=410, lastline=418]{src/1174066.py}

\item Jawaban nomor 7
\lstinputlisting[firstline=420, lastline=428]{src/1174066.py}

\item Jawaban nomor 8
\lstinputlisting[firstline=431, lastline=439]{src/1174066.py}

\item Jawaban nomor 9
\lstinputlisting[firstline=441, lastline=448]{src/1174066.py}

\item Jawaban nomor 10
\lstinputlisting[firstline=451, lastline=468]{src/1174066.py}

\item Jawaban nomor 11
\lstinputlisting[firstline=8, lastline=20]{src/main_1174066.py}

\item Jawaban nomor 12
\lstinputlisting[firstline=23, lastline=37]{src/main_1174066.py}
\end{enumerate}

\subsection{Ketrampilan Penanganan Error}
\begin{itemize}
\item Syntax Errors

Syntax Errors adalah kesalahan pada penulisan syntax atau kode. Solusinya adalah memperbaiki penulisan syntax atau kode

\item Zero Division Error

ZeroDivisonError adalah exceptions yang terjadi saat eksekusi program menghasilkan perhitungan matematika pembagian dengan angka nol (0). Solusinya adalah tidak membagi suatu yang hasilnya nol.

\item Name Error

NameError adalah exception saat kode melakukan eksekusi terhadap local name atau global name yang tidak terdefinisi atau tidak ada. Solusinya adalah memastikan variabel atau function yang akan dipanggil ada didalam program atau tidak salah mengetikannya.

\item Type Error

TypeError adalah exception saat melakukan eksekusi terhadap suatu operasi atau fungsi dengan type object yang tidak sesuai. Solusinya adalah mengkoversi varibelnya sesuai dengan tipe data sesuai dengan yang akan digunakan.
\end{itemize}
\lstinputlisting[firstline=23, lastline=37]{src/main_1174066.py}


%%%%%%%%%%%%%%%%%%%%%%%%%%%%%%%%%%%%%%%%%%%%%%%%%%%%%%%%%%%%%%%%%%%%%%%


\section{Nurul Izza Hamka | 1174062 | Teori}
\begin{enumerate}

\item Fungsi pada python menggunakan kata kunci def. setelah menulis kata def, 
kita tulis lagi nama fungsi kemudian diikuti dengan parameter yang diberi tanda kurung dan diakhirnya diberi tanda titi dua (:). 
Setelah itu kita menulis lagi fungsi yang akan di panggil untuk di jalankan.
Inputan fungsi adalah untuk memanggil fungsi dengan menuliskan nama fungsi.
Kembalian fungsi adalah keluaran fungsi dan untuk kembali ke baris selanjutnya untuk memanggil fungsi tadi.
\lstinputlisting[firstline=9, lastline=11]{src/1174062.py}

\item Paket adalah sebuah file yang berisi kode program python yang bisa digunakan berulang ketika sebuah paket itu dipanggil.
\lstinputlisting[firstline=17, lastline=18]{src/kelas3lib.py}

\item Apa itu kelas, apa itu objek, apa itu atribut, apa itu method  :
\begin{itemize}
\item Kelas adalah sebuah objek yang di dalamnya terdapat sebuah metode atau seperangkat atribut.\\
\item\ Objek adalah struktur data yang di definisikan dalam kelas, objek ini memiliki atribut dan juga aksi /behaviour.\\
\item Atribut adalah data dari variable kelas dan juga method.\\
\item Method adalah sebuah kode yang di gunakan untuk melakukan perintah.
\end{itemize}
\lstinputlisting[firstline=13, lastline=35]{src/1174062.py}

\item Pemanggikan library kelas dari instansiasi 
\begin{itemize}
\item Pertama, lakukan import file.\\
\item Membuat variable untuk menyimpan data.\\
\item Lakukan pemanggilan data dan class.\\
\item Ketikkan print untuk melihat hasilnya.\\
\end{itemize}
\lstinputlisting[firstline=38, lastline=43]{src/1174062.py}

\item Penggunaan paket from kalkulator import untuk penambahan berfungsi untuk memanggil file yang di masukkan dan juga fungsinya.
\lstinputlisting[firstline=38, lastline=43]{src/1174062.py}

\item Pemakaian paket fungsi adalah sebuah kumpulan fungsi-fungsi.

\item Pemakaian paket kelas apabila file library ada di dalam folder.
\lstinputlisting[firstline=20, lastline=20]{src/1174062.py}

%%%Pemrograman%%%
\item No1
\lstinputlisting[firstline=54, lastline=102]{src/1174062.py}

\item no2
\lstinputlisting[firstline=105, lastline=113]{src/1174062.py}

\item no3
\lstinputlisting[firstline=114, lastline=125]{src/1174062.py}

\item no4
\lstinputlisting[firstline=126, lastline=133]{src/1174062.py}

item no5
\lstinputlisting[firstline=134, lastline=140]{src/1174062.py}

\item no6
\lstinputlisting[firstline=141, lastline=149]{src/1174062.py}

\item no7
\lstinputlisting[firstline=150, lastline=158]{src/1174062.py}

\item no8
\lstinputlisting[firstline=159, lastline=165]{src/1174062.py}

\item no9
\lstinputlisting[firstline=167, lastline=173]{src/1174062.py}

\item no10
\lstinputlisting[firstline=175, lastline=189]{src/1174062.py}


\end{enumerate}

%%%%%%%%%%%%%%%%%%%%%%%%%%%%%%%%%%%%%%%%%%%%%%%%%%%%%%%%%%%%%
\section{Fanny Shafira Damayanti | 1174069}
\section{Teori}
\begin{enumerate}

\item fungsi adalah sebuah program untuk melakukan tugas tertentu secara berulang. Fungsi di dalam Python di tandai dengan def, yang artinya definition.
Inputan fungsi berarti memanggil fungsi yang ditulis oleh user dan dikembalikannya dalam bentuk string.
\lstinputlisting[firstline=8, lastline=11]{src/1174069/1174069.py}

\item paket adalah kumpulan fungsi yang siap untuk di pakai. Paket digunakan untuk memudahkan programmer agar tidak menuliskan kembali kode-kodenya.
\lstinputlisting[firstline=25, lastline=26]{src/1174069/1174069.py}

\item objek adalah instansi dari kelas. Kelas adalah cetakan biru yang berisikan variable dan method. Method adalah fungsi dari suatu objek. Atribut adalah nilai dari suatu objek.
\lstinputlisting[firstline=29, lastline=51]{src/1174069/1174069.py}

\item cara pemanggilan libarary yaitu :

\begin{itemize}

\item Import file yang akan di panggil
\item Buat variable nya
\item Panggil nama class dan methodnya
\item Print untuk menampilkan outputnya

\lstinputlisting[firstline=54, lastline=56]{src/1174069/1174069.py}
\end{itemize}

\item import berfungsi untuk memanngil fungsi dari kelas lain.
\lstinputlisting[firstline=30, lastline=31]{src/1174069/1174069.py}

\item pemakaian paket fungsi apabila ke library ada di dalam folder
\lstinputlisting[firstline=8, lastline=8]{src/1174069/main_1174069.py}

\item pemakaian paket kelas apabila ke library ada di dalam folder
\lstinputlisting[firstline=9, lastline=9]{src/1174069/main_1174069.py}
\end{enumerate}

\section{keterampilan pemrograman}
\begin{enumerate}

\item No 1
\lstinputlisting[firstline=62, lastline=110]{src/1174069/1174069.py}

\item N0 2
\lstinputlisting[firstline=113, lastline=117]{src/1174069/1174069.py}

\item N0 3
\lstinputlisting[firstline=122, lastline=131]{src/1174069/1174069.py}

\item N0 4
\lstinputlisting[firstline=134, lastline=139]{src/1174069/1174069.py}

\item N0 5
\lstinputlisting[firstline=142, lastline=144]{src/1174069/1174069.py}

\item N0 6
\lstinputlisting[firstline=149, lastline=153]{src/1174069/1174069.py}

\item N0 7
\lstinputlisting[firstline=158, lastline=164]{src/1174069/1174069.py}

\item N0 8
\lstinputlisting[firstline=167, lastline=172]{src/1174069/1174069.py}

\item N0 9
\lstinputlisting[firstline=175, lastline=180]{src/1174069/1174069.py}

\item N0 10
\lstinputlisting[firstline=183, lastline=196]{src/1174069/1174069.py}

\item N0 11
\lstinputlisting[firstline=8, lastline=8]{src/1174069/main_1174069.py}

\item N0 12
\lstinputlisting[firstline=9, lastline=9]{src/1174069/main_1174069.py}
\end{enumerate}

\section{penanganan Error}
\lstinputlisting[firstline=199, lastline=204]{src/1174069/1174069.py}

%%%%%%%%%%%%%%%%%%%%%%%%%%%%%%%%%%%%%%%%%%%%%%%%%%%%%%%%%%
\section{1174054 | Aulyardha Anindita}
\section{Pemahaman Teori}
\subsection{Fungsi}
Fungsi adalah bagian dari suatu sub program yang terdiri dari nama fungsi itu sendiri dan variabel yang dapat digunakan ulang dan nama tersebut dapat dipanggil dimanapun dalam suatu program. Fungsi dalam python menggunakan kata kunci 'def'. Dan setelah 'def' biasanya terdapat nama pengenal fungsi yang diikuti oleh parameter yang diapit oleh tanda kurung dan diakhiri dengan tanda titik dua (:). Dan baris berikutnya adalah blok fungsi yang akan dijalankan jika fungsi dipanggil.\\
Contoh fungsi :
\lstinputlisting[firstline=8, lastline=13]{src/1174054/1174054.py}

Inputan fungsi adalah memanggil fungsi dengan fungsi yang telah dibuat dari inputan user dan mengembalikannya dalam bentuk string. \\
Contoh :
\lstinputlisting[firstline=14, lastline=19]{src/1174054/1174054.py}

Kembalian fungsi adalah keluar dari suatu fungsi dan kembali ke baris selanjutnya dimana suatu fungsi dipanggil.\\
Contoh :
\lstinputlisting[firstline=20, lastline=29]{src/1174054/1174054.py}

\subsection{Paket}
\paragraph{}
Paket adalah suatu teknik pengumpulan dari beberapa file-file modul. Paket memudahkan programmer dalam mengelompokkan dan mengorganisasikan modul yang telah dibuat. Ringkasnya, kita tidak perlu membuat script untuk beberapa kasus, namun kita bisa mengelompokkannya dalam 1 file tiap kasus dan memanggilnya dalam satu program.

Cara memanggil paket adalah menggunakan kata kunci ‘import’ untuk mengimport file yang telah dibuat.\\
Contoh paket :
\lstinputlisting[firstline=30, lastline=33]{src/1174054/1174054.py}

\subsection{Kelas, Objek, Atribut, dan Method}
\paragraph{}
Kelas adalah suatu entitas atau struktur data yang biasa digunakan yang terdiri dari objek, atribut, dan method didalamnya. Dengan kata lain kelas adalah sebuah cetak biru atau blueprint dari sebuah objek (instans)

Objek adalah suatu entitas yang biasanya memiliki variabel dan method didalamnya dengan kata lain memiliki keadaan (state) dan kelakukan (behavior).

Atribut adalah suatu entitas atau berupa fungsi-fungsi yang dimiliki oleh kelas atau objek. Biasanya atribut berisi variabel-variabel yang telah dideklarasikan.

Method adalah suatu fungsi yang melekat pada sebuah objek atau instan kelas untuk merepresentasikan suatu behavior (kelakuan).\\
Contoh :
\lstinputlisting[firstline=34, lastline=58]{src/1174054/1174054.py}

\subsection{Cara Pemanggilan Library Kelas dari Instansiansi}
Instansiansi adalah suatu pembuatan instance atau objek dari suatu kelas. Untuk memanggil nama kelas yaitu dengan menggunakan fungsi init() pada saat kita mendefinisikannya.\\
Contoh :
\lstinputlisting[firstline=59, lastline=66]{src/1174054/1174054.py}

\subsection{Pemakaian Paket dengan Perintah from dan import}
\lstinputlisting[firstline=30, lastline=33]{src/1174054/1174054.py}

\subsection{Pemakaian Paket Fungsi}
\lstinputlisting[firstline=8, lastline=8]{src/1174054/main_1174054.py}

\subsection{Pemakaian Paket Kelas}
\lstinputlisting[firstline=9, lastline=9]{src/1174054/main_1174054.py}

\section{Keterampilan Pemrograman}
\begin{enumerate}
	\item Jawaban Soal No.1
	\lstinputlisting[firstline=71, lastline=121]{src/1174054/1174054.py}
	
	\item Jawaban Soal No.2
	\lstinputlisting[firstline=122, lastline=130]{src/1174054/1174054.py}
	
	\item Jawaban Soal No.3
	\lstinputlisting[firstline=131, lastline=142]{src/1174054/1174054.py}
	
	\item Jawaban Soal No.4
	\lstinputlisting[firstline=143, lastline=150]{src/1174054/1174054.py}
	
	\item Jawaban Soal No.5
	\lstinputlisting[firstline=151, lastline=157]{src/1174054/1174054.py}
	
	\item Jawaban Soal No.6
	\lstinputlisting[firstline=158, lastline=166]{src/1174054/1174054.py}
	
	\item Jawaban Soal No.7
	\lstinputlisting[firstline=167, lastline=175]{src/1174054/1174054.py}
	
	\item Jawaban Soal No.8
	\lstinputlisting[firstline=176, lastline=183]{src/1174054/1174054.py}
	
	\item Jawaban Soal No.9
	\lstinputlisting[firstline=184, lastline=191]{src/1174054/1174054.py}
	
	\item Jawaban Soal No.10
	\lstinputlisting[firstline=192, lastline=207]{src/1174054/1174054.py}
	
	\item Jawaban Soal No.11
	\lstinputlisting[firstline=8, lastline=8]{src/1174054/main_1174054.py}
	
	\item Jawaban Soal No.12
	\lstinputlisting[firstline=9, lastline=9]{src/1174054/main_1174054.py}
\end{enumerate}

\section{Penanganan Error}
\lstinputlisting[firstline=14, lastline=19]{src/1174054/1174054.py}

\section{Dini Permata Putri}


1. Apa itu fungsi, inputan fugsi dan kembalian fungsi dengan contoh kode program lainnya.\\
fungsi / function adalah satu blok kode yang melakukan tugas tertentu atau satu blok kode yang melakukan tugas tertentu atau satu blok instruksi yang eksekusi ketika dipanggil dari bagian lain dalam seuatu program. tujuan pembuatan fungsi adalah : memudahkan dalam pembuatan program.\\
contoh kodenya :\\
def function\_name(parameters):\\
	"""function\_docstring"""\\
	statement(s)\\
	return [expression]\\
	
2. Apa itu paket dan cara pemanggilan paket atau library dengan contoh kode program lainnya.\\
paket digunakan untuk mengelompokkan kelas-kelas yang mempunyai kemiripan fungsi (related class). kelas-kelas java yang akan digunakan didalam program, terlebih dahulu harus diimpor beserta dengan nama paket dimana kelas tersebut berada.\\
cara memanggilnya :\\
penampung = namaClass()\\
penampung.namaMetode()\\

3. Jelaskan apa itu kelas, apa itu objek, apa itu atribut, apa itu method dan contoh kode program lainnya masing-masing.\\
- class adalah salah satu cara bagaimana kita membuuat sebuah kode yang mempunyai behaviour tertentu dan lebih mudah dalam mengprganisasi berbagai fungsi dan state-nya. dalam sebuah class kamu dapat menyimpan sebuah state tanpa harus membuat banyak state bila tidak menggunakan class.\\
contohnya : \\
Class Product:\\
	vendor.message = "Ini adalah rahasia"\\
	name = ""\\
	price = ""\\
	size = ""\\
	unit = ""\\
	
	def.init\_(self, name):\\
		print "ini adalah consuctor"\\
		self.name = name\\
		self.unit = "ml"\\
		self.size = 250\\
		
	def get.vendor.message(self):\\
		print self.vendor.message\\
		
	def set\_price(Self, price):\\
		self.price = price\\
		
- objek adalah instansi atau perwujudan dari sebuah kelas. bila kelas adalah prototypenya, dan objek adalah barang jadinya.\\
contohnya :\\
obj=Karyawan("K001", "Dini", "Teknisi")\\
obj.infoKaryawan()\\

\#tambah karyawan baru\\
obj2=Karyawan("K002", "Ayu", "Akunting")\\
obj2.infoKaryawan()

\#tampilkan total karyawan\\
print "Total Karyawan " %d " % Karyawan.jml_karyawan\\

- atribut adalah instance spesifik untuk setiap objek, atribut class sama untuk semua contoh -- yang dalam hal ini adalah semua dog.\\
contohnya :\\
class Dog:\\
\#class Attribute\\
	species = 'mammal'\\
\#Initializer / Instance Attributes\\
	def.init(self, name, age):\\
		self.name = name\\
		self.age = age\\
jadi, sementara setiap Dog memiliki nama dan umur yang unik, setiap Dog akan menjadi mamalia.\\

\-Method digunakan untuk melakukan operasi dengan atribut objek, seperti init metodenya, argumen pertama selalu self:\\
contohnya :\\
class dog:\\
\# instance method
	def description(self):\\
		return "() is () years old".format(self.name, self.age)\\
\# instance method\\
	def speak(self, sound):\\
		return "() says ()".format(self.name, sound)\\
\#instantiate the Dog object\\
mikey = Dog("Mikey", 6)\\
\#call our instance methods\\
print(mikey.description())\\
print(mikey.speak("Gruff Gruff"))\\

4. Jelaskan cara pemanggilan library kelas dari instansiasi dan pemakaiannya dengan contoh program lainnya.\\
library adaalh salah satu cara yang paling efisien untu menghemat waktu ketika membangun aplikasi. untuk memanggil file library digunakan perintah include() atau require(), selain include() dan require() ada juga include.once() dan require.once(), perbedaaannya adalah jika suatu include ke suatu file dilakukan selama lebih dari 1 kali dalam suatu file, maka akan menghasilkan eror karena dianggapnya ada pendklarasian ulang(redeclare), tetapi jika menggunakan include.once() atau require.once() maka kejadian tersebut dihindaari.\\
contohnya :\\
\$aoutoload\['libraries'] = array('form_validation','database','session');\\
\$this load library ('nama.library')\\
$$

5.Jelaskan dengan contoh pemakaian paket denga  perintah from kalkulator import penambahan disertai dengan contoh kode lainnya.\\
paket dengan perintah from kalkuator import penambahan pertama yaitu tentukan nama fungsi, variabel, dan inputannya apa saja, setiap penuliskan harus menggunakaan () dan : dan identasi (jaraknya harus sama)\\
contoh :\\
def penambahan (a+b):\\
r=(a+b)\\
return\\
a=5\\
b=6\\
anu=penambahan(a,b)

6. Jelaskan dengan contoh kodenya, pemakaian paket fungsi apabila file library ada di dalam folder.\\
def tambah(bil1,bil2):\\
	total = bil1+bil2\\
	return total\\

def kurang(bil1,bil2):\\
	total = bil1-bil2\\
	return total\\
	
def kali(bil1,bil2):\\
	total = bil1*bil2\\
	return total\\
	
def bagi(bil1,bil2):\\
	total = bil1/bil2\\
	return total\\
	
def nilai(n1,n2):\\
	hasil = n1+n2\\
	if 80 <= hasil <= 100:\\
		print("nilai anda adalah A")\\
	elif 70 <= hasil <80:\\
		print("nilai anda adalah B")\\
	elif 60 <= hasil <70:\\
		print("nilai anda adalah C")\\
	elif 50 <= hasil <60:\\
		print("nilai anda adalah D, SIlahkan Mengulang kembali")\\
	else:
		print("anda GAGAL")\\
		
7.Jelaskan dengan contoh kodenya, pemakaian paket kelas apabila file library ada di dalam folder\\
mendefinisikan sebuah class dengan menggunakan kata kunci class diikuti oleh nama class tersebut.\\
class ClassName:
	'''class docstring'''\\
	class.body\\
class memiliki docstring atau string dokumentasi yang bersifat opsional artinya bisa atau tidak. Docstring bisa diakses menggunakan format\\
ClassName.doc

\section{ainul filiani}



1. Apa itu fungsi, inputan fungsi dan kembalian fungsi dengan contoh kode program lainnya.\\
fungsi / function adalah satu blok kode yang melakukan tugas tertentu atau satu blok kode yang melakukan tugas tertentu atau satu blok instruksi yang eksekusi ketika dipanggil dari bagian lain dalam seuatu program. tujuan pembuatan fungsi adalah : memudahkan dalam pembuatan program.\\
contoh kodenya :\\
def function\_name(parameters):\\
	"""function\_docstring"""\\
	statement(s)\\
	return [expression]\\
	
2. Apa itu paket dan cara pemanggilan paket atau library dengan contoh kode program lainnya.\\
paket digunakan untuk mengelompokkan kelas-kelas yang mempunyai kemiripan fungsi (related class). kelas-kelas java yang akan digunakan didalam program, terlebih dahulu harus diimpor beserta dengan nama paket dimana kelas tersebut berada.\\
cara memanggilnya :\\
penampung = namaClass()\\
penampung.namaMetode()\\

3. Jelaskan apa itu kelas, apa itu objek, apa itu atribut, apa itu method dan contoh kode program lainnya masing-masing.\\
- class adalah salah satu cara bagaimana kita membuuat sebuah kode yang mempunyai behaviour tertentu dan lebih mudah dalam mengprganisasi berbagai fungsi dan state-nya. dalam sebuah class kamu dapat menyimpan sebuah state tanpa harus membuat banyak state bila tidak menggunakan class.\\
contohnya : \\
Class Product:\\
	vendor.message = "Ini adalah rahasia"\\
	name = ""\\
	price = ""\\
	size = ""\\
	unit = ""\\
	
	def.init\_(self, name):\\
		print "ini adalah consuctor"\\
		self.name = name\\
		self.unit = "ml"\\
		self.size = 250\\
		
	def get.vendor.message(self):\\
		print self.vendor.message\\
		
	def set\_price(Self, price):\\
		self.price = price\\
		
- objek adalah instansi atau perwujudan dari sebuah kelas. bila kelas adalah prototypenya, dan objek adalah barang jadinya.\\
contohnya :\\
obj=Karyawan("K001", "Dini", "Teknisi")\\
obj.infoKaryawan()\\

\#tambah karyawan baru\\
obj2=Karyawan("K002", "Ayu", "Akunting")\\
obj2.infoKaryawan()

\#tampilkan total karyawan\\
print "Total Karyawan " %d " % Karyawan.jml_karyawan\\

- atribut adalah instance spesifik untuk setiap objek, atribut class sama untuk semua contoh -- yang dalam hal ini adalah semua dog.\\
contohnya :\\
class Dog:\\
\#class Attribute\\
	species = 'mammal'\\
\#Initializer / Instance Attributes\\
	def.init(self, name, age):\\
		self.name = name\\
		self.age = age\\
jadi, sementara setiap Dog memiliki nama dan umur yang unik, setiap Dog akan menjadi mamalia.\\

\-Method digunakan untuk melakukan operasi dengan atribut objek, seperti init metodenya, argumen pertama selalu self:\\
contohnya :\\
class dog:\\
\# instance method
	def description(self):\\
		return "() is () years old".format(self.name, self.age)\\
\# instance method\\
	def speak(self, sound):\\
		return "() says ()".format(self.name, sound)\\
\#instantiate the Dog object\\
mikey = Dog("Mikey", 6)\\
\#call our instance methods\\
print(mikey.description())\\
print(mikey.speak("Gruff Gruff"))\\

4. Jelaskan cara pemanggilan library kelas dari instansiasi dan pemakaiannya dengan contoh program lainnya.\\
library adaalh salah satu cara yang paling efisien untu menghemat waktu ketika membangun aplikasi. untuk memanggil file library digunakan perintah include() atau require(), selain include() dan require() ada juga include.once() dan require.once(), perbedaaannya adalah jika suatu include ke suatu file dilakukan selama lebih dari 1 kali dalam suatu file, maka akan menghasilkan eror karena dianggapnya ada pendklarasian ulang(redeclare), tetapi jika menggunakan include.once() atau require.once() maka kejadian tersebut dihindaari.\\
contohnya :\\
\$aoutoload\['libraries'] = array('form_validation','database','session');\\
\$this load library ('nama.library')\\
$$

5.Jelaskan dengan contoh pemakaian paket denga  perintah from kalkulator import penambahan disertai dengan contoh kode lainnya.\\
paket dengan perintah from kalkuator import penambahan pertama yaitu tentukan nama fungsi, variabel, dan inputannya apa saja, setiap penuliskan harus menggunakaan () dan : dan identasi (jaraknya harus sama)\\
contoh :\\
def penambahan (a+b):\\
r=(a+b)\\
return\\
a=5\\
b=6\\
anu=penambahan(a,b)

6. Jelaskan dengan contoh kodenya, pemakaian paket fungsi apabila file library ada di dalam folder.\\
def tambah(bil1,bil2):\\
	total = bil1+bil2\\
	return total\\

def kurang(bil1,bil2):\\
	total = bil1-bil2\\
	return total\\
	
def kali(bil1,bil2):\\
	total = bil1*bil2\\
	return total\\
	
def bagi(bil1,bil2):\\
	total = bil1/bil2\\
	return total\\
	
def nilai(n1,n2):\\
	hasil = n1+n2\\
	if 80 <= hasil <= 100:\\
		print("nilai anda adalah A")\\
	elif 70 <= hasil <80:\\
		print("nilai anda adalah B")\\
	elif 60 <= hasil <70:\\
		print("nilai anda adalah C")\\
	elif 50 <= hasil <60:\\
		print("nilai anda adalah D, SIlahkan Mengulang kembali")\\
	else:
		print("anda GAGAL")\\
		
7.Jelaskan dengan contoh kodenya, pemakaian paket kelas apabila file library ada di dalam folder\\
mendefinisikan sebuah class dengan menggunakan kata kunci class diikuti oleh nama class tersebut.\\
class ClassName:
	'''class docstring'''\\
	class.body\\
class memiliki docstring atau string dokumentasi yang bersifat opsional artinya bisa atau tidak. Docstring bisa diakses menggunakan format\\
ClassName.doc

%%%%%%%%%%%%%%%%%%%%%%%%%%%%%%%%%%%%%%%%%%%%%%%%%%%%%%%%%%%
\section{Chandra Kirana Poetra} 
\subsection{Pemahaman Teori}
\begin{enumerate}

\item Apa itu fungsi, inputan fungsi dan kembalian fungsi dengan contoh kode program lainnya.
Fungsi merupakan suatu fitur yang dapat digunakan untuk membuat suatu program lebih mudah dimengerti dan juga agar program tersusun rapih serta mengurangi redudansi kode yang sama, misalkan kita ingin membuat suatu fungsin pemjumlahan maka, kita membuat fungsi penjumlahan lalu ketika kita ingin menghitung penjumlahan, maka yang perlu kita lakukan tinggal memanggil fungsinya itu sendiri agar fungsi itu lah yang akan melakukan eksekusi perhitungan. fungsi pada pyhton biasanya dimulai dengan mengetikan kata kunci def yang selanjutnya diikuti dengan nama fungsinya itu sendiri contohnya seperti dibawah ini : 
\lstinputlisting[firstline=10, lastline=10]{src/1174079/1174079.py}

Inputan fungsi sendiri merupakan suatu input/masukan data yang kita berikan ke program
\lstinputlisting[firstline=13, lastline=13]{src/1174079/1174079.py}

Pengembalian fungsi atau return merupakan suatu fitur yang digunakan untuk mengembalikan nilai yang telah di proses oleh fungsi, penulisan return sendiri yaitu return diikuti dengan variabel mana yang nantinya ingin nilainya dikembalikan
\lstinputlisting[firstline=11, lastline=11]{src/1174079/1174079.py}



\item Apa itu paket dan cara pemanggilan paket atau library dengan contoh kode program lainnya.
Library atau paket sendiri merupakan suatu kumpulan dari fungsi fungsi yang ditulis oleh berbagai macam developer yang biasanya digunakan untuk menghemat waktu, dikarenakan misalkan untuk fungsi koneksi, biasanya orang sudah menyediakan librarynya, oleh karena itu dikarenakan sudah ada yang membuat, maka akan lebih cepat jika kita menggunakan library yang dibuat oleh orang lain dibandingkan dengan kita harus membuat lagi dari awal.
\lstinputlisting[firstline=16, lastline=18]{src/1174079/1174079.py}




\item Jelaskan Apa itu kelas, apa itu objek, apa itu atribut, apa itu method dan contoh kode program lainnya masing-masing.
Kelas merupakan blueprint atau template dari suatu objek yang nantinya akan di munculkan menjadi instance objek
objek merupakan sebuah instance dari suati kelas, kelas merupakan rancangan yang mendeskripsikan sesuatu hal didunia nyata secara umum misalkan kita punya class mobil, maka kita bisa membuat objek mobil sedan, mobil sport, mobil offroad dan lain lain seperti itulah contoh class dan objek,
attribut adalah sesuatu hal yang melekat pada suatu objek atau kelas
sedangkan method sendiri merupakan suatu fungsi yang ada pada suatu kelas yang cara penggunaanya dengan dipanggil
\lstinputlisting[firstline=20, lastline=41]{src/1174079/1174079.py}


\item Jelaskan cara pemanggilan library kelas dari instansiasi dan pemakaiannya dengan contoh program lainnya.
cara pemanggilannya yaitu dengan import terdahulu filenya, lalu buat variabel untuk menampung data dan setelah itu panggil nama class dan panggil method yang ada di class itu.
\lstinputlisting[firstline=42, lastline=50]{src/1174079/1174079.py}


\item Jelaskan dengan contoh pemakaian paket dengan perintah from kalkulator import Penambahan disertai dengan contoh kode lainnya. contohnya sebagai berikut :
\lstinputlisting[firstline=52, lastline=56]{src/1174079/1174079.py}

\item Jelaskan dengan contoh kodenya, pemakaian paket fungsi apabila file library ada di dalam folder.
pemakaian paket sendiri adalah kumpulan fungsi fungsi contoh dari kodenya sebagai berikut : 
\lstinputlisting[firstline=58, lastline=72]{src/1174079/1174079.py}

\item Jelaskan dengan contoh kodenya, pemakaian paket kelas apabila file library ada di dalam folder.
pemakaian paket sendiri adalah kumpulan fungsi fungsi contoh dari kodenya sebagai berikut : 
\lstinputlisting[firstline=74, lastline=84]{src/1174079/1174079.py}
\end{enumerate}

\subsection{Praktek}
\begin{enumerate}
    \item Jawaban soal no 1
    \lstinputlisting[firstline=86, lastline=121]{src/1174079/1174079.py}
    \item Jawaban soal no 2
    \lstinputlisting[firstline=123, lastline=129]{src/1174079/1174079.py}
    \item Jawaban soal no 3
    \lstinputlisting[firstline=131, lastline=139]{src/1174079/1174079.py}
    \item Jawaban soal no 4
    \lstinputlisting[firstline=141, lastline=146]{src/1174079/1174079.py}
    \item Jawaban soal no 5
    \lstinputlisting[firstline=148, lastline=153]{src/1174079/1174079.py}
    \item Jawaban soal no 6
    \lstinputlisting[firstline=155, lastline=161]{src/1174079/1174079.py}
    \item Jawaban soal no 7
    \lstinputlisting[firstline=163, lastline=169]{src/1174079/1174079.py}
    \item Jawaban soal no 8
    \lstinputlisting[firstline=171, lastline=178]{src/1174079/1174079.py}
    \item Jawaban soal no 9
    \lstinputlisting[firstline=180, lastline=186]{src/1174079/1174079.py}
    \item Jawaban soal no 10
    \lstinputlisting[firstline=188, lastline=204]{src/1174079/1174079.py}
    \item Jawaban soal no 11
    \lstinputlisting[firstline=8, lastline=22]{src/1174079/main.py}
    \item Jawaban soal no 12
    \lstinputlisting[firstline=24, lastline=39]{src/1174079/main.py}
\end{enumerate}

\subsection{Praktek}
\begin{enumerate}

\item peringantan error dan juga penjelasan

\item syntax error
terjadi ketika kode yang dijalankan terdapat kesalahan penulisan, penempatan dan lain lain dengan solusi mencari di line yang terdapat masalah

\item zero division error
eror ini terjadi ketika eksekusi program yang memiliki rumus matematika menghasilkan 0 dari eksekusi bagi yang memiliki hasil 0, jangan pernah membagi dengan 0

\item name error
name error adalah error yang terjadi ketika kesalahan penamaan terdeteksi, solusinya jangan salah ketik

contoh fungsi yang menggunakan try dan except
    \lstinputlisting[firstline=209, lastline=214]{src/1174079/1174079.py}
\end{enumerate}



=======
\section {Sekar }

\documentclass[10pt]{article}

\title{Fungsi}

\begin{document}
Pada contoh dibawah , sebuah fungsi dengan nama perkalian(), memiliki dua buah argumen yaitu a dan b. Isi dari fungsi tersebut adalah melakukan perhitungan perkalian yang diambil dari nilai a dan b, yang di simpan ke dalam variabel c. Nilai dari c lah yang akan dikembalikan oleh fungsi dari hasil pemanggilan fungsi melalui statemen perkalian(5,10)\\
\include 
\begin{equation}
Contoh:\\
def perkalian(a,b):
	c = a*b
return c
	#Program Utama
print( perkalian(5,10))

>>> def nama():
	gelar = 'Mr'
	aksi = (lambda x: gelar + ' ' + x)
	return aksi

>>> act = nama()
>>> act('Namjoon')
'Sir Namjoon'
\end{equation}

\begin{Scope Variabel}
cakupan variabel merupakan suatu keadaan dimana pendeklarasian sebuah variabel di tentukan , Dalam scope variabel dikenal dua istilah yaitu local dan global.
Contoh penggunaan scope variabel() :
x = 12
y = 3
	print "Sebelum memanggil fungsi, x bernilai", x
	print "Sebelum memanggil fungsi, y bernilai", y
swap(x,y)
	print "Setelah memanggil fungsi, x bernilai", x
	print "Setelah memanggil fungsi, y bernilai", y
	
\begin{Fungsi Rekursif}
untuk menyederhanakan penulisan program dan menggantikan bentuk iterasi. Dengan rekursi, program akan lebih mudah dilihat.
# Fungsi Rekursif faktorial
	def faktorial(nilai):
		if nilai <= 1:
	return 1
		else:
	return nilai * faktorial(nilai - 1)
#Program utama
	for i in range(11):
	print "%2d ! = %d" % (i, faktorial(i))
	
\begin{Melewatkan Argumen dengan Kata Kunci}
Jika fungsi perkalian kita panggil dengan memberi pernyataan perkalian(10,8), maka nilai 10 akan disalin ke variabel x dan nilai 8 ke variabel y.\\
def perkalian(a, b):
	"Mengalikan dua bilangan"
	z = x * y
		print "Nilai a =",a
		print "Nilai b =",b
		print "a* b =",c
# program utama mulai di sini
	perkalian(5,3)
		print perkalian(b=4,a=2)
Hasilnya:
Nilai a = 5
Nilai b = 3
a*b = 15
Nilai a = 2

Jadi nilai default hanya boleh diberikan kepada deretan akhir parameter. Setelah pemberian nilai default, semua parameter di belakangnya juga harus diberi nilai default. Satu catatan, nilai awal argumen akan dievaluasi pada saat dideklarasikan. Perhatikan contoh berikut :
usernm="admin"
passwd="aa"
def login(username=usernm, password=passwd):
	print "Your username ",username
	print "Your password ",password
	print 
	usernm="tamu" 
	passwd="cc"
login()
Untuk memanggil fungsi dengan deklarasi seperti ini, kita harus menyebutkan daftar argumen beserta kata-kuncinya. 
Contoh ():
	def cetak1():
print ‘Hello World’
	def cetak2(n):
print n
	cetak1()
hallo world
	cetak2(123)
123
	cetak2('apa kabar?')
apa kabar
	def cetak3(x,y,z):
print x,y,z
	def cetak4(x,y,z=4):
print x,y,z
	cetak3(1,2,3)
1 2 3
	cetak4(1,2)
1 2 4
	cetak4(1,2,3)
1 2 3

\class Ngitung:
  def __init

\end{document}

\begin{Kelas}
Class adalah salah satu cara bagaimana kita membuat sebuah kode yang mempunyai behaviour tertentu dan lebih mudah dalam mengorganisasi berbagai fungsi dan state-nya. Dalam sebuah class kamu dapat menyimpan sebuah state tanpa harus membuat banyak state bila tidak menggunakan class.\\
Contoh :\\
class Product:
    __vendor_message = "Ini adalah rahasia"
    name = ""
    price = ""
    size = ""
    unit = ""
    
    def __init__(self, name):
        print "Ini adalah constructor"
        self.name = name
        self.unit = "ml"
        self.size = 350
        
    def get_vendor_message(self):
        print self.__vendor_message
        
	def set_price(self, price):
        self.price = price
        
p = Product("Banana Milk")
p.set_price(5500)

print "%s dengan ukuran %s %s harganya Rp. %d" % (p.name, p.size, p.unit, p.price)
# print p.__vendor_message

p.get_vendor_message()

p1 = Product("UltraMilk")
p1.set_price(3000)

print "%s dengan ukuran %s %s harganya Rp. %d" % (p.name, p.size, p.unit, p.price)

print p == p
print p1 == p1
print p == p1

\begin{Pemahanan Teori}
1.void(fungsi tanpa nilai balik)
	Fungsi yang void sering disebut juga prosedur. Disebut void karena fungsi tersebut tidak mengembalikan suatu nilai keluaran yang didapat dari hasil proses tersebut.
	Ciri-ciri dari jenis fungsi Void , yaitu :
	1. tidak adanya keyword return
	2. tidak adanya tipe data di dalam deklarasi fungsi
	3. menggunakan keyword void
	4. tidak memiliki nilai kembalian fungsi
	5. Keyword void juga digunakan jika suatu function tidak 				   mengandung suatu parameter apapun.
Contohnya:\\
	void menampilkan_jumlah(int a, int b)}
		in jumlah;
		jumlah = a + b;
		cout << jumlah;
	}
2.Non void (fungsi dengan nilai balik
	Fungsi non-void disebut juga function . disebut non-void karena mengembalikan nilai kembalian yang berasal dari keluaran hasil proses function tersebut.
	Ciri-ciri dari jenis fungsi Non-Void , yaitu :
	1. Ada keyword return
	2. ada tipe data yang mengawali fungsi
	3. tidak ada keyword void
	4. memiliki nilai keyword
	5. Non-void : int jumlah (int a, int b)

3. Prototype Function
	Sebuah program C++ dapat terdiri dari banyak fungsi. Salah satu fungsi tersebut harus bernama main(). Jika fungsi yang lain dituliskan setelah fungsi main(), sebelum fungsi main ditambahkan.
	Contohnya:\\
#include <stdio.h>
\\prototype function
	int hitung(int angka, int bilangan);
	int tulis(char);
	int tampil(int angka[],char huruf);
//fungsi main
	int main(){
		int array[3]={1,2,3};
		char huruf="D";
		//memanggil fungsi
		hitung (2,3);
		tulis("A");
		tampil(array,huruf);
}

4. Fungsi Rekursif
	Fungsi yang memanggil dirinya sendiri. Artinya , fungsi tersebut dipanggil di dalam tubuh fungsi itu sendiri. Parameter yang dilewatkan berubah sebanyak fungsi itu dipanggil.
	
\begin{Apa itu paket dan cara pemanggilan paket atau library dengan contoh kode program lainnya}
	<?php if ( ! defined('BASEPATH'))
		exit('No direct script access allowed');
	class Blog extends ci_controller {
	function __construct()
	{
		parent ::__construct();
	}
	function index()
	{
		echo "Hallo.. saya min yoongi adalah contoh dari boyband BTS 				yang mendunia";
	}
	
	
}

\begin{Jelaskan Apa itu kelas, apa itu objek, apa itu atribut, apa itu method dan contoh kode program lainnya masing-masing.}
gambaran umum tentang sebuah benda. Di dalam pemrograman nantinya, contoh class seperti: koneksi_database dan profile_user. penulisan class diawali dengan keyword class, kemudian diikuti dengan nama dari class. Aturan penulisan nama class sama seperti aturan penulisan variabel
Contoh :\\
<?php
	class laptop {
   		// isi dari class laptop...
}
?>

\end{document}

\begin{Jelaskan cara pemanggikan library kelas dari instansiasi dan pemakaiannya dengan contoh program lainnya.}\\
1. Membuat sebuah objek atau sebuah instance pada sebuah kelas disebut instansiasi atau instantiation.
Contoh:\\
	String str = new String("Hello");
	String str2 = "OOP Yes";
Komputer a = new Komputer();
Komputer b = new Komputer();

2. Atribut suatu class harus didefinisikan sebagai instance
variable.\\
Contoh:\\
	public class Time {
		private int hour;
		private int minute;
		private int second;
	//penulisan kode selanjutnya
}

\begin{Jelaskan dengan contoh pemakaian paket dengan perintah from kalkulator import Penambahan disertai dengan contoh kode lainnya.}\\
Paket dengan perintah from kalkulator import import penambahan pertama , yaitu tentukan nama fungsi , variabel dan inputnya. setiap penulisan harus menggunakan () dan : dan identasi.
Contoh  :\\
	def penambahan (a+b):\\
	r=(a+b)\\
	return\\
	a=5\\
	b=6\\
	anu=penambahan(a,b)

\begin{6.Jelaskan dengan contoh kodenya, pemakaian paket fungsi apabila file library ada di dalam folder.}\\
// Meletakkan kelas ke paket
package bangun.datar;
 
// Mendefinisikan kelas Segi3ABC
public class Segi3ABC {
 
   // Metoda hitungKeliling
   // Untuk mencari keliling segi tiga
   public static double hitungKeliling(double sisiAB, double sisiBC, double sisiCA) {
 
      double keliling;
      keliling = sisiAB + sisiBC + sisiCA;
      return keliling;
   }
 
   // Metoda hitungLuas
   // Untuk mencari luas segi tiga
   public static double hitungLuas(double sisiAB) {
 
      // Deklarasi variabel
      double luas;
 
      // Mencari tinggi segi tiga
      double tinggi = Math.sqrt(Math.pow(sisiAB, 2) - Math.pow((0.5 * sisiAB), 2));
 
      // Mencari luas segi tiga
      luas = sisiAB * tinggi;
      return luas;
   }
}
\end {enumerate}
	\subsection{Keterampilan Pemrograman}
\begin{itemize}

\end{itemize}
	\item 
		\lstinputlisting {firstline = 58, lastline 64} {src/1174075}
	\item 
        \lstinputlisting {firstline = 67, lastline 76} {src/1174075}
	\item 
        \lstinputlisting {firstline = 79, lastline 84} {src/1174075}
	\item 
        \lstinputlisting {firstline = 87, lastline 91} {src/1174075}
	\item 
        \lstinputlisting {firstline = 94, lastline 100} {src/1174075}
	\item 
        \lstinputlisting {firstline = 103, lastline 109} {src/1174075}
	\item 
        \lstinputlisting {firstline = 112, lastline 117} {src/1174075}
	\item 
	    \lstinputlisting {firstline = 120, lastline 125} {src/1174075}
	\item 
        \lstinputlisting {firstline = 128, lastline 141} {src/1174075}

\end {itemize}

\section {Sekar }
\title{Fungsi}

Pada contoh dibawah , sebuah fungsi dengan nama perkalian(), memiliki dua buah argumen yaitu a dan b. Isi dari fungsi tersebut adalah melakukan perhitungan perkalian yang diambil dari nilai a dan b, yang di simpan ke dalam variabel c. Nilai dari c lah yang akan dikembalikan oleh fungsi dari hasil pemanggilan fungsi melalui statemen perkalian(5,10)\\
\include 
\begin{equation}
Contoh:\\
def perkalian(a,b):
	c = a*b
return c
	#Program Utama
print( perkalian(5,10))

>>> def nama():
	gelar = 'Mr'
	aksi = (lambda x: gelar + ' ' + x)
	return aksi

>>> act = nama()
>>> act('Namjoon')
'Sir Namjoon'
\end{equation}

\begin{Scope Variabel}
cakupan variabel merupakan suatu keadaan dimana pendeklarasian sebuah variabel di tentukan , Dalam scope variabel dikenal dua istilah yaitu local dan global.
Contoh penggunaan scope variabel() :
x = 12
y = 3
	print "Sebelum memanggil fungsi, x bernilai", x
	print "Sebelum memanggil fungsi, y bernilai", y
swap(x,y)
	print "Setelah memanggil fungsi, x bernilai", x
	print "Setelah memanggil fungsi, y bernilai", y
	
\begin{Fungsi Rekursif}
untuk menyederhanakan penulisan program dan menggantikan bentuk iterasi. Dengan rekursi, program akan lebih mudah dilihat.
# Fungsi Rekursif faktorial
	def faktorial(nilai):
		if nilai <= 1:
	return 1
		else:
	return nilai * faktorial(nilai - 1)
#Program utama
	for i in range(11):
	print "%2d ! = %d" % (i, faktorial(i))
	
\begin{Melewatkan Argumen dengan Kata Kunci}
Jika fungsi perkalian kita panggil dengan memberi pernyataan perkalian(10,8), maka nilai 10 akan disalin ke variabel x dan nilai 8 ke variabel y.\\
def perkalian(a, b):
	"Mengalikan dua bilangan"
	z = x * y
		print "Nilai a =",a
		print "Nilai b =",b
		print "a* b =",c
# program utama mulai di sini
	perkalian(5,3)
		print perkalian(b=4,a=2)
Hasilnya:
Nilai a = 5
Nilai b = 3
a*b = 15
Nilai a = 2

Jadi nilai default hanya boleh diberikan kepada deretan akhir parameter. Setelah pemberian nilai default, semua parameter di belakangnya juga harus diberi nilai default. Satu catatan, nilai awal argumen akan dievaluasi pada saat dideklarasikan. Perhatikan contoh berikut :
usernm="admin"
passwd="aa"
def login(username=usernm, password=passwd):
	print "Your username ",username
	print "Your password ",password
	print 
	usernm="tamu" 
	passwd="cc"
login()
Untuk memanggil fungsi dengan deklarasi seperti ini, kita harus menyebutkan daftar argumen beserta kata-kuncinya. 
Contoh ():
	def cetak1():
print ‘Hello World’
	def cetak2(n):
print n
	cetak1()
hallo world
	cetak2(123)
123
	cetak2('apa kabar?')
apa kabar
	def cetak3(x,y,z):
print x,y,z
	def cetak4(x,y,z=4):
print x,y,z
	cetak3(1,2,3)
1 2 3
	cetak4(1,2)
1 2 4
	cetak4(1,2,3)
1 2 3

\class Ngitung:
  def __init

\end{document}

\begin{Kelas}
Class adalah salah satu cara bagaimana kita membuat sebuah kode yang mempunyai behaviour tertentu dan lebih mudah dalam mengorganisasi berbagai fungsi dan state-nya. Dalam sebuah class kamu dapat menyimpan sebuah state tanpa harus membuat banyak state bila tidak menggunakan class.\\
Contoh :\\
class Product:
    __vendor_message = "Ini adalah rahasia"
    name = ""
    price = ""
    size = ""
    unit = ""
    
    def __init__(self, name):
        print "Ini adalah constructor"
        self.name = name
        self.unit = "ml"
        self.size = 350
        
    def get_vendor_message(self):
        print self.__vendor_message
        
	def set_price(self, price):
        self.price = price
        
p = Product("Banana Milk")
p.set_price(5500)

print "%s dengan ukuran %s %s harganya Rp. %d" % (p.name, p.size, p.unit, p.price)
# print p.__vendor_message

p.get_vendor_message()

p1 = Product("UltraMilk")
p1.set_price(3000)

print "%s dengan ukuran %s %s harganya Rp. %d" % (p.name, p.size, p.unit, p.price)

print p == p
print p1 == p1
print p == p1

\begin{Pemahanan Teori}
1.void(fungsi tanpa nilai balik)
	Fungsi yang void sering disebut juga prosedur. Disebut void karena fungsi tersebut tidak mengembalikan suatu nilai keluaran yang didapat dari hasil proses tersebut.
	Ciri-ciri dari jenis fungsi Void , yaitu :
	1. tidak adanya keyword return
	2. tidak adanya tipe data di dalam deklarasi fungsi
	3. menggunakan keyword void
	4. tidak memiliki nilai kembalian fungsi
	5. Keyword void juga digunakan jika suatu function tidak 				   mengandung suatu parameter apapun.
Contohnya:\\
	void menampilkan_jumlah(int a, int b)}
		in jumlah;
		jumlah = a + b;
		cout << jumlah;
	}
2.Non void (fungsi dengan nilai balik
	Fungsi non-void disebut juga function . disebut non-void karena mengembalikan nilai kembalian yang berasal dari keluaran hasil proses function tersebut.
	Ciri-ciri dari jenis fungsi Non-Void , yaitu :
	1. Ada keyword return
	2. ada tipe data yang mengawali fungsi
	3. tidak ada keyword void
	4. memiliki nilai keyword
	5. Non-void : int jumlah (int a, int b)

3. Prototype Function
	Sebuah program C++ dapat terdiri dari banyak fungsi. Salah satu fungsi tersebut harus bernama main(). Jika fungsi yang lain dituliskan setelah fungsi main(), sebelum fungsi main ditambahkan.
	Contohnya:\\
#include <stdio.h>
\\prototype function
	int hitung(int angka, int bilangan);
	int tulis(char);
	int tampil(int angka[],char huruf);
//fungsi main
	int main(){
		int array[3]={1,2,3};
		char huruf="D";
		//memanggil fungsi
		hitung (2,3);
		tulis("A");
		tampil(array,huruf);
}

4. Fungsi Rekursif
	Fungsi yang memanggil dirinya sendiri. Artinya , fungsi tersebut dipanggil di dalam tubuh fungsi itu sendiri. Parameter yang dilewatkan berubah sebanyak fungsi itu dipanggil.
	
\begin{Apa itu paket dan cara pemanggilan paket atau library dengan contoh kode program lainnya}
	<?php if ( ! defined('BASEPATH'))
		exit('No direct script access allowed');
	class Blog extends ci_controller {
	function __construct()
	{
		parent ::__construct();
	}
	function index()
	{
		echo "Hallo.. saya min yoongi adalah contoh dari boyband BTS 				yang mendunia";
	}
	
	
}

\begin{Jelaskan Apa itu kelas, apa itu objek, apa itu atribut, apa itu method dan contoh kode program lainnya masing-masing.}
gambaran umum tentang sebuah benda. Di dalam pemrograman nantinya, contoh class seperti: koneksi_database dan profile_user. penulisan class diawali dengan keyword class, kemudian diikuti dengan nama dari class. Aturan penulisan nama class sama seperti aturan penulisan variabel
Contoh :\\
<?php
	class laptop {
   		// isi dari class laptop...
}
?>

\end{document}

\begin{Jelaskan cara pemanggikan library kelas dari instansiasi dan pemakaiannya dengan contoh program lainnya.}\\
1. Membuat sebuah objek atau sebuah instance pada sebuah kelas disebut instansiasi atau instantiation.
Contoh:\\
	String str = new String("Hello");
	String str2 = "OOP Yes";
Komputer a = new Komputer();
Komputer b = new Komputer();

2. Atribut suatu class harus didefinisikan sebagai instance
variable.\\
Contoh:\\
	public class Time {
		private int hour;
		private int minute;
		private int second;
	//penulisan kode selanjutnya
}

\begin{Jelaskan dengan contoh pemakaian paket dengan perintah from kalkulator import Penambahan disertai dengan contoh kode lainnya.}\\
Paket dengan perintah from kalkulator import import penambahan pertama , yaitu tentukan nama fungsi , variabel dan inputnya. setiap penulisan harus menggunakan () dan : dan identasi.
Contoh  :\\
	def penambahan (a+b):\\
	r=(a+b)\\
	return\\
	a=5\\
	b=6\\
	anu=penambahan(a,b)

\begin{6.Jelaskan dengan contoh kodenya, pemakaian paket fungsi apabila file library ada di dalam folder.}\\
// Meletakkan kelas ke paket
package bangun.datar;
 
// Mendefinisikan kelas Segi3ABC
public class Segi3ABC {
 
   // Metoda hitungKeliling
   // Untuk mencari keliling segi tiga
   public static double hitungKeliling(double sisiAB, double sisiBC, double sisiCA) {
 
      double keliling;
      keliling = sisiAB + sisiBC + sisiCA;
      return keliling;
   }
 
   // Metoda hitungLuas
   // Untuk mencari luas segi tiga
   public static double hitungLuas(double sisiAB) {
 
      // Deklarasi variabel
      double luas;
 
      // Mencari tinggi segi tiga
      double tinggi = Math.sqrt(Math.pow(sisiAB, 2) - Math.pow((0.5 * sisiAB), 2));
 
      // Mencari luas segi tiga
      luas = sisiAB * tinggi;
      return luas;
   }
}
\end {enumerate}
	\subsection{Keterampilan Pemrograman}
\begin{itemize}

\end{itemize}
	\item 
		\lstinputlisting {firstline = 58, lastline 64} {src/1174075}
	\item 
        \lstinputlisting {firstline = 67, lastline 76} {src/1174075}
	\item 
        \lstinputlisting {firstline = 79, lastline 84} {src/1174075}
	\item 
        \lstinputlisting {firstline = 87, lastline 91} {src/1174075}
	\item 
        \lstinputlisting {firstline = 94, lastline 100} {src/1174075}
	\item 
        \lstinputlisting {firstline = 103, lastline 109} {src/1174075}
	\item 
        \lstinputlisting {firstline = 112, lastline 117} {src/1174075}
	\item 
	    \lstinputlisting {firstline = 120, lastline 125} {src/1174075}
	\item 
        \lstinputlisting {firstline = 128, lastline 141} {src/1174075}

\end {itemize}


%%%%%%%%%%%%%%%%%%%%%%%%%%%%%%%%%%%%%%%%%%%%%%%%%%%%%%%%%%%%%%%%%%%%%%%

\section{Bakti Qilan Mufid}
\subsection{Teori}
\begin{enumerate}
\item apa itu fungsi, inputan fungsi, dan kembalian fungsi dan contohnya

Fungsi adalah bagian dari program yang dapat digunakan ulang. yaitu dengan cara memberinya nama dan kemudian nama itu dapat dipanggil dimanapun dalam program. fungsi dalam pyhton didefinisikan dengan kata kunci \textbf{def}. Setelah \textbf{def} ada nama pengenal fungsi diikuti dengan parameter yang diapit oleh tanda kurung dan diakhiri dengan tanda titik dua (\textbf{:}). Baris berikutnya berupa blok fungsi atau perintah yang akan dijalankan jika fungsi tersebut dipanggil. untuk pengisian variabel bisa diisikan lebih dari satu variabel, dengan menggunakan tanda pemisah yaitu koma (,). contoh dari fungsi sederhana itu bisa dilihat dibawah dimana hasil akhir dari variabel c adalah 25
\lstinputlisting[firstline=92, lastline=99]{src/1174083.py}

\item Apa itu paket dan cara pemanggilan paket atau library dengan contoh kode
program lainnya.

Paket atau package ialah sebuah file, contoh nya file bernama \textit{117408.py} yang didalmnya berisi semua fugnsi, seperti penambahan, pengurangan, pembagian dan perkalian. sehingga file tersebut kita namakan sebagai paket/ package/ library. contoh nya seperti pada kode program dibawah.
\lstinputlisting[firstline=101, lastline=112]{src/1174083.py}

	
\item Jelaskan Apa itu kelas, apa itu objek, apa itu atribut, apa itu method dan contoh kode program lainnya masing-masing

kelas, ialah sebuah prototipe atau kerangka yang digunakan oleh pengguna untuk objek 	yang mendefinisikan seperangkat atribut yang menjadi ciri objek kelas.
\begin{itemize}
\item atribut, ialah data atau apapun yang menempel pada objek.
	\item objek, ialah contoh unik dari struktur data yang didefinisikan oleh kelasnya.
	\item method, ialah perilaku, ataupun yang dikerjakan oleh objek.
	\item contoh kode programnya
	\lstinputlisting[firstline=115, lastline=131]{src/1174083.py}	
\end{itemize}		
	
\item Jelaskan cara pemanggikan library kelas dari instansiasi dan pemakaiannya dengan contoh program lainnya

Untuk memanggil library atau paket, kita cukup dengan menggunakan keyword import, seperti pada kode program dibawah. Di sini kita mencoba memanggil libarary pada kelas 1174083.py dengan menggunakan nama test. 
\lstinputlisting[firstline=134, lastline=144]{src/1174083.py}
	
\item Jelaskan dengan contoh pemakaian paket dengan perintah from kalkulator import Penambahan disertai dengan contoh kode lainnya

Penggunaan paket from \textit{namafile} import, itu berfungsi untuk memanggil file dan fungsinya, seperti pada kode program dibawah
\lstinputlisting[firstline=149, lastline=150]{src/1174083.py}
	
	
\item Jelaskan dengan contoh kodenya, pemakaian paket fungsi apabila file library ada di dalam folder
\item Jelaskan dengan contoh kodenya, pemakaian paket kelas apabila file library ada di dalam folder

Contoh pemakaian paket seperti pada kode program dibawah
\lstinputlisting[firstline=153, lastline=159]{src/1174083.py}

\end{enumerate}

\subsection{Ketrampilan Pemrograman}
\begin{enumerate}
\item Jawaban
\lstinputlisting[firstline=167, lastline=202]{src/1174083.py}

\item Jawaban
\lstinputlisting[firstline=205, lastline=212]{src/1174083.py}

\item Jawaban
\lstinputlisting[firstline=215, lastline=223]{src/1174083.py}

\item Jawaban
\lstinputlisting[firstline=226, lastline=229]{src/1174083.py}

\item Jawaban
\lstinputlisting[firstline=232, lastline=237]{src/1174083.py}

\item Jawaban
\lstinputlisting[firstline=240, lastline=248]{src/1174083.py}

\item Jawaban 
\lstinputlisting[firstline=251, lastline=259]{src/1174083.py}

\item Jawaban 
\lstinputlisting[firstline=262, lastline=269]{src/1174083.py}

\item Jawaban 
\lstinputlisting[firstline=272, lastline=278]{src/1174083.py}

\item Jawaban 
\lstinputlisting[firstline=281, lastline=297]{src/1174083.py}

\item Jawaban 
\lstinputlisting[firstline=8, lastline=24]{src/1174083/main.py}

\item Jawaban 
\lstinputlisting[firstline=26, lastline=43]{src/1174083/main.py}

\end{enumerate}

\subsection{Ketrampilan Penanganan Error}
\begin{enumerate}
\item Peringatan error yang ditemukan dan penjelasannya serta buat sebuah fungsi try except untuk menanggulangi error.
	
Peringatan error di praktek ketiga ini, yaitu:
\begin{itemize}
	\item Syntax Errors
		Syntax Errors adalah suatu keadaan saat kode python mengalami kesalahan penulisan. Solusinya adalah memperbaiki penulisan kode yang salah.
		
	\item Zero Division Error
		ZeroDivisonError adalah exceptions yang terjadi saat eksekusi program menghasilkan perhitungan matematika pembagian dengan angka nol (0). Solusinya adalah tidak membagi suatu yang hasilnya nol.
		
	\item Name Error
		NameError adalah exception yang terjadi saat kode melakukan eksekusi terhadap local name atau global name yang tidak terdefinisi. Solusinya adalah memastikan variabel atau function yang dipanggil ada atau tidak salah ketik.
		
	\item Type Error
		TypeError adalah exception yang terjadi saat dilakukan eksekusi terhadap suatu operasi atau fungsi dengan type object yang tidak sesuai. Solusinya adalah mengkoversi varibelnya sesuai dengan tipe data yang akan digunakan.
\end{itemize}
	
Contoh fungsi yang menggunakan try except
\lstinputlisting[firstline=306, lastline=312]{src/1174083.py}
\end{enumerate}
%%%%%%%%%%%%%%%%%%%%%%%%%%%%%%%%%%%%%%%%%%%%%%%%%%%%%%%%%%%
\section{Advent Nopele Olansi Damiahan Sihite 1174089}
\subsubsection{Pemahanan Teori}
\begin{enumerate}
    \item Apa itu fungsi, inputan fungsi dan kembalian fungsi dengan contoh kode program
    lainnya.
    Fungsi adalah bagian dari program yang dapat digunakan ulang.
    Berikut merupakan contoh fungsi dan cara pemanggilannya
    \lstinputlisting[firstline=124, lastline=127]{src/1174089.py}

    Fungsi dapat membaca parameter, parameter adalah nilai yang disediakan kepada fungsi, dimana nilai ini akan menentukan output yang akan dihasilkan fungsi.
    \lstinputlisting[firstline=129, lastline=132]{src/1174089.py}

    Statemen return digunakan untuk keluar dari fungsi. Kita juga dapat menspesifikasikan nilai kembalian.
    \lstinputlisting[firstline=134, lastline=141]{src/1174089.py}

    \item Apa itu paket dan cara pemanggilan paket atau library dengan contoh kode
    program lainnya.
    Untuk memudahkan dalam pemanggilan fungsi yang di butuhkan, agar dapat dipanggil berulang.
    Cara pemanggilannya
    \lstinputlisting[firstline=143, lastline=144]{src/1174089.py}

    \item Jelaskan Apa itu kelas, apa itu objek, apa itu atribut, apa itu method dan
    contoh kode program lainnya masing-masing.
    kelas merupakan sebuah blueprint yang mepresentasikan objek.
    objek adalah hasil cetakan dadri sebuah kelas.
    method adalah suatu upaya yang digunakan oleh object.
    \lstinputlisting[firstline=146, lastline=168]{src/1174089.py}

    \item Jelaskan cara pemanggikan library kelas dari instansiasi dan pemakaiannya den-
    gan contoh program lainnya.
    Cara Pemanggilanya 
    \begin{itemize}
        \item pertama import terlebih dahulu filenya.
        \item kemudian buat variabel untuk menampung datanya
        \item setelah itu panggil nama classnya dan panggil methodnya
        \item Gunakan perintah print untuk menampilkan hasilnya

    \end{itemize}
    \lstinputlisting[firstline=170, lastline=175]{src/1174089.py}

    \item Jelaskan dengan contoh pemakaian paket dengan perintah from kalkulator im-
    port Penambahan disertai dengan contoh kode lainnya.
    Penggunaan paket from namafile import, itu berfungsi untuk memanggil file dan fungsinya
    \lstinputlisting[firstline=143, lastline=144]{src/1174089.py}

    \item Jelaskan dengan contoh kodenya, pemakaian paket fungsi apabila le library
    ada di dalam folder.
    Pemakaian paket adalah perkumpulan fungsi-fungsi. contoh kodenya adalah sebagai berikut :

    \item Jelaskan dengan contoh kodenya, pemakaian paket kelas apabila le library ada
    di dalam folder.
    \lstinputlisting[firstline=184, lastline=184]{src/1174089.py}

\end{enumerate}
\subsubsection{Ketrampilan Pemrograman}
\begin{enumerate}
    \item Buatlah fungsi dengan inputan variabel NPM, dan melakukan print luaran huruf
    yang dirangkai dari tanda bintang, pagar atau plus dari NPM kita. Tanda
    bintang untuk NPM mod 3=0, tanda pagar untuk NPM mod 3 =1, tanda plus
    untuk NPM mod3=2.
    \lstinputlisting[firstline=184, lastline=234]{src/1174089.py}

    \item Buatlah fungsi dengan inputan variabel berupa NPM. kemudian dengan meng-
    gunakan perulangan mengeluarkan print output sebanyak dua dijit belakang
    NPM.
    \lstinputlisting[firstline=237, lastline=243]{src/1174089.py}

    \item Buatlah fungsi dengan dengan input variabel string bernama NPM dan beri
    luaran output dengan perulangan berupa tiga karakter belakang dari NPM se-
    banyak penjumlahan tiga dijit tersebut.
    \lstinputlisting[firstline=245, lastline=255]{src/1174089.py}

    \item Buatlah fungsi hello word dengan input variabel string bernama NPM dan
    beri luaran output berupa digit ketiga dari belakang dari variabel NPM meng-
    gunakan akses langsung manipulasi string pada baris ketiga dari variabel NPM.
    \lstinputlisting[firstline=257, lastline=263]{src/1174089.py}

    \item buat fungsi program dengan input variabel NPM dan melakukan print nomor npm satu persatu kebawah.
    \lstinputlisting[firstline=265, lastline=269]{src/1174089.py}

    \item Buatlah fungsi dengan inputan variabel NPM, didalamnya melakukan penjum-
    lahan dari seluruh dijit NPM tersebut, wajib menggunakan perulangan dan
    atau kondisi.
    \lstinputlisting[firstline=272, lastline=279]{src/1174089.py}

    \item Buatlah fungsi dengan inputan variabel NPM, didalamnya melakukan melakukan
    perkalian dari seluruh dijit NPM tersebut, wajib menggunakan perulangan dan
    atau kondisi.
    \lstinputlisting[firstline=281, lastline=288]{src/1174089.py}

    \item Buatlah fungsi dengan inputan variabel NPM, Lakukan print NPM anda tapi
    hanya dijit genap saja. wajib menggunakan perulangan dan atau kondisi.
    \lstinputlisting[firstline=290, lastline=296]{src/1174089.py}

    \item Buatlah fungsi dengan inputan variabel NPM, Lakukan print NPM anda tapi
    hanya dijit ganjil saja. wajib menggunakan perulangan dan atau kondisi.
    \lstinputlisting[firstline=298, lastline=304]{src/1174089.py}

    \item Buatlah fungsi dengan inputan variabel NPM, Lakukan print NPM anda tapi
    hanya dijit yang termasuk bilangan prima saja. wajib menggunakan perulangan
    dan atau kondisi.
    \lstinputlisting[firstline=306, lastline=320]{src/1174089.py}

    \item Buatlah satu library yang berisi fungsi-fungsi dari nomor diatas dengan nama
    le 3lib.py dan berikan contoh cara pemanggilannya pada le main.py.
    \lstinputlisting[firstline=7, lastline=7]{src/main_advent.py}

    \item Buatlah satu library class dengan nama le kelas3lib.py yang merupakan mod-
    ikasi dari fungsi-fungsi nomor diatas dan berikan contoh cara pemanggilannya
    pada le main.py.
    \lstinputlisting[firstline=8, lastline=9]{src/main_advent.py}
    
\end{enumerate}
\subsubsection{Ketrampilan Penanganan Error}
Error yang di dapat dari mengerjakan tugas ini adalah type error, cara menaggulaginya dengan cara mengecheck kembali codingannya
kemudian run kembali aplikasinya
berikut contoh Penggunaan fungsi try dan exception
\lstinputlisting[firstline=177, lastline=182]{src/1174089.py}
%%%%%%%%%%%%%%%%%%%%%%%%%%%%%%%%%%%%%%%%%%%%%%%%%%%%%%%%%%%%%%%%%%%%%%%%%%%%%%%%%%%%%%%%%%%%%%%%%%%%%%%%%%%%%%%%%%%%%%%%%%%%%%%%

\section {Arrizal Furqona Gifary}
\subsubsection{Pemahaman Teori}
\begin{enumerate}
    \item Apa itu fungsi, inputan fungsi dan kembalian fungsi dengan contoh kode program
    lainnya, Fungsi adalah bagian dari program yang dapat digunakan ulang.
    Berikut merupakan contoh fungsi dan cara pemanggilannya
\lstinputlisting[firstline=7, lastline=10]{src/1174070.py}
 Fungsi dapat membaca parameter, parameter adalah nilai yang disediakan kepada fungsi, dimana nilai ini akan menentukan output yang akan dihasilkan fungsi.
\lstinputlisting[firstline=12, lastline=15]{src/1174070.py}

Statemen return digunakan untuk keluar dari fungsi. Kita juga dapat menspesifikasikan nilai kembalian.
\lstinputlisting[firstline=17, lastline=24]{src/1174070.py}

\item Apa itu paket dan cara pemanggilan paket atau library dengan contoh kode
program lainnya.
Untuk memudahkan dalam pemanggilan fungsi yang di butuhkan, agar dapat dipanggil berulang.
Cara pemanggilannya
\lstinputlisting[firstline=26, lastline=27]{src/1174070.py}

\item Jelaskan Apa itu kelas, apa itu objek, apa itu atribut, apa itu method dan
contoh kode program lainnya masing-masing.
kelas merupakan sebuah blueprint yang mepresentasikan objek.
objek adalah hasil cetakan dadri sebuah kelas.
method adalah suatu upaya yang digunakan oleh object.
\lstinputlisting[firstline=29, lastline=51]{src/1174070.py}

\item Jelaskan cara pemanggikan library kelas dari instansiasi dan pemakaiannya den-
gan contoh program lainnya.
Cara Pemanggilanya 
\begin{itemize}
\item pertama import terlebih dahulu filenya.
\item kemudian buat variabel untuk menampung datanya
\item setelah itu panggil nama classnya dan panggil methodnya
\item Gunakan perintah print untuk menampilkan hasilnya

\end{itemize}
\lstinputlisting[firstline=53, lastline=58]{src/1174070.py}

\item Jelaskan dengan contoh pemakaian paket dengan perintah from kalkulator im-
port Penambahan disertai dengan contoh kode lainnya.
Penggunaan paket from namafile import, itu berfungsi untuk memanggil file dan fungsinya
\lstinputlisting[firstline=26, lastline=27]{src/1174070.py}

\item Jelaskan dengan contoh kodenya, pemakaian paket fungsi apabila  
le library
ada di dalam folder.
Pemakaian paket adalah perkumpulan fungsi-fungsi. contoh kodenya adalah sebagai berikut :

\item Jelaskan dengan contoh kodenya, pemakaian paket kelas apabila  
le library ada
di dalam folder.
\lstinputlisting[firstline=60, lastline=96]{src/1174070.py}

\end{enumerate}
\subsubsection{Ketrampilan Pemrograman}
\begin{enumerate}
\item Buatlah fungsi dengan inputan variabel NPM, dan melakukan print luaran huruf
yang dirangkai dari tanda bintang, pagar atau plus dari NPM kita. Tanda
bintang untuk NPM mod 3=0, tanda pagar untuk NPM mod 3 =1, tanda plus
untuk NPM mod3=2.
\lstinputlisting[firstline=60, lastline=96]{src/1174070.py}

\item Buatlah fungsi dengan inputan variabel berupa NPM. kemudian dengan meng-
gunakan perulangan mengeluarkan print output sebanyak dua dijit belakang
NPM.
\lstinputlisting[firstline=98, lastline=104]{src/1174070.py}

\item Buatlah fungsi dengan dengan input variabel string bernama NPM dan beri
luaran output dengan perulangan berupa tiga karakter belakang dari NPM se-
banyak penjumlahan tiga dijit tersebut.
\lstinputlisting[firstline=106, lastline=114]{src/1174070.py}

\item Buatlah fungsi hello word dengan input variabel string bernama NPM dan
beri luaran output berupa digit ketiga dari belakang dari variabel NPM meng-
gunakan akses langsung manipulasi string pada baris ketiga dari variabel NPM.
\lstinputlisting[firstline=116, lastline=121]{src/1174070.py}

\item buat fungsi program dengan input variabel NPM dan melakukan print nomor npm satu persatu kebawah.
\lstinputlisting[firstline123=, lastline=128]{src/1174070.py}

\item Buatlah fungsi dengan inputan variabel NPM, didalamnya melakukan penjum-
lahan dari seluruh dijit NPM tersebut, wajib menggunakan perulangan dan
atau kondisi.
\lstinputlisting[firstline=130, lastline=136]{src/1174070.py}

\item Buatlah fungsi dengan inputan variabel NPM, didalamnya melakukan melakukan
perkalian dari seluruh dijit NPM tersebut, wajib menggunakan perulangan dan
atau kondisi.
\lstinputlisting[firstline=138, lastline=144]{src/1174070.py}

\item Buatlah fungsi dengan inputan variabel NPM, Lakukan print NPM anda tapi
hanya dijit genap saja. wajib menggunakan perulangan dan atau kondisi.
\lstinputlisting[firstline=146, lastline=153]{src/1174070.py}

\item Buatlah fungsi dengan inputan variabel NPM, Lakukan print NPM anda tapi
hanya dijit ganjil saja. wajib menggunakan perulangan dan atau kondisi.
\lstinputlisting[firstline=155, lastline=161]{src/1174070.py}

\item Buatlah fungsi dengan inputan variabel NPM, Lakukan print NPM anda tapi
hanya dijit yang termasuk bilangan prima saja. wajib menggunakan perulangan
dan atau kondisi.
\lstinputlisting[firstline=163, lastline=179]{src/1174070.py}

\item Buatlah satu library yang berisi fungsi-fungsi dari nomor diatas dengan nama
 
le 3lib.py dan berikan contoh cara pemanggilannya pada  
le main.py.
\lstinputlisting[firstline=7, lastline=7]{src/main_1174070.py}

\item Buatlah satu library class dengan nama  
le kelas3lib.py yang merupakan mod-
i 
kasi dari fungsi-fungsi nomor diatas dan berikan contoh cara pemanggilannya
pada  
le main.py.
\lstinputlisting[firstline=7, lastline=9]{src/main_1174070.py}
\end{enumerate}
\subsubsection{Ketrampilan Penanganan Error}
Error yang di dapat dari mengerjakan tugas ini adalah type error, cara menaggulaginya dengan cara mengecheck kembali codingannya
kemudian run kembali aplikasinya
berikut contoh Penggunaan fungsi try dan exception
\lstinputlisting[firstline=181, lastline=186]{src/1174070.py}
%%%%%%%%%%%%%%%%%%%%%%%%%%%%%%%%%%%%%%%%%%%%%%%%%%%%%%%%%%%%%%%%%%%%%%%%%%%%%%%%%%%%%%%%%%%%%%

\section{Muhammad Reza Syachrani / 1174084}
\subsection{Pemahaman Teori}
\begin{enumerate}
    \item Fungsi adalah satu blok program yang terdiri dari nama fungsi, input variabel dan variabel kembalian. Fungsi tersebuat dapat digunakan ulang. Hal ini bisa dicapai dengan memberi nama pada blok statemen, kemudian nama ini dapat dipanggil di manapun dalam program.
    \par Nama fungsi dalam Python didefinisikan menggunakan kata kunci \textit{def}. Setelah def ada nama pengenal fungsi diikut dengan parameter yang merupakan input variable yang dapat bernilai lebih dari satu mengguanakan pemisah tanda koma  yang diapit oleh tanda kurung dan diakhir dingan tanda titik dua :. Variabel kembalian pasti satu.
    \par Contoh Kode Pemrograman : \lstinputlisting[caption=Fungsi sederhana, firstline=130, lastline=133, ]{src/1174084/1174084.py}
    
    \item Paket atau library adalah sekumpulan kelas dan fungsi yang dibuat untuk membantu pengembang aplikasi untuk dapat membangun aplikasi dengan lebih cepat dan lebih efisien. Cara pemanggilan yaitu mengunakan perintah import pada file tempat digunakan library tersebut.
    \par Contoh Kode Pemrograman :
    \begin{lstlisting}[caption=Library atau paket luas,label={lst:luaslib}]
def persegi(p,l):
    luas = p * l #p = panjang, l = lebar
    return luas

def lingkaran(r): 
    luas = 3.14 * (r**2) #r = jari-jari
    return luas

def segitiga(a,t):
    luas = (a * t)/2 #a = alas, t = tinggi
    return luas

\end{lstlisting}
\begin{lstlisting}[caption=Cara penggunan Library luas,label={lst:luaslib}]
import luas

p = 10
l = 5
r = 15
a = 12
t = 5

jumlah1=luas.persegi(p,l)
jumlah2=luas.lingkaran(r)
jumlah3=luas.segitiga(a,t)
\end{lstlisting}
    
\item Kelas merupakan suatu blueprint atau cetakan untuk menciptakan suatu instant dari objek. ciri dari kelas itu diawalai dengan kata kunci \textit{class} dan diikuti nama kelas tersebut.
\par Objek adalah Semua hal yang ada dalam dunia nyata baik konkrit maupun abstrak.
\par Atribut merupakan nilai data yang terdapat pada suatu objek yang berasal dari kelas.
\par Method merupakan apa saja yang dapat dilakukan / dialami oleh suatu objek
Contoh Kode Program :
\begin{lstlisting}[caption=Kelas Library Menghitung,label={lst:menghitunglib}]
class Menghitung:
def __init__(self,p,l,r,a,t):
    self.p = p
    self.l = l
    self.r = r
    self.a = a
    self.t = t
    
def persegi(self):
    luas = self.p * self.l #p = panjang, l = lebar
    return luas

def lingkaran(self): 
     luas = 3.14 * (self.r**2) #r = jari-jari
     return luas

def segitiga(self):
     luas = (self.a * self.t)/2 #a = alas, t = tinggi
     return luas
\end{lstlisting}
    
    \item Untuk pemanggilan kelas library menggunakn kata kunci \textit{import} diikuti nama kelas library tersebut. sedangkan cara pemakain untuk membuat objek dari sebuah kelas, kita bisa memanggil nama kelas dengan argumen sesuai dengan fungsi init pada saat kita mendefinisikannya. contohnya menggunakan kelas library menghitung :
    
\begin{lstlisting}[caption=Cara penggunan  kelas Library Menghitung,label={lst:menghitunglib}]
import menghitung

p = 10
l = 5
r = 15
a = 12
t = 5
hitung = menghitung.Menghitung(p,l,r,s,a,t)

jumlah1=hitung.persegi()
jumlah2=hitung.lingkaran()
jumlah3=hitung.segitiga()
\end{lstlisting}

\item Penggunaan paket from nama file import nama kelas library berfungsi untuk memanggil file dan fungsi yang terdapat pada kelas library. contoh pengunaan pada kode program lain :
\begin{lstlisting}[caption= Contoh kode lain pemakaian paket fungsi from import, label={fungsiimp}]
from menghitung import Menghitung

p = 10
l = 5
r = 15
a = 12
t = 5

hitung = Menghitung(p,l,r,a,t)

jumlah1=hitung.persegi()
jumlah2=hitung.lingkaran()
jumlah3=hitung.segitiga()
\end{lstlisting}

\item Contoh kodenya pemakaian paket fungsi apabila file library
ada di dalam folder.
\begin{lstlisting}[caption= Contoh kode pemakaian paket fungsi dimana file library ada di dalam folder, label={funsilib}]
from folder import kalkulator

a=100
b=50

hasil1=kalkulator.Penambahan(a,b)
hasil2=kalkulator.Pengurangan(a,b)
hasil3=kalkulator.Perkalian(a,b)
hasil4=kalkulator.Pembagian(a,b)

print(hasil1)
print(hasil2)
print(hasil3)
print(hasil4)
\end{lstlisting}

\item Contoh kode pemakaian paket kelas apabila file library ada di dalam folder. Berikut ini adalah pemakaian paket kelas apabila file library ada di dalam folder.
\begin{lstlisting}[caption=Contoh kode pemakaian paket kelas dimana file library ada di dalam folder, label={kelaslib}]
from folder import ngitung

a=100
b=50

hitung = ngitung.Ngitung ( a , b )

hasil1=hitung.Penambahan( )
hasil2=hitung.Pengurangan( )
hasil3=hitung.Perkalian( )
hasil4=hitung.Pembagian( )

print(hasil1)
print(hasil2)
print(hasil3)
print(hasil4)

\end{lstlisting}

    
\end{enumerate}

\subsection{ Ketrampilan Pemrograman }
\begin{enumerate}
    \item Jawaban no. 1
    \lstinputlisting[firstline=151, lastline=183]{src/1174084/1174084.py}
    \item Jawaban no. 2
    \lstinputlisting[firstline=186, lastline=193]{src/1174084/1174084.py}
    \item Jawaban no. 3
    \lstinputlisting[firstline=196, lastline=204]{src/1174084/1174084.py}
    \item Jawaban no. 4
    \lstinputlisting[firstline=207, lastline=211]{src/1174084/1174084.py}
    \item Jawaban no. 5
    \lstinputlisting[firstline=215, lastline=220]{src/1174084/1174084.py}
    \item Jawaban no. 6
    \lstinputlisting[firstline=223, lastline=230]{src/1174084/1174084.py}
    \item Jawaban no. 7
    \lstinputlisting[firstline=233, lastline=240]{src/1174084/1174084.py}
    \item Jawaban no. 8
    \lstinputlisting[firstline=243, lastline=249]{src/1174084/1174084.py}
    \item Jawaban no. 9
    \lstinputlisting[firstline=252, lastline=257]{src/1174084/1174084.py}
    \item Jawaban no. 10
    \lstinputlisting[firstline=260, lastline=275]{src/1174084/1174084.py}
    \item Jawaban no. 11
    \lstinputlisting[firstline=30, lastline=43]{src/1174084/main.py}
    \item Jawaban no. 12
    \lstinputlisting[firstline=45, lastline=60]{src/1174084/main.py}
\end{enumerate}
\subsection{Ketrampilan Penanganan Error}
\begin{enumerate}
	\item Peringatan error yang ditemukan dan penjelasannya serta buat sebuah fungsi try except untuk menanggulangi error.Peringatan error di praktek ketiga ini, yaitu:
	\begin{itemize}
	\item Syntax Errors
	Syntax Errors adalah suatu keadaan saat kode python mengalami kesalahan penulisan. Solusinya adalah memperbaiki penulisan kode yang salah.
	\item Zero Division Error
	ZeroDivisonError adalah exceptions yang terjadi saat eksekusi program menghasilkan perhitungan matematika pembagian dengan angka nol (0). Solusinya adalah tidak membagi suatu yang hasilnya nol.
		
	\item Name Error
	NameError adalah exception yang terjadi saat kode melakukan eksekusi terhadap local name atau global name yang tidak terdefinisi. Solusinya adalah memastikan variabel atau function yang dipanggil ada atau tidak salah ketik.
		
	\item Type Error
	TypeError adalah exception yang terjadi saat dilakukan eksekusi terhadap suatu operasi atau fungsi dengan type object yang tidak sesuai. Solusinya adalah mengkoversi varibelnya sesuai dengan tipe data yang akan digunakan.
	\end{itemize}
	\par Contoh fungsi yang menggunakan try except :
	\lstinputlisting[firstline=279, lastline=285]{src/1174084/1174084.py}
	
\end{enumerate}

%%%%%%%%%%%%%%%%%%%%%%%%%%%%%%%%%%%%%%%%%%%%%%%%%%%%%%%%%%%%%%%%%%%%%%%%%%%%%%%%%%%%%%%%%%%%%%%%%%%%%%%%%%%%%%%%%%%%%%%%%%%%%%%%%%%%%%%%%%%%%%%%%%%
\section{Difa Al Fansha}
\subsection{Jawaban Teori}
\begin{enumerate}
\item Fungsi adalah satu blok program yang terdiri dari nama fungsi, input variabel dan variabel kembalian.
\item Paket atau library adalah kumpulan dari fungsi-fungsi.\\
{Contoh Nomor 1,2 dan 3}
\lstinputlisting[firstline=8, lastline=20]{src/1174076/kalkulator.py}


\item Pengertian Kelas, Objek, Atribut, dan Method.
\begin{itemize}
\item Kelas		: Cetak biru dari objek.
\item Objek		: Segala sesuatu dapat dipanggil objek.
\item Atribut	: Segala sesuatu yang melekat pada objek.
\item Method	: Perilaku dari objek, biasa disebut function.
\end{itemize}
\item Cara memanggil Library
\begin{itemize}
\item mengimport library.
\item File harus dalam satu folder.
\end{itemize}
\item pemakaian paket dengan perintah from kalkulator import Penambahan
\item pemakaian fungsi paket function
\item pemakaian fungsi paket function\\
{Contoh Nomor 4,5,6 dan 7}
\lstinputlisting[firstline=8, lastline=20]{src/1174076/1174076.py}
\end{enumerate}

\subsection{Jawaban Praktek}
\begin{enumerate}
\item Inputan variabel NPM dengan tanda pagar
\lstinputlisting[firstline=22, lastline=57]{src/1174076/1174076.py}

\item Perulangan NPM sampai 87 kali
\lstinputlisting[firstline=59, lastline=69]{src/1174076/1174076.py}

\item Perulangan berupa tiga karakter belakang dari NPM sebanyak penjumlahan tiga dijit
\lstinputlisting[firstline=71, lastline=82]{src/1174076/1174076.py}

\item  fungsi hello word dengan input variabel string
\lstinputlisting[firstline=84, lastline=90]{src/1174076/1174076.py}

\item Mengurutkan NPM
\lstinputlisting[firstline=92, lastline=100]{src/1174076/1174076.py}

\item Fungsi dengan inputan variabel NPM, didalamnya melakukan penjumlahan dari seluruh dijit NPM tersebut
\lstinputlisting[firstline=102, lastline=113]{src/1174076/1174076.py}
 
\item fungsi dengan inputan variabel NPM, didalamnya melakukan melakukan perkalian dari seluruh dijit NPM
\lstinputlisting[firstline=102, lastline=126]{src/1174076/1174076.py}
 
\item  fungsi dengan inputan variabel NPM, Lakukan print NPM anda tapi hanya dijit genap saja
\lstinputlisting[firstline=128, lastline=139]{src/1174076/1174076.py}

\item fungsi dengan inputan variabel NPM, Lakukan print NPM anda tapi hanya dijit ganjil saja
\lstinputlisting[firstline=141, lastline=151]{src/1174076/1174076.py}

\item  fungsi dengan inputan variabel NPM, Lakukan print NPM anda tapi hanya dijit yang termasuk bilangan prima
\lstinputlisting[firstline=153, lastline=172]{src/1174076/1174076.py}

\end{enumerate}
%%%%%%%%%%%%%%%%%%%%%%%%%%%%%%%%%%%%%%%%%%%%%%%%%%%%%%%%%%%%%%%%%%%%%%%%%%%%%%%%%%%%%%%%%%%%%%%%%%%%%%%%%%%%%%%%%%
\section{Alfadian Owen}
\subsection{Teori}
\begin{enumerate}

\item Apa itu fungsi, inputan fingsi dan kembalian fungsi dengan contoh kode program
lainnya!
fungsi adalah sebuah construct untuk menyusun program. Fungsi digunakan untuk memanfaatkan kode lebih dari satu tempat dalam suatu program.
\lstinputlisting[firstline=8, lastline=15]{src/t_1174091.py}

\item Apa itu paket dan cara pemanggilan paket atau library dengan contoh kode program lainnya!
paket adalah kumpulan fungsi-fungsi di dalam sebuah file. cara pemanggilannya yaitu dengan cara "import"
\lstinputlisting[firstline=8, lastline=19]{src/p_1174091.py}
\lstinputlisting[firstline=8, lastline=13]{src/m_1174091.py}

\item Jelaskan Apa itu kelas, apa itu objek, apa itu atribut, apa itu method dan contoh kode program lainnya masing-masing!
kelas adalah blueprint untuk menciptakan instant dari objek, objek adalah instance dari class secara umum merepresentasikan sebuah objek. attribute adalah nilai yang dimiliki objek.	
Terdapat penambahan, pengurangan perkalian, perkalian, pembagian, modulus
\lstinputlisting[firstline=8, lastline=23]{src/c_1174091.py}

\item Jelaskan cara pemanggikan library kelas dari instansiasi dan pemakaiannya dengan contoh program lainnya!
pertama buat file kelas, setelah itu buat file yang isi nya untuk memanggil method" yang ada di kelas
\lstinputlisting[firstline=8, lastline=23]{src/c_1174091.py}
\lstinputlisting[firstline=8, lastline=16]{src/c2_1174091.py}

\item Jelaskan dengan contoh pemakaian paket dengan perintah from kalkulator import Penambahan disertai dengan contoh kode lainnya!
\lstinputlisting[firstline=17, lastline=19]{src/t_1174091.py}

\item Jelaskan dengan contoh kodenya, pemakaian paket fungsi apabila file library
ada di dalam folder!pertama buat file kelas, setelah itu import fungsi nya kedalam file lain
\lstinputlisting[firstline=9, lastline=20]{src/kalkulator_1174091.py}
\lstinputlisting[firstline=17, lastline=19]{src/t_1174091.py}

\item Jelaskan dengan contoh kodenya, pemakaian paket kelas apabila file library ada
di dalam folder!
buat paket kelas terlebih dahulu, setelah itu import file kelas kedalam file lain.
\lstinputlisting[firstline=8, lastline=23]{src/c_1174091.py}
\lstinputlisting[firstline=8, lastline=16]{src/c2_1174091.py}


\end{enumerate}

\subsection{Praktek}
\begin{enumerate}
\item Jawaban soal no 1
\lstinputlisting[firstline=9, lastline=20]{src/1174091.py}
\item Jawaban soal no 2
\lstinputlisting[firstline=7, lastline=11]{src/1174091.py}
\item Jawaban soal no 3
\lstinputlisting[firstline=13, lastline=18]{src/1174091.py}
\item Jawaban soal no 4
\lstinputlisting[firstline=21, lastline=24]{src/1174091.py}
\item Jawaban soal no 5
\lstinputlisting[firstline=27, lastline=31]{src/1174091.py}
\item Jawaban soal no 6
\lstinputlisting[firstline=34, lastline=41]{src/1174091.py}
\item Jawaban soal no 7
\lstinputlisting[firstline=44, lastline=51]{src/1174091.py}
\item Jawaban soal no 8
\lstinputlisting[firstline=54, lastline=59]{src/1174091.py}
\item Jawaban soal no 9
\lstinputlisting[firstline=21, lastline=25]{src/1174091.py}
\item Jawaban soal no 10
\lstinputlisting[firstline=21, lastline=25]{src/1174091.py}
\item Jawaban soal no 11
\lstinputlisting[firstline=21, lastline=25]{src/1174091.py}
\end{enumerate}
<<<<<<< HEAD
<<<<<<< HEAD

<<<<<<< HEAD
%%%%%%%%%%%%%%%%%%%%%%%%%%%%%%%%%%%%%%%%%%%%%%%%%%%%%%%%%%%%%%
\section{Muhammad Abdul Gani Wijaya}
\subsubsection{Teori}
=======

%%%%%%%%%%%%%%%%%%%%%%%%%%%%%%%%%%%%%%%%%%%%%%%%%%%%%%%%%%%%%%%%%
\section{Engelbertus Adiputra Mau Leto/1174078}
\subsubsection{Pemahanan Teori}
>>>>>>> 7d8892247354d94215b72bddd9fc4db0dae75669
\begin{enumerate}
    \item Apa itu fungsi, inputan fungsi dan kembalian fungsi dengan contoh kode program
    lainnya.
    Fungsi adalah bagian dari program yang dapat digunakan ulang.
    Berikut merupakan contoh fungsi dan cara pemanggilannya
<<<<<<< HEAD
    \lstinputlisting[firstline=124, lastline=127]{src/1174071.py}

    Fungsi dapat membaca parameter, parameter adalah nilai yang disediakan kepada fungsi, dimana nilai ini akan menentukan output yang akan dihasilkan fungsi.
    \lstinputlisting[firstline=129, lastline=132]{src/1174071.py}

    Statemen return digunakan untuk keluar dari fungsi. Kita juga dapat menspesifikasikan nilai kembalian.
    \lstinputlisting[firstline=134, lastline=141]{src/1174071.py}
=======
    \lstinputlisting[firstline=124, lastline=127]{src/1174078.py}

    Fungsi dapat membaca parameter, parameter adalah nilai yang disediakan kepada fungsi, dimana nilai ini akan menentukan output yang akan dihasilkan fungsi.
    \lstinputlisting[firstline=129, lastline=132]{src/1174078.py}

    Statemen return digunakan untuk keluar dari fungsi. Kita juga dapat menspesifikasikan nilai kembalian.
    \lstinputlisting[firstline=134, lastline=141]{src/1174078.py}
>>>>>>> 7d8892247354d94215b72bddd9fc4db0dae75669

    \item Apa itu paket dan cara pemanggilan paket atau library dengan contoh kode
    program lainnya.
    Untuk memudahkan dalam pemanggilan fungsi yang di butuhkan, agar dapat dipanggil berulang.
    Cara pemanggilannya
<<<<<<< HEAD
    \lstinputlisting[firstline=143, lastline=144]{src/1174071.py}
=======
    \lstinputlisting[firstline=143, lastline=144]{src/1174078.py}
>>>>>>> 7d8892247354d94215b72bddd9fc4db0dae75669

    \item Jelaskan Apa itu kelas, apa itu objek, apa itu atribut, apa itu method dan
    contoh kode program lainnya masing-masing.
    kelas merupakan sebuah blueprint yang mepresentasikan objek.
    objek adalah hasil cetakan dadri sebuah kelas.
    method adalah suatu upaya yang digunakan oleh object.
<<<<<<< HEAD
    \lstinputlisting[firstline=146, lastline=168]{src/1174071.py}
=======
    \lstinputlisting[firstline=146, lastline=168]{src/1174078.py}
>>>>>>> 7d8892247354d94215b72bddd9fc4db0dae75669

    \item Jelaskan cara pemanggikan library kelas dari instansiasi dan pemakaiannya den-
    gan contoh program lainnya.
    Cara Pemanggilanya 
    \begin{itemize}
        \item pertama import terlebih dahulu filenya.
        \item kemudian buat variabel untuk menampung datanya
        \item setelah itu panggil nama classnya dan panggil methodnya
        \item Gunakan perintah print untuk menampilkan hasilnya

    \end{itemize}
<<<<<<< HEAD
    \lstinputlisting[firstline=170, lastline=175]{src/1174071.py}
=======
    \lstinputlisting[firstline=170, lastline=175]{src/1174078.py}
>>>>>>> 7d8892247354d94215b72bddd9fc4db0dae75669

    \item Jelaskan dengan contoh pemakaian paket dengan perintah from kalkulator im-
    port Penambahan disertai dengan contoh kode lainnya.
    Penggunaan paket from namafile import, itu berfungsi untuk memanggil file dan fungsinya
<<<<<<< HEAD
    \lstinputlisting[firstline=143, lastline=144]{src/1174071.py}
=======
    \lstinputlisting[firstline=143, lastline=144]{src/1174078.py}
>>>>>>> 7d8892247354d94215b72bddd9fc4db0dae75669

    \item Jelaskan dengan contoh kodenya, pemakaian paket fungsi apabila le library
    ada di dalam folder.
    Pemakaian paket adalah perkumpulan fungsi-fungsi. contoh kodenya adalah sebagai berikut :

    \item Jelaskan dengan contoh kodenya, pemakaian paket kelas apabila le library ada
    di dalam folder.
<<<<<<< HEAD
    \lstinputlisting[firstline=184, lastline=184]{src/1174071.py}
=======
    \lstinputlisting[firstline=184, lastline=184]{src/1174078.py}
>>>>>>> 7d8892247354d94215b72bddd9fc4db0dae75669

\end{enumerate}
\subsubsection{Ketrampilan Pemrograman}
\begin{enumerate}
    \item Buatlah fungsi dengan inputan variabel NPM, dan melakukan print luaran huruf
    yang dirangkai dari tanda bintang, pagar atau plus dari NPM kita. Tanda
    bintang untuk NPM mod 3=0, tanda pagar untuk NPM mod 3 =1, tanda plus
    untuk NPM mod3=2.
<<<<<<< HEAD
    \lstinputlisting[firstline=184, lastline=234]{src/1174071.py}
=======
    \lstinputlisting[firstline=184, lastline=234]{src/1174078.py}
>>>>>>> 7d8892247354d94215b72bddd9fc4db0dae75669

    \item Buatlah fungsi dengan inputan variabel berupa NPM. kemudian dengan meng-
    gunakan perulangan mengeluarkan print output sebanyak dua dijit belakang
    NPM.
<<<<<<< HEAD
    \lstinputlisting[firstline=237, lastline=243]{src/1174071.py}
=======
    \lstinputlisting[firstline=237, lastline=243]{src/1174078.py}
>>>>>>> 7d8892247354d94215b72bddd9fc4db0dae75669

    \item Buatlah fungsi dengan dengan input variabel string bernama NPM dan beri
    luaran output dengan perulangan berupa tiga karakter belakang dari NPM se-
    banyak penjumlahan tiga dijit tersebut.
<<<<<<< HEAD
    \lstinputlisting[firstline=245, lastline=255]{src/1174071.py}
=======
    \lstinputlisting[firstline=245, lastline=255]{src/1174078.py}
>>>>>>> 7d8892247354d94215b72bddd9fc4db0dae75669

    \item Buatlah fungsi hello word dengan input variabel string bernama NPM dan
    beri luaran output berupa digit ketiga dari belakang dari variabel NPM meng-
    gunakan akses langsung manipulasi string pada baris ketiga dari variabel NPM.
<<<<<<< HEAD
    \lstinputlisting[firstline=257, lastline=263]{src/1174071.py}

    \item buat fungsi program dengan input variabel NPM dan melakukan print nomor npm satu persatu kebawah.
    \lstinputlisting[firstline=265, lastline=269]{src/1174071.py}
=======
    \lstinputlisting[firstline=257, lastline=263]{src/1174078.py}

    \item buat fungsi program dengan input variabel NPM dan melakukan print nomor npm satu persatu kebawah.
    \lstinputlisting[firstline=265, lastline=269]{src/1174078.py}
>>>>>>> 7d8892247354d94215b72bddd9fc4db0dae75669

    \item Buatlah fungsi dengan inputan variabel NPM, didalamnya melakukan penjum-
    lahan dari seluruh dijit NPM tersebut, wajib menggunakan perulangan dan
    atau kondisi.
<<<<<<< HEAD
    \lstinputlisting[firstline=272, lastline=279]{src/1174071.py}
=======
    \lstinputlisting[firstline=272, lastline=279]{src/1174078.py}
>>>>>>> 7d8892247354d94215b72bddd9fc4db0dae75669

    \item Buatlah fungsi dengan inputan variabel NPM, didalamnya melakukan melakukan
    perkalian dari seluruh dijit NPM tersebut, wajib menggunakan perulangan dan
    atau kondisi.
<<<<<<< HEAD
    \lstinputlisting[firstline=281, lastline=288]{src/1174071.py}

    \item Buatlah fungsi dengan inputan variabel NPM, Lakukan print NPM anda tapi
    hanya dijit genap saja. wajib menggunakan perulangan dan atau kondisi.
    \lstinputlisting[firstline=290, lastline=296]{src/1174071.py}

    \item Buatlah fungsi dengan inputan variabel NPM, Lakukan print NPM anda tapi
    hanya dijit ganjil saja. wajib menggunakan perulangan dan atau kondisi.
    \lstinputlisting[firstline=298, lastline=304]{src/1174071.py}
=======
    \lstinputlisting[firstline=281, lastline=288]{src/1174078.py}

    \item Buatlah fungsi dengan inputan variabel NPM, Lakukan print NPM anda tapi
    hanya dijit genap saja. wajib menggunakan perulangan dan atau kondisi.
    \lstinputlisting[firstline=290, lastline=296]{src/1174078.py}

    \item Buatlah fungsi dengan inputan variabel NPM, Lakukan print NPM anda tapi
    hanya dijit ganjil saja. wajib menggunakan perulangan dan atau kondisi.
    \lstinputlisting[firstline=298, lastline=304]{src/1174078.py}
>>>>>>> 7d8892247354d94215b72bddd9fc4db0dae75669

    \item Buatlah fungsi dengan inputan variabel NPM, Lakukan print NPM anda tapi
    hanya dijit yang termasuk bilangan prima saja. wajib menggunakan perulangan
    dan atau kondisi.
<<<<<<< HEAD
    \lstinputlisting[firstline=306, lastline=320]{src/1174071.py}

    \item Buatlah satu library yang berisi fungsi-fungsi dari nomor diatas dengan nama
    le 3lib.py dan berikan contoh cara pemanggilannya pada le main.py.
    \lstinputlisting[firstline=7, lastline=7]{src/main_gani.py}
=======
    \lstinputlisting[firstline=306, lastline=320]{src/1174078.py}

    \item Buatlah satu library yang berisi fungsi-fungsi dari nomor diatas dengan nama
    le 3lib.py dan berikan contoh cara pemanggilannya pada le main.py.
    \lstinputlisting[firstline=7, lastline=7]{src/main_engel.py}
>>>>>>> 7d8892247354d94215b72bddd9fc4db0dae75669

    \item Buatlah satu library class dengan nama le kelas3lib.py yang merupakan mod-
    ikasi dari fungsi-fungsi nomor diatas dan berikan contoh cara pemanggilannya
    pada le main.py.
<<<<<<< HEAD
    \lstinputlisting[firstline=8, lastline=9]{src/main_gani.py}
=======
    \lstinputlisting[firstline=8, lastline=9]{src/main_engel.py}
>>>>>>> 7d8892247354d94215b72bddd9fc4db0dae75669
    
\end{enumerate}
\subsubsection{Ketrampilan Penanganan Error}
Error yang di dapat dari mengerjakan tugas ini adalah type error, cara menaggulaginya dengan cara mengecheck kembali codingannya
kemudian run kembali aplikasinya
berikut contoh Penggunaan fungsi try dan exception
<<<<<<< HEAD
\lstinputlisting[firstline=177, lastline=182]{src/1174071.py}
=======
%%%%%%%%%%%%%%%%%%%%%%%%%%%%%%%%%%%%%%%%%%%%%%%%%%%%%%%%%%%%%%%%%%%%%%%%%%%%%%%%%%%%%%%%%%%%%%%%%

\section{Ilham Muhammad Ariq D4TI2C 1174087}
\subsection{Pemahaman Teori}
\begin{enumerate}
    \item Apa itu fungsi,inputan fungsi dan kembalian fungsi dengan contoh kode program lainnya.
    
    Fungsi adalah satu blok program yang terdiri dari nama fungsi, input variabel dan variabel kembalian.
    Berikut merupakan contoh fungsi :
    \lstinputlisting[firstline=11, lastline=18]{src/1174087/3lib.py}
    \begin{itemize}
    \item \textbf{cek} merupakan nama fungsi
    \item \textbf{a dan b} merupakan inputan fungsi yang mana adalah inputan yang akan diproses program
    \item \textbf{c} merupakan hasil dari kembalian fungsi tersebut
    \item \textbf{hasil} merupakan output program
    \end{itemize}
    
    \item Apa itu paket dan cara pemanggilan paket atau library dengan contoh kode program lainnya.
    
    Paket merupakan kumpulan dari beberapa fungsi untuk mempermudah pemanggilan fungsi
    cara pemanggilannya :
    \lstinputlisting[firstline=9, lastline=15]{src/1174087/main.py}
    \begin{itemize}
    \item \textbf{import} merupakan cara pemanggilan paket tersebut
    \end{itemize}
	
	\item Jelaskan apa itu kelas, apa itu objek, apa itu atribut, apa itu method dan contoh kode program lainnya masing-masing.
	\begin{itemize}
	\lstinputlisting[firstline=9, lastline=16]{src/1174087/kelas3lib.py}
	\item \textbf{Mahasiswa} = kelas adalah sebuah blueprint yang mepresentasikan objek.
	\item \textbf{self.nama} = atribut adalah variabel yang menyimpan data.
    \item \textbf{NamaMhs} = objek adalah perwujudan dari sebuah kelas.
    \item \textbf{Biodata} = method adalah suatu tindakan yang digunakan oleh object.
	\end{itemize}
	
	\item Jelaskan cara pemanggikan library kelas dari instansiasi dan pemakaiannya dengan contoh program lainnya.
	\lstinputlisting[firstline=19, lastline=22]{src/1174087/main.py}
	\begin{itemize}
	\item pertama import terlebih dahulu filenya.
    \item kemudian buat variabel untuk nenampung nilai data nya
    \item setelah itu panggil nama class dan methodnya
	\end{itemize}		
	
	\item Jelaskan dengan contoh pemakaian paket dengan perintah from kalkulator import Penambahan disertai dengan contoh kode lainnya.
	
	Berikut cara pemakaiannya,
	\lstinputlisting[firstline=25, lastline=29]{src/1174087/main.py}
	\begin{itemize}
	\item import paketnya
	\item panggil fungsi Penambahan
	\end{itemize}
	
	\item Jelaskan dengan contoh kodenya, pemakaian paket fungsi apabila le librar yada di dalam folder.
	
	Pemanggilan fungsi-fungsi pada suatu file, berikut kodenya :
	
	\lstinputlisting[firstline=32, lastline=38]{src/1174087/main.py}

	\item Jelaskan dengan contoh kodenya, pemakaian paket kelas apabila le library ada
    di dalam folder.
	
	Pemanggilan nama kelas pada suatu file, berikut kodenya :
	
	\lstinputlisting[firstline=41, lastline=44]{src/1174087/main.py}

\end{enumerate}   
\subsection{Ketrampilan Pemrograman}
\begin{enumerate}
    \item Buatlah fungsi dengan inputan variabel NPM, dan melakukan print luaran huruf yang dirangkai dari tanda bintang, pagar atau plus dari NPM kita. Tanda bintang untuk NPM mod 3=0, tanda pagar untuk NPM mod 3 =1, tanda plus untuk NPM mod3=2.
    \lstinputlisting[firstline=22, lastline=55]{src/1174087/3lib.py}
    
     \item Buatlah fungsi dengan inputan variabel berupa NPM. kemudian dengan menggunakan perulangan mengeluarkan print output sebanyak dua dijit belakang NPM.
     \lstinputlisting[firstline=58, lastline=65]{src/1174087/3lib.py}

	\item Buatlah fungsi dengan dengan input variabel string bernama NPM dan beri luaran output dengan perulangan berupa tiga karakter belakang dari NPM sebanyak penjumlahan tiga dijit tersebut.
	\lstinputlisting[firstline=68, lastline=77]{src/1174087/3lib.py}
	
	\item Buatlah fungsi hello word dengan input variabel string bernama NPM dan beri luaran output berupa digit ketiga dari belakang dari variabel NPM menggunakan akses langsung manipulasi string pada baris ketiga dari variabel NPM.
	\lstinputlisting[firstline=80, lastline=84]{src/1174087/3lib.py}
	
	\item buat fungsi program dengan input variabel NPM dan melakukan print nomor npm satu persatu kebawah.
	\lstinputlisting[firstline=87, lastline=93]{src/1174087/3lib.py}
	
	\item Buatlah fungsi dengan inputan variabel NPM, didalamnya melakukan penjumlahan dari seluruh dijit NPM tersebut, wajib menggunakan perulangan dan atau kondisi.
    \lstinputlisting[firstline=96, lastline=105]{src/1174087/3lib.py}
    
    \item Buatlah fungsi dengan inputan variabel NPM, didalamnya melakukan melakukan perkalian dari seluruh dijit NPM tersebut, wajib menggunakan perulangan dan atau kondisi.
     \lstinputlisting[firstline=108, lastline=117]{src/1174087/3lib.py}
     
     \item Buatlah fungsi dengan inputan variabel NPM, Lakukan print NPM anda tapi hanya dijit genap saja. wajib menggunakan perulangan dan atau kondisi.
      \lstinputlisting[firstline=120, lastline=128]{src/1174087/3lib.py}
      
       \item Buatlah fungsi dengan inputan variabel NPM, Lakukan print NPM anda tapi hanya dijit ganjil saja. wajib menggunakan perulangan dan atau kondisi.
       \lstinputlisting[firstline=131, lastline=138]{src/1174087/3lib.py}
       
       \item Buatlah fungsi dengan inputan variabel NPM, Lakukan print NPM anda tapi hanya dijit yang termasuk bilangan prima saja. wajib menggunakan perulangan dan atau kondisi.
       \lstinputlisting[firstline=141, lastline=158]{src/1174087/3lib.py}
       
       \item Buatlah satu library yang berisi fungsi-fungsi dari nomor diatas dengan nama file 3lib.py dan berikan contoh cara pemanggilannya pada file main.py.
       \lstinputlisting[firstline=7, lastline=44]{src/1174087/main.py}
       
       \item Buatlah satu library class dengan nama file kelas3lib.py yang merupakan modifikasi dari fungsi-fungsi nomor diatas dan berikan contoh cara pemanggilannya pada file main.py
    	\lstinputlisting[firstline=9, lastline=16]{src/1174087/kelas3lib.py}
\end{enumerate}
\subsection{Keterampilan Penanganan Error}
Error yang di dapat dari mengerjakan tugas ini adalah type error, cara menaggulaginya dengan cara mengecheck kembali codingannya kemudian run kembali aplikasinya berikut contoh Penggunaan fungsi try dan exception
\lstinputlisting[firstline=161, lastline=167]{src/1174087/3lib.py}
%%%%%%%%%%%%%%%%%%%%%%%%%%%%%%%%%%%%%%%%%%%%%%%%%%%%%%%%%%%%%%%%%%%%%%%%%%%%%%%%%%%%%%%%%%%%
\section{Ilham Muhammad Ariq D4TI2C 1174087}
\subsection{Pemahaman Teori}
\begin{enumerate}
    \item Apa itu fungsi,inputan fungsi dan kembalian fungsi dengan contoh kode program lainnya.
    
    Fungsi adalah satu blok program yang terdiri dari nama fungsi, input variabel dan variabel kembalian.
    Berikut merupakan contoh fungsi :
    \lstinputlisting[firstline=11, lastline=18]{src/1174087/3lib.py}
    \begin{itemize}
    \item \textbf{cek} merupakan nama fungsi
    \item \textbf{a dan b} merupakan inputan fungsi yang mana adalah inputan yang akan diproses program
    \item \textbf{c} merupakan hasil dari kembalian fungsi tersebut
    \item \textbf{hasil} merupakan output program
    \end{itemize}
    
    \item Apa itu paket dan cara pemanggilan paket atau library dengan contoh kode program lainnya.
    
    Paket merupakan kumpulan dari beberapa fungsi untuk mempermudah pemanggilan fungsi
    cara pemanggilannya :
    \lstinputlisting[firstline=9, lastline=15]{src/1174087/main.py}
    \begin{itemize}
    \item \textbf{import} merupakan cara pemanggilan paket tersebut
    \end{itemize}
	
	\item Jelaskan apa itu kelas, apa itu objek, apa itu atribut, apa itu method dan contoh kode program lainnya masing-masing.
	\begin{itemize}
	\lstinputlisting[firstline=9, lastline=16]{src/1174087/kelas3lib.py}
	\item \textbf{Mahasiswa} = kelas adalah sebuah blueprint yang mepresentasikan objek.
	\item \textbf{self.nama} = atribut adalah variabel yang menyimpan data.
    \item \textbf{NamaMhs} = objek adalah perwujudan dari sebuah kelas.
    \item \textbf{Biodata} = method adalah suatu tindakan yang digunakan oleh object.
	\end{itemize}
	
	\item Jelaskan cara pemanggikan library kelas dari instansiasi dan pemakaiannya dengan contoh program lainnya.
	\lstinputlisting[firstline=19, lastline=22]{src/1174087/main.py}
	\begin{itemize}
	\item pertama import terlebih dahulu filenya.
    \item kemudian buat variabel untuk nenampung nilai data nya
    \item setelah itu panggil nama class dan methodnya
	\end{itemize}		
	
	\item Jelaskan dengan contoh pemakaian paket dengan perintah from kalkulator import Penambahan disertai dengan contoh kode lainnya.
	
	Berikut cara pemakaiannya,
	\lstinputlisting[firstline=25, lastline=29]{src/1174087/main.py}
	\begin{itemize}
	\item import paketnya
	\item panggil fungsi Penambahan
	\end{itemize}
	
	\item Jelaskan dengan contoh kodenya, pemakaian paket fungsi apabila le librar yada di dalam folder.
	
	Pemanggilan fungsi-fungsi pada suatu file, berikut kodenya :
	
	\lstinputlisting[firstline=32, lastline=38]{src/1174087/main.py}

	\item Jelaskan dengan contoh kodenya, pemakaian paket kelas apabila le library ada
    di dalam folder.
	
	Pemanggilan nama kelas pada suatu file, berikut kodenya :
	
	\lstinputlisting[firstline=41, lastline=44]{src/1174087/main.py}

\end{enumerate}   
\subsection{Ketrampilan Pemrograman}
\begin{enumerate}
    \item Buatlah fungsi dengan inputan variabel NPM, dan melakukan print luaran huruf yang dirangkai dari tanda bintang, pagar atau plus dari NPM kita. Tanda bintang untuk NPM mod 3=0, tanda pagar untuk NPM mod 3 =1, tanda plus untuk NPM mod3=2.
    \lstinputlisting[firstline=22, lastline=55]{src/1174087/3lib.py}
    
     \item Buatlah fungsi dengan inputan variabel berupa NPM. kemudian dengan menggunakan perulangan mengeluarkan print output sebanyak dua dijit belakang NPM.
     \lstinputlisting[firstline=58, lastline=65]{src/1174087/3lib.py}

	\item Buatlah fungsi dengan dengan input variabel string bernama NPM dan beri luaran output dengan perulangan berupa tiga karakter belakang dari NPM sebanyak penjumlahan tiga dijit tersebut.
	\lstinputlisting[firstline=68, lastline=77]{src/1174087/3lib.py}
	
	\item Buatlah fungsi hello word dengan input variabel string bernama NPM dan beri luaran output berupa digit ketiga dari belakang dari variabel NPM menggunakan akses langsung manipulasi string pada baris ketiga dari variabel NPM.
	\lstinputlisting[firstline=80, lastline=84]{src/1174087/3lib.py}
	
	\item buat fungsi program dengan input variabel NPM dan melakukan print nomor npm satu persatu kebawah.
	\lstinputlisting[firstline=87, lastline=93]{src/1174087/3lib.py}
	
	\item Buatlah fungsi dengan inputan variabel NPM, didalamnya melakukan penjumlahan dari seluruh dijit NPM tersebut, wajib menggunakan perulangan dan atau kondisi.
    \lstinputlisting[firstline=96, lastline=105]{src/1174087/3lib.py}
    
    \item Buatlah fungsi dengan inputan variabel NPM, didalamnya melakukan melakukan perkalian dari seluruh dijit NPM tersebut, wajib menggunakan perulangan dan atau kondisi.
     \lstinputlisting[firstline=108, lastline=117]{src/1174087/3lib.py}
     
     \item Buatlah fungsi dengan inputan variabel NPM, Lakukan print NPM anda tapi hanya dijit genap saja. wajib menggunakan perulangan dan atau kondisi.
      \lstinputlisting[firstline=120, lastline=128]{src/1174087/3lib.py}
      
       \item Buatlah fungsi dengan inputan variabel NPM, Lakukan print NPM anda tapi hanya dijit ganjil saja. wajib menggunakan perulangan dan atau kondisi.
       \lstinputlisting[firstline=131, lastline=138]{src/1174087/3lib.py}
       
       \item Buatlah fungsi dengan inputan variabel NPM, Lakukan print NPM anda tapi hanya dijit yang termasuk bilangan prima saja. wajib menggunakan perulangan dan atau kondisi.
       \lstinputlisting[firstline=141, lastline=158]{src/1174087/3lib.py}
       
       \item Buatlah satu library yang berisi fungsi-fungsi dari nomor diatas dengan nama file 3lib.py dan berikan contoh cara pemanggilannya pada file main.py.
       \lstinputlisting[firstline=7, lastline=44]{src/1174087/main.py}
       
       \item Buatlah satu library class dengan nama file kelas3lib.py yang merupakan modifikasi dari fungsi-fungsi nomor diatas dan berikan contoh cara pemanggilannya pada file main.py
    	\lstinputlisting[firstline=9, lastline=16]{src/1174087/kelas3lib.py}
\end{enumerate}
\subsection{Keterampilan Penanganan Error}
Error yang di dapat dari mengerjakan tugas ini adalah type error, cara menaggulaginya dengan cara mengecheck kembali codingannya kemudian run kembali aplikasinya berikut contoh Penggunaan fungsi try dan exception
\lstinputlisting[firstline=161, lastline=167]{src/1174087/3lib.py}
=======
%%%%%%%%%%%%%%%%%%%%%%%%%%%%%%%%%%%%%%%%%%%%%%%%%%%%%%%%%%%%%%%%%%%%%%%%%%%%%%%%%%%%%%%%%%%%%%%%%%%%%%%%%%%%%
\section{Alvan Alvanzah/1174077}
\subsection{Pemahaman Teori}

\begin{enumerate}
\item Apa itu fungsi, inputan fungsi dan kembalian fungsi dengan contoh kode program
lainnya.
\begin{itemize}
\item Fungsi adalah blok program untuk melakukan tugas-tugas tertentu yang dilakukan berulang dan dapat digunakan berulang kali dari tempat lain di dalam program. namaFungsi dari fungsi yang dibuat.
\lstinputlisting[firstline=10, lastline=10]{src/1174077/1174077.py}
		
\item Inputan fungsi adalah inputan yang berasal dari luar fungsi yang akan di proses di dalam fungsi itu sendiri. inputanFungsi dari inputan fungsi yang diterima dari luar fungsi namaFungsi.
\lstinputlisting[firstline=10, lastline=10]{src/1174077/1174077.py}

\item Kembalian fungsi adalah untuk mengembalikan suatu nilai ekspresi dari proses yang dilakukan fungsi. return inputanFungsi merupakan kembalian dari fungsi namaFungsi.
\lstinputlisting[firstline=11, lastline=11]{src/1174077.py}
\end{itemize}

\item Apa itu paket dan cara pemanggilan paket atau library dengan contoh kode program lainnya. Untuk memudahkan dalam pemanggilan fungsi yang di butuhkan, agar dapat dipanggil berulang. Cara pemanggilannya
\lstinputlisting[firstline=17, lastline=18]{src/1174077/1174077.py}

\item Pengertian kelas, objek, atribut, method, dan contoh kode programnya.

\begin{itemize}
\item Kelas
\par Kelas adalah cetak biru atau prototipe dari objek dimana kita mendefinisikan atribut dari suatu objek.
Contoh penggunaan kelas di python.
\lstinputlisting[ firstline=21, lastline=40]{src/1174077/1174077.py}
		
\item Objek
\par Objek adalah instansi atau perwujudan dari sebuah kelas. Contoh penggunaan objek di python.
\lstinputlisting[firstline=34, lastline=35]{src/1174077/1174077.py}
		
\item Atribut
\par Atribut adalah variabel yang menyimpan data yang berhubungan dengan kelas dan objeknya. Contoh penggunaan atribut di python.
\lstinputlisting[firstline=22, lastline=22]{src/1174077/1174077.py}
		
\item Method
\par Metode adalah fungsi yang didefinisikan di dalam suatu kelas. Contoh penggunaan method di python.
\lstinputlisting[firstline=29, lastline=32]{src/1174077/1174077.py}
\end{itemize}

\item Cara pemanggilan library kelas, dan contoh kode programnya.
	
Berikut ini adalah cara pemanggilan library kelas dari instansi dan pemakaiannya. Library kelasnya adalah Mahasiswa dari file Mahasiswa.py. Lalu dipanggil dengan import. Kemudian instansi dengan mhs1 dan mhs1, dengan 2 parameter. Contoh pemanggilan library kelas dari instansi dan pemakaiannya.
\lstinputlisting[firstline=43, lastline=51]{src/1174077/1174077.py}

\item Penjelasan pemakaian paket disertai dengan contoh kode programnya.

Berikut ini adalah contoh pemakaian paket dengan perintah from kalkulator import Penambahan. Setelah mengimport paketnya, lalu panggil fungsi penambahannya. Contoh pemakaian paket dengan perintah from kalkulator import Penambahan.
\lstinputlisting[firstline=54, lastline=57]{src/1174077/1174077.py}

\item Contoh kode pemakaian paket fungsi apabila file library ada di dalam folder. Berikut ini adalah pemakaian paket fungsi apabila file library ada di dalam folder. Contoh kode pemakaian paket fungsi dimana file library ada di dalam folder.
\lstinputlisting[firstline=60, lastline=73]{src/1174077/1174077.py}

\item Contoh kode pemakaian paket kelas apabila file library ada di dalam folder. Berikut ini adalah pemakaian paket kelas apabila file library ada di dalam folder. Contoh kode pemakaian paket kelas dimana file library ada di dalam folder.
\lstinputlisting[firstline=76, lastline=84]{src/1174077/1174077.py}
\end{enumerate}

\subsection{Ketrampilan Pemrograman}
\begin{enumerate}
\item Jawaban soal No. 1
\lstinputlisting[firstline=88, lastline=121]{src/1174077/1174077.py}
	
\item Jawaban soal No. 2
\lstinputlisting[firstline=124, lastline=131]{src/1174077/1174077.py}
	
\item Jawaban soal No. 3
\lstinputlisting[firstline=134, lastline=142]{src/1174077/1174077.py}
	
\item Jawaban soal No. 4
\lstinputlisting[firstline=145, lastline=149]{src/1174077/1174077.py}
	
\item Jawaban soal No. 5
\lstinputlisting[firstline=152, lastline=157]{src/1174077/1174077.py}
	
\item Jawaban soal No. 6
\lstinputlisting[firstline=160, lastline=168]{src/1174077/1174077.py}
	
\item Jawaban soal No. 7
\lstinputlisting[firstline=171, lastline=179]{src/1174077/1174077.py}
	
\item Jawaban soal No. 8
\lstinputlisting[firstline=182, lastline=189]{src/1174077/1174077.py}
	
\item Jawaban soal No. 9
\lstinputlisting[firstline=192, lastline=198]{src/1174077/1174077.py}
	
\item Jawaban soal No. 10
\lstinputlisting[firstline=201, lastline=217]{src/1174077/1174077.py}
	
\item Jawaban soal No. 11
\lstinputlisting[firstline=30, lastline=43]{src/1174077/main.py}
	
\item Jawaban soal No. 12
\lstinputlisting[firstline=45, lastline=60]{src/1174077/main.py}
	
\end{enumerate}
\subsection{Ketrampilan Penanganan Error}
\begin{enumerate}
\item Peringatan error yang ditemukan dan penjelasannya serta buat sebuah fungsi try except untuk menanggulangi error.
	
Peringatan error di praktek ketiga ini, yaitu:
\begin{itemize}
\item Syntax Errors
Syntax Errors adalah suatu keadaan saat kode python mengalami kesalahan penulisan. Solusinya adalah memperbaiki penulisan kode yang salah.
		
\item Zero Division Error
ZeroDivisonError adalah exceptions yang terjadi saat eksekusi program menghasilkan perhitungan matematika pembagian dengan angka nol (0). Solusinya adalah tidak membagi suatu yang hasilnya nol.
		
\item Name Error
NameError adalah exception yang terjadi saat kode melakukan eksekusi terhadap local name atau global name yang tidak terdefinisi. Solusinya adalah memastikan variabel atau function yang dipanggil ada atau tidak salah ketik.
		
\item Type Error
TypeError adalah exception yang terjadi saat dilakukan eksekusi terhadap suatu operasi atau fungsi dengan type object yang tidak sesuai. Solusinya adalah mengkoversi varibelnya sesuai dengan tipe data yang akan digunakan.
\end{itemize}
	
Contoh fungsi yang menggunakan try except
\lstinputlisting[firstline=225, lastline=231]{src/1174077/1174077.py}
\end{enumerate}	
%%%%%%%%%%%%%%%%%%%%%%%%%%%%%%%%%%%%%%%%%%%%%%%%%
\section{Engelbertus Adiputra Mau Leto/1174078}
\subsubsection{Pemahanan Teori}
\begin{enumerate}
    \item Apa itu fungsi, inputan fungsi dan kembalian fungsi dengan contoh kode program
    lainnya.
    Fungsi adalah bagian dari program yang dapat digunakan ulang.
    Berikut merupakan contoh fungsi dan cara pemanggilannya
    \lstinputlisting[firstline=124, lastline=127]{src/1174078.py}

    Fungsi dapat membaca parameter, parameter adalah nilai yang disediakan kepada fungsi, dimana nilai ini akan menentukan output yang akan dihasilkan fungsi.
    \lstinputlisting[firstline=129, lastline=132]{src/1174078.py}

    Statemen return digunakan untuk keluar dari fungsi. Kita juga dapat menspesifikasikan nilai kembalian.
    \lstinputlisting[firstline=134, lastline=141]{src/1174078.py}

    \item Apa itu paket dan cara pemanggilan paket atau library dengan contoh kode
    program lainnya.
    Untuk memudahkan dalam pemanggilan fungsi yang di butuhkan, agar dapat dipanggil berulang.
    Cara pemanggilannya
    \lstinputlisting[firstline=143, lastline=144]{src/1174078.py}

    \item Jelaskan Apa itu kelas, apa itu objek, apa itu atribut, apa itu method dan
    contoh kode program lainnya masing-masing.
    kelas merupakan sebuah blueprint yang mepresentasikan objek.
    objek adalah hasil cetakan dadri sebuah kelas.
    method adalah suatu upaya yang digunakan oleh object.
    \lstinputlisting[firstline=146, lastline=168]{src/1174078.py}

    \item Jelaskan cara pemanggikan library kelas dari instansiasi dan pemakaiannya den-
    gan contoh program lainnya.
    Cara Pemanggilanya 
    \begin{itemize}
        \item pertama import terlebih dahulu filenya.
        \item kemudian buat variabel untuk menampung datanya
        \item setelah itu panggil nama classnya dan panggil methodnya
        \item Gunakan perintah print untuk menampilkan hasilnya

    \end{itemize}
    \lstinputlisting[firstline=170, lastline=175]{src/1174078.py}

    \item Jelaskan dengan contoh pemakaian paket dengan perintah from kalkulator im-
    port Penambahan disertai dengan contoh kode lainnya.
    Penggunaan paket from namafile import, itu berfungsi untuk memanggil file dan fungsinya
    \lstinputlisting[firstline=143, lastline=144]{src/1174078.py}

    \item Jelaskan dengan contoh kodenya, pemakaian paket fungsi apabila le library
    ada di dalam folder.
    Pemakaian paket adalah perkumpulan fungsi-fungsi. contoh kodenya adalah sebagai berikut :

    \item Jelaskan dengan contoh kodenya, pemakaian paket kelas apabila le library ada
    di dalam folder.
    \lstinputlisting[firstline=184, lastline=184]{src/1174078.py}

\end{enumerate}
\subsubsection{Ketrampilan Pemrograman}
\begin{enumerate}
    \item Buatlah fungsi dengan inputan variabel NPM, dan melakukan print luaran huruf
    yang dirangkai dari tanda bintang, pagar atau plus dari NPM kita. Tanda
    bintang untuk NPM mod 3=0, tanda pagar untuk NPM mod 3 =1, tanda plus
    untuk NPM mod3=2.
    \lstinputlisting[firstline=184, lastline=234]{src/1174078.py}

    \item Buatlah fungsi dengan inputan variabel berupa NPM. kemudian dengan meng-
    gunakan perulangan mengeluarkan print output sebanyak dua dijit belakang
    NPM.
    \lstinputlisting[firstline=237, lastline=243]{src/1174078.py}

    \item Buatlah fungsi dengan dengan input variabel string bernama NPM dan beri
    luaran output dengan perulangan berupa tiga karakter belakang dari NPM se-
    banyak penjumlahan tiga dijit tersebut.
    \lstinputlisting[firstline=245, lastline=255]{src/1174078.py}

    \item Buatlah fungsi hello word dengan input variabel string bernama NPM dan
    beri luaran output berupa digit ketiga dari belakang dari variabel NPM meng-
    gunakan akses langsung manipulasi string pada baris ketiga dari variabel NPM.
    \lstinputlisting[firstline=257, lastline=263]{src/1174078.py}

    \item buat fungsi program dengan input variabel NPM dan melakukan print nomor npm satu persatu kebawah.
    \lstinputlisting[firstline=265, lastline=269]{src/1174078.py}

    \item Buatlah fungsi dengan inputan variabel NPM, didalamnya melakukan penjum-
    lahan dari seluruh dijit NPM tersebut, wajib menggunakan perulangan dan
    atau kondisi.
    \lstinputlisting[firstline=272, lastline=279]{src/1174078.py}

    \item Buatlah fungsi dengan inputan variabel NPM, didalamnya melakukan melakukan
    perkalian dari seluruh dijit NPM tersebut, wajib menggunakan perulangan dan
    atau kondisi.
    \lstinputlisting[firstline=281, lastline=288]{src/1174078.py}

    \item Buatlah fungsi dengan inputan variabel NPM, Lakukan print NPM anda tapi
    hanya dijit genap saja. wajib menggunakan perulangan dan atau kondisi.
    \lstinputlisting[firstline=290, lastline=296]{src/1174078.py}

    \item Buatlah fungsi dengan inputan variabel NPM, Lakukan print NPM anda tapi
    hanya dijit ganjil saja. wajib menggunakan perulangan dan atau kondisi.
    \lstinputlisting[firstline=298, lastline=304]{src/1174078.py}

    \item Buatlah fungsi dengan inputan variabel NPM, Lakukan print NPM anda tapi
    hanya dijit yang termasuk bilangan prima saja. wajib menggunakan perulangan
    dan atau kondisi.
    \lstinputlisting[firstline=306, lastline=320]{src/1174078.py}

    \item Buatlah satu library yang berisi fungsi-fungsi dari nomor diatas dengan nama
    le 3lib.py dan berikan contoh cara pemanggilannya pada le main.py.
    \lstinputlisting[firstline=7, lastline=7]{src/main_engel.py}

    \item Buatlah satu library class dengan nama le kelas3lib.py yang merupakan mod-
    ikasi dari fungsi-fungsi nomor diatas dan berikan contoh cara pemanggilannya
    pada le main.py.
    \lstinputlisting[firstline=8, lastline=9]{src/main_engel.py}
    
\end{enumerate}
\subsubsection{Ketrampilan Penanganan Error}
Error yang di dapat dari mengerjakan tugas ini adalah type error, cara menaggulaginya dengan cara mengecheck kembali codingannya
kemudian run kembali aplikasinya
berikut contoh Penggunaan fungsi try dan exception
\lstinputlisting[firstline=177, lastline=182]{src/1174078.py}


