\documentclass[lipt]{Article}
\begin{document}
\begin{center}
Aulyardha Anindita
D4 TI 2 C
1174054
\end{center}
\chapter{Resume}
\section{Sejarah Python}
Python adalah bahasa pemrograman yang mendukung multi paradigma pemrograman seperti pada pemrograman berorientasi objek, pemrograman imperatif, dan pemrograman fungsional. Python juga pada umumnya digunakan sebagai bahasa skrip walaupun pada praktiknya penggunaan python lebih luas. 

Bahasa pemrograman python dikembangkan oleh Guido van Rossum pada tahun 1990 di Stichting Mathematisch Centrum (CWI), Amsterdam yang merupakan kelanjutan dari bahasa pemrograman ABC dimana merupakan versi terakhir yang dikeluarkan oleh CWI adalah 1.2. Nama Python sendiri dipilih oleh Guido sebagai nama bahasa ciptaannya karena rasa cintanya pada acara televisi Monty Python’s Flying Circus.

Sekitar tahun 1995, Guido pindah ke CNRI di Virginia Amerika sambil dia melanjutkan pengembangan Python. Disinilah dia merilis beberapa versi dari python. Versi terakhir yang dikeluarkan saat itu adalah 1.6. Dan sekitar tahun 2000,  Guido dan timnya pindah ke BeOpen.com. BeOpen.com merupakan sebuah perusahaan komersial yang membentuk BeOpen PythonLabs. Python 2.0 dikeluarkan oleh BeOpen. Setelah mereka mengeluarkan Python 2.0, Guido dan beberapa anggota tim PythonLabs pindah ke DigitalCreations.
Dan pada tahun 2001, dibentuklah Organisasi Python yaitu Python Software  Foundation(PSF). PSF adalah suatu organisasi nirlaba yang dibuat untuk semua hal yang berkaitan dengan hak intelektual Python. 


\section{Perbedaan Python 2 dan Python 3}

Pada python 2 kita bisa menggunakan tanda kurung atau tidak sedangkan pada python3 kita wajib menggunakan tanda kurung, jika tidak maka kita akan mendapatkan hasil error.

Pada python 2 dalam melakukan sebuah inputan kita harus menggunakan “raw_input(‘teks’)”, sedangkan untuk python 3 kita hanya perlu menggunakan syntax “input(‘teks’)” saja.

Pada python2 dilengkapi dengan berbagai fitur programatikal seperti sysle-detecting garbage collector untuk mengotomasi manajemen memori, peningkatan dukungan untuk Unicode, list comprehension untuk membuat list berdasarkan dari list yang sudah ada. Sedangkan pada python 3 adalah melakukan perapian pada codebase dan menghapuskan duplikat atau redudansi.

\section{Implementasi dan Penggunaan Python di Perusahaan Dunia}
1.	Google adalah perusahaan besar yang menggunakan banyak kode Python di dalam mesin pencarinya. Dan mesin pencari google adalah yang paling terkenal di dunia.\\
2.	Youtube, situs video terbesar dan terpopuler di dunia, sebagian besar kodenya ditulis dalam bahasa Python.\\
3.	Facebook, media sosial terbesar di dunia, menggunakan Tornado, sebuah framework Python untuk menampilkan timeline.\\
4.	Instagram, siapa yang tidak kenal. Instagram menggunakan Django, framework python sebagai mesin pengolah sisi server dari aplikasinya.\\
5.	Pinterest, banyak menggunakan python untuk membangun aplikasinya.\\
6.	Dropbox, barangkali Anda adalah salah seorang pengguna layanan ini. Dropbox menggunakan python baik di sisi server maupun di sisi pengguna layanannya.\\
7.	Quora, salah satu situs tanya jawab terbesar di dunia, dibangun menggunakan Python.\\
8.	NASA, badan antariksa Amerika ini menggunakan Python untuk bidang sainsnya.\\
9.	NSA, badan mata – mata Amerika banyak menggunakan Python untuk analisa kriptografi dan intelijen.\\
10.	Industrial Light & Magic, Pixar, banyak menggunakan Python dalam animasi movie.\\
11.	Blender, Maya, software pembuat animasi 3D terkenal, menggunakan Python sebagai salah satu bahasa skrip pemrogramannya.\\
12.	Raspberry Pi, komputer mini yang banyak digunakan sebagai mikrokontroller, menggunakan Python sebagai bahasa utamanya.\\
13.	ESRI, produsen terkenal pembuat software pemetaan GIS banyak menggunakan Python di produknya.\\

\chapter{Instalasi}
\section{Proses Instalasi Python}
1. Download terlebih dahulu file Phyton nya. jika sudah didownload, maka kita masuk ke proses instalasi dengan mendouble klik pada file pyhton nya.\\
2. Centang Install launcher for all user untuk mengaktifkan python pada semua user Windows dan centang Python 3.6 to PATH untuk menambah path command Python. Kemudian klik Install Now. Klik Yes saat muncul notifikasi User Account Control. \\
3. Tunggu sampai proses instalasi selesai\\
4. Instalasi pyhton berhasil\\
5. Untuk mengetahui apakah pyhton nya sudah berjalan apa tidak yaitu dengan masuk ke Command Prompt dan ketikkan Phyton, jika ada berarti sudah tersambung\\

\section{Proses Instalasi Anaconda}
1. Download terlebih dahulu file Anaconda. Jika sudah didownload, maka langsung saja masuk ke proses instalasi dengan mendouble klik pada file Anacondanya.\\
2. Akan muncul tampilan pertama, lalu pilih next\\
3. Kemudian read lisensi dan klik I Agree\\
4. Kemudian pilih tempat penyimpanan nya, bagusnya yang default saja lalu pilih next\\
5. Kemudian pilih add anaconda to PATH atau tidak. Pilih apakah akan mendaftarkan Anaconda sebagai default Python 3.7?. Kecuali kita berencana menginstal dan menjalankan beberapa versi Anaconda, atau beberapa versi Python, biarkan default dan biarkan kotak ini dicentang. kemudian klik next.\\
6. Klik tombol Install. Jika Kita ingin melihat packages Anaconda yang sedang dipasang, klik Show Details, lalu pilih Next.\\
7. Untuk menginstal VS Code, klik tombol Install Microsoft VS Code. Setelah instalasi selesai, klik tombol Next Atau untuk menginstal Anaconda tanpa VS code, klik tombol skip. Memasang VS code dengan pemasang Anaconda membutuhkan koneksi internet. Pengguna offline mungkin dapat menemukan pemasang offline VS Code dari Microsoft.\\
8. Setelah proses instalasi berhasil, Kita akan melihat kotak dialog "Thanks for installing Anaconda3" lalu pilih Finish\\

\section{Cara Pemakaian Script dan Interpreter Python}
\subsection{Script}
1. Gunakan teks editor untuk menulis skrip\\
2. kemudian simpan dengan nama yang kalian inginkan\\
3. Kemudian untuk menjalankan skripnya, gunakan perintah berikut: python nama_skrip.py\\
4. Skrip python diterjemahkan ke dalam kode biner oleh (intepreter) python, sehingga komputer dapat mengerti arti perintah tersebut sehingga komputer mengerjakan perintah tersebut.\\
\subsection{Interpreter Python}
1. Membuka interpreter python pada submenu dari Aplikasi Python yang terdapat pada All Programs. Untuk keluar dari interpreter, ketik Ctrl+D / Ctrl+Q atau menggunakan perintah quit().\\
2. help() untuk menampilkan bantuan informasi kita dapat menggunakan perintah help(). Perintah help() dapat digunakan dengan 2 cara, yaitu dengan menggunakannya beserta object yang diinginkan, contohnya help(int). Kedua adalah dengan mengetikan perintah help() didalam interpreter yang akan merubah mode interpreter ‘>>>’ menjadi mode ‘help>’.\\
3. Setelah kita berada dalam mode ‘help>’ kita dapat langsung menggunakannya dengan memasukan keywords atau object yang diinginkan. Contohnya adalah keywords.\\
4. Jika kita mengetik salah satu keywords, maka interpreter akan memberikan informasi yang bersangkutan dengan keywords tersebut. Contohnya adalah if.\\
5. Disamping keywords kita juga dapat mendapatkan informasi tentang topics. Untuk mengetahui macam-macam topics, cukup dengan mengetikan perintah topics kedalam mode ‘help>’\\
6. Perintah topics memberikan informasi yang berguna kepada kita mengenai bahasa pemrograman python. Selain cara help() diatas, kita juga dapat menggunakan cara yang kedua yaitu langsung bersama object yang diinginkan misalnya adalah string. Untuk mencobanya ketik help(str)\\
7. Dengan menggunakan perintah help(str) kita dapat mengexplore object string beserta attribut dan method-method yang dimilikinya. Dan hal tersebut berlaku untuk semua keywords python. Tanda titik dua diatas menandakan informasi yang disampaikan masih bersambung, untuk mengetahui informasi selanjutnya tekan space sampai muncul kata (END)\\

\section{Cara Pemakaian Spyder Termasuk Variable Explorer}
1. Variable Explorer Spyder menawarkan dukungan bawaan untuk mengedit daftar, string, kamus, array NumPy, Pandas DataFrames, dan banyak lagi, dan dapat juga histogram, plot, atau bahkan menampilkan beberapa di antaranya sebagai gambar RGB.\\
2. Variable Explorer memiliki editor khusus untuk serangkaian objek Python internal dan pihak ketiga yang umum, dan dapat melihat, mengedit, dan mengintrospeksi objek paling arbitrer secara mendalam melalui Penjelajah Objek yang lebih umum\\

\chapter{Mencoba Phyton}
print ("Hello World Python!")

\chapter{Identasi}
Indentasi merupakan keluarnya suatu teks/naskah dari batas kiri, batas kanan atau keduanya. Kita dapat mengatur indentasi hanya pada baris pertama (first line indent) dari paragraf atau mengatur indentasi mengandung (hanging).\\

Cara mengatasinya yaitu dengan mensetting ulang indent nya, mengatur panjang indent. sehingga ruas kiri dan kanan nya menjadi rata.



\end{document}