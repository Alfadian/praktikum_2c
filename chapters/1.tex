
\section{MuhammadRezaSyachrani/1174084}
\subsection{Background}
	Python merupakan salah satu bahasa pemrograman tingkat tinggi (high level language) yang dikembangkan oleh Guido van Rossum pada tahun 1989 dan diperkenalkan untuk pertama kalinya pada tahun 1991 di Scitchting Mathematisch Centrum (CWI).
	Python dirancang untuk memberikan kemudahan bagi programmer melalui segi efisiensi waktu, kemudahan dalam pengembangan dan kompatibilitas dengan sistem. Python bisa digunakan untuk membuat aplikasi standalone (berdiri sendiri) dan
	pemrograman script (scripting programming).
	\par
	Python adalah bahasa pemrograman interpretatif multiguna dengan filosofi perancangan yang berfokus pada tingkat keterbacaan kode. Python diklaim sebagai bahasa yang menggabungkan kapabilitas, kemampuan, dengan sintaksis kode yang sangat jelas,
	dan dilengkapi dengan fungsionalitas pustaka standar yang besar serta komprehensif. 
	Sisi utama yang membedakan Python dengan bahasa pemrograman lain adalah dalam hal aturan penulisan kode
	program. Programmer yang menggunakn selain python dapat dibingungkan dengan aturan indentasi, tipe data,
	tuple, dan dictionary. Python memiliki kelebihan tersendiri dibandingkan dengan bahasa lain terutama
	dalam hal penanganan modul, ini yang membuat beberapa programmer menyukai python. Selain itu
	python merupakan salah satu produk yang opensource, free, dan multiplatform.
	
\subsection{Problems}
\begin{itemize}
	\item Bagaimana cara mengimplementasikan bahasa pemrograman python
\end{itemize}
	
\subsection{Objective and Contribution}
\subsubsection{Objective}
\begin{itemize}
	\item Dapat mengimplementasikan bahasa pemrograman python
\end{itemize}
	
\subsubsection{Contribution}
\begin{itemize}
	\item Dapat membangun suatu aplikasi dan alat yang mengimplementasikan bahasa pemrograman python
\end{itemize}

\subsection{Scoop and Environtment}
\begin{itemize}
	\item Menginplementasikan Python dalam pemrograman
\end{itemize}