
\begin{document}

\section{Resume}
\begin{flushleft}
\qquad Python dibuat oleh Guido van Rossum yang dirilis pertama  kali pada tahun 1992, 'python 1.0' dirilis pada januari 1994 dan versi terakhir yang ini dirilis saat ini adalah 'Python 3.7' pada 27 Juni 2018. Nama Python sendiri diambil dari acara televisi Monty Python's Flying Circus. Python mendukung multi paradigma pemrograman. pengembangan Python masih dilakukan oleh Team Guido dan Python Software Foundation sebagai pemegang hak cipta intelektual Python sejak versi 2.1 . perbedaan antara python2 dan python3 contohnya pada statement print, pada python2 : 

print "Hello World"

sedangkan pada python3 :

print ("Hello World")
\end{flushleft}

\section{Instalasi}
\subsection{Install Anacaconda}
\begin{enumerate}
\item Download Anaconda3 https://www.anaconda.com/distribution/#linux
\item 'bash Anaconda3-2018.12-Linux-x86-64.sh
\item review license agreement
\item agree license agreement, text 'yes'
\item lokasi install default di /root/anaconda3
\item tambahkan lokasi di /root/.bashrc, text 'yes'
\item selesai
\end{enumerate}

\subsection{Install Python}
\begin{enumerate}
\item apt-get install python3
\item selesai
\end{enumerate}

\subsection{Install Python pip}
\begin{enumerate}
\item apt-get install python3-pip
\item selesai
\end{enumerate}


\end{document}


\section{MuhammadRezaSyachrani/1174084}
\subsection{Background}
	Python merupakan salah satu bahasa pemrograman tingkat tinggi (high level language) yang dikembangkan oleh Guido van Rossum pada tahun 1989 dan diperkenalkan untuk pertama kalinya pada tahun 1991 di Scitchting Mathematisch Centrum (CWI).
	Python dirancang untuk memberikan kemudahan bagi programmer melalui segi efisiensi waktu, kemudahan dalam pengembangan dan kompatibilitas dengan sistem. Python bisa digunakan untuk membuat aplikasi standalone (berdiri sendiri) dan
	pemrograman script (scripting programming).
	\par
	Python adalah bahasa pemrograman interpretatif multiguna dengan filosofi perancangan yang berfokus pada tingkat keterbacaan kode. Python diklaim sebagai bahasa yang menggabungkan kapabilitas, kemampuan, dengan sintaksis kode yang sangat jelas,
	dan dilengkapi dengan fungsionalitas pustaka standar yang besar serta komprehensif. 
	Sisi utama yang membedakan Python dengan bahasa pemrograman lain adalah dalam hal aturan penulisan kode
	program. Programmer yang menggunakn selain python dapat dibingungkan dengan aturan indentasi, tipe data,
	tuple, dan dictionary. Python memiliki kelebihan tersendiri dibandingkan dengan bahasa lain terutama
	dalam hal penanganan modul, ini yang membuat beberapa programmer menyukai python. Selain itu
	python merupakan salah satu produk yang opensource, free, dan multiplatform.
	
\subsection{Problems}
\begin{itemize}
	\item Bagaimana cara mengimplementasikan bahasa pemrograman python
\end{itemize}
	
\subsection{Objective and Contribution}
\subsubsection{Objective}
\begin{itemize}
	\item Dapat mengimplementasikan bahasa pemrograman python
\end{itemize}
	
\subsubsection{Contribution}
\begin{itemize}
	\item Dapat membangun suatu aplikasi dan alat yang mengimplementasikan bahasa pemrograman python
\end{itemize}


\subsection{Scoop and Environtment}
\begin{itemize}
	\item Menginplementasikan Python dalam pemrograman
\end{itemize}

\section{Instalasi}
\subsection{Cara Pemakaian Script dan interpreter python}
\subsection{Cara Pemakaian spyder termasuk variable explorer}

\section{Mencoba Python}
Untuk memulai suatu pemrograman, kita akan awali dengan membuat sebuah hello world. Di Python, cukup mudah untuk membuat sebuah hello world. Silahkan buat sebuah file dengan nama helloworld.py kemudian buat kode berikut di dalam file tersebut:
\paragraph{}
print "Hello world..."
\paragraph{}
Sekarang mari kita eksekusi file tersebut di konsol dengan perintah berikut:
python helloworld.py
\paragraph{}
Hello world...

\section{Identasi}
Ketika menulis kode program Python perlu memperhatikan indentasi, karena kode program Python distrukturkan berdasarkan indentasi. Kode program yang berada pada sisi kiri yang sama maka dibaca sebagai satu blok, untuk membuat sub blok maka cukup dengan memberikan jarak spasi atau tab ke kanan.
Soal indentasi ini akan lebih jelas ketika pembahasan tentang pencabangan, perulangan, fungsi, class, dan materi yang lain yang membutuhkan penulisan kode program bersarang.
Contohnya adalah sebagai berikut:
import sys
\paragraph{}
if len(sys.argv) < 2:
\paragraph{}
    print("Harap memasukkan argumen.")
    \paragraph{}
    sys.exit(1)

	\section{AlvanAlvanzah/1174077}
\subsection{Background}
Python adalah bahasa pemrograman interpretatif multiguna dengan filosofi perancangan yang berfokus pada tingkat keterbacaan kode. Python diklaim dijadikan bahasa yang menggabungkan kapabilitas, kesanggupan, dengan sintaksis kode yang sangat jelas, dan dilengkapi dengan fungsionalitas pustaka standar yang besar serta komprehensif.
\par
Python adalah bahasa pemrograman yang bersifat open source. Bahasa pemrograman ini dioptimalisasikan untuk software quality, developer productivity, program portability, dan component integration. Python telah digunakan untuk mengembangkan berbagai macam perangkat lunak, seperti internet scripting, systems programming, user interfaces, product customization, numberic programming dll. Python saat ini telah menduduki posisi 4 atau 5 bahasa pemrograman paling sering digunakan di seluruh dunia. Menggunakan alat pihak ketiga, kode Python dapat dikemas ke dalam program yang dapat dieksekusi mandiri. Penerjemah python tersedia untuk banyak sistem operasi.
\subsection{Problems}
\begin{itemize}
\item Bagaimana cara agar memahami bahasa pemrograman python
\end{itemize}
\subsection{Objective and Contribution}
\subsubsection{Objective}
\begin{itemize}
\item Dapat memahami bahasa pemrograman Python
\end{itemize}
\subsubsection{Contribution}
\begin{itemize}
\item Dapat mengimplementasikan bahasa pemrograman python
\end{itemize}

\subsection{Scoop and Environtment}
\begin{itemize}
\item Mempelajari tentang bahasa pemrograman python
\end{itemize}


