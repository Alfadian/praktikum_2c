\section{Resume}
\subsection{Resume Sejarah Python}
\begin{flushleft}
\qquad Bahasa pemrograman Python dirilis pertama kali oleh Guido van Rossum di tahun 1991, yang sudah dikembangkan sejak tahun 1989. Awal pemilihan nama Python tidak secara langsung berasal dari nama ular piton, tapi sebuah acara humor di BBC pada era 1980an dengan judul “Monty Python’s Flying Circus“. Monty Python adalah kelompok lawak yang membawakan acara tersebut. Kebetulan Guido van Rossum adalah penggemar dari acara ini. Pada tahun 1994, Python 1.0 dirilis, yang diikuti dengan Python 2.0 pada tahun 2000. Python 3.0 keluar pada tahun 2008.
\end{flushleft}
\subsection{Perbedaan Python 2 dan Python 3}
\subsubsection{Python 2}
\paragraph{}
Dipublikasikan pada akhir tahun 2000, Python 2 dinilai lebih transparan dan inklusif untuk pengembangan software ketimbang versi sebelumnya. Hal ini didukung dengan adanya PEP – Python Enhancement Proposal, sebuah spesifikasi teknis yang menjadi tuntunan informasi untuk penggunanya dan menggambarkan fitur baru pada Python itu sendiri. Sebagai tambahan, Python 2 dilengkapi dengan berbagai fitur programatikal seperti cycle-detecting garbage collector untuk mengotomasi manajemen memori, peningkatan dukungan untuk Unicode, list comprehension untuk membuat sebuah list berdasarkan list yang sudah ada. Unifikasi pada tipe data Python dan class ke satu hirarki terjadi pada rilis Python 2.2
\subsubsection{Python 3}
\paragraph{}
Python 3 diharapkan sebagai masa depan Python dan merupakan versi yang saat tulisan ini dibuat masih aktif dikembangkan. Python 3 sendiri adalah versi dengan banyak perubahan yang dirilis akhir tahun 2008. Fokus dari Python 3 itu sendiri adalah untuk melakukan perapian pada codebase dan menghapuskan duplikasi (redundancy). Perubahan terbesar pada Python 3 termasuk memasukkan statemen print ke dalam built-in function. Awalnya, Python 3 mengalami hambatan pada pengadopsiannya. Itu akibat dari tidak adanya backwards compatibility dengan Python 2. Hal ini membuat pengguna Python sangat berat hati untuk pindah ke versi 3 ini. Tambahannya, banyak sekali library yang hanya tersedia untuk Python 2., tapi setelah tim pengembangan di balik Python 3 telah berulang kali menjelaskan bahwa dukungan terhadap Python 2 akan segera dihentikan, dan semakin banyak libary disalin ke Python 3, maka penerapan Python 3 semakin lama semakin meningkat.
\subsection{Implementasi dan penggunaan Python pada Perusahaan}
daftar berikut adalah beberapa perusahaan yang menggunakan Python, diantaranya:
\begin{enumerate}
\item
Google adalah perusahaan besar yang menggunakan banyak kode Python di dalam mesin pencarinya. Dan mesin pencari google adalah yang paling terkenal di dunia.
\item
Youtube, situs video terbesar dan terpopuler di dunia, sebagian besar kodenya ditulis dalam bahasa Python.
\item
Facebook, media sosial terbesar di dunia, menggunakan Tornado, sebuah framework Python untuk menampilkan timeline.
\item
Instagram, siapa yang tidak kenal. Instagram menggunakan Django, framework python sebagai mesin pengolah sisi server dari aplikasinya.
\item
Pinterest, banyak menggunakan python untuk membangun aplikasinya.
\item
Dropbox, barangkali Anda adalah salah seorang pengguna layanan ini. Dropbox menggunakan python baik di sisi server maupun di sisi pengguna layanannya.
\item
Quora, salah satu situs tanya jawab terbesar di dunia, dibangun menggunakan Python.
\item
NASA, badan antariksa Amerika ini menggunakan Python untuk bidang sainsnya.
\item
NSA, badan mata – mata Amerika banyak menggunakan Python untuk analisa kriptografi dan intelijen.
\item
Blender, Maya, software pembuat animasi 3D terkenal, menggunakan Python sebagai salah satu bahasa skrip pemrogramannya.
\item
Raspberry Pi, komputer mini yang banyak digunakan sebagai mikrokontroller, menggunakan Python sebagai bahasa utamanya.
\end{enumerate}


\section{Instalasi}
\subsection{Cara Pemakaian Script dan interpreter python}
\subsection{Cara Pemakaian spyder termasuk variable explorer}

\section{Mencoba Python}
Untuk memulai suatu pemrograman, kita akan awali dengan membuat sebuah hello world. Di Python, cukup mudah untuk membuat sebuah hello world. Silahkan buat sebuah file dengan nama helloworld.py kemudian buat kode berikut di dalam file tersebut:
\paragraph{}
print "Hello world..."
\paragraph{}
Sekarang mari kita eksekusi file tersebut di konsol dengan perintah berikut:
python helloworld.py
\paragraph{}
Hello world...

\section{Identasi}
Ketika menulis kode program Python perlu memperhatikan indentasi, karena kode program Python distrukturkan berdasarkan indentasi. Kode program yang berada pada sisi kiri yang sama maka dibaca sebagai satu blok, untuk membuat sub blok maka cukup dengan memberikan jarak spasi atau tab ke kanan.
Soal indentasi ini akan lebih jelas ketika pembahasan tentang pencabangan, perulangan, fungsi, class, dan materi yang lain yang membutuhkan penulisan kode program bersarang.
Contohnya adalah sebagai berikut:
import sys
\paragraph{}
if len(sys.argv) < 2:
\paragraph{}
    print("Harap memasukkan argumen.")
    \paragraph{}
    sys.exit(1)