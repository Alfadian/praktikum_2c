\section{Chandra Kirana Poetra}
\subsection{Teori}
\begin{enumerate}

\item sebutkan jenis-jenis variabel dan jelaskan cara pemakaian variabel tersebut di kode Python
Variable merupakan tempat yang dapat digunakan untuk menyimpan data, dalam phyton kita bisa membuat variable dengan cara berikut
\lstinputlisting[firstline=7, lastline=54]{src/Teori.py}

\item Kode untuk meminta input dari user dan bagaimana melakukan output ke layar seperti pada gambar2
\lstinputlisting[firstline=55, lastline=57]{src/Teori.py}

\item Operator dasar aritmatika
Terdapat penambahan, pengurangan perkalian, perkalian, pembagian, modulus
\lstinputlisting[firstline=60, lastline=68]{src/Teori.py}


\item Perulangan
ada dua jenis perulangan di dalam phyton mereka adalah perulangan while dan perulangan for
\lstinputlisting[firstline=70, lastline=82]{src/teori.py}

\item sintak Untuk memilih kondisi
Kondisi IF digunakan ketika ingin menentukan tindakan apa yang harus digunakan sesuai dengan kondisi yang telah diatur
\lstinputlisting[firstline=83, lastline=109]{src/teori.py}


\item Jenis-jenis error pada phyton
Syntax Errors adalah keadaan dimana kode python mengalami kesalahan penulisan. 
IndentationError adalah eror yang terjadi saat indentasi error.
SystemError adalah eror yang terjadi ketika interpreter mendeteksi error internal
TypeError adalah eror yang terjadi saat dilakukan eksekusi pada suatu operasi atau fungsi dengan type object yang tidak sesuai.
ValueError adalah error ketika value yang dimasukan tidak sesuai
UnicodeTranslateError adalah error yang muncul ketika mentranslate unicode
UnicodeDecodeError adalah error yang muncul ketika  proses decode unicode
UnicodeEncodeError adalah error yang muncul ketika  proses encode unicode
UnicodeError adalah error yang muncul ketika error terkait unicode terdeteksi

\item Cara memakai try except
Cara pemakaian try except adalah sebagai berikut :
    \lstinputlisting[firstline=110, lastline=114]{src/teori.py}


\end{enumerate}

\subsection{Praktek}
\begin{enumerate}
    \item Jawaban soal no 1
    \lstinputlisting[firstline=9, lastline=20]{src/1174079.py}
    \item Jawaban soal no 2
    \lstinputlisting[firstline=21, lastline=28]{src/1174079.py}
    \item Jawaban soal no 3
    \lstinputlisting[firstline=29, lastline=35]{src/1174079.py}
    \item Jawaban soal no 4
    \lstinputlisting[firstline=36, lastline=39]{src/1174079.py}
    \item Jawaban soal no 5
    \lstinputlisting[firstline=41, lastline=53]{src/1174079.py}
    \item Jawaban soal no 6
    \lstinputlisting[firstline=54, lastline=56]{src/1174079.py}
    \item Jawaban soal no 7
    \lstinputlisting[firstline=57, lastline=59]{src/1174079.py}
    \item Jawaban soal no 8
    \lstinputlisting[firstline=60, lastline=68]{src/1174079.py}
    \item Jawaban soal no 9
    \lstinputlisting[firstline=69, lastline=71]{src/1174079.py}
    \item Jawaban soal no 10
    \lstinputlisting[firstline=72, lastline=75]{src/1174079.py}
    \item Jawaban soal no 11
    \lstinputlisting[firstline=76, lastline=77]{src/1174079.py}
\end{enumerate}

\subsection{Keterampilan dan penanganan error}
    \lstinputlisting[firstline=7, lastline=14]{src/error.py}


