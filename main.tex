%%%%%%%%%%%%%%
%% Run LaTeX on this file several times to get Table of Contents,
%% cross-references, and citations.

%% If you have font problems, you may edit the w-bookps.sty file
%% to customize the font names to match those on your system.

%% w-bksamp.tex. Current Version: Feb 16, 2012
%%%%%%%%%%%%%%%%%%%%%%%%%%%%%%%%%%%%%%%%%%%%%%%%%%%%%%%%%%%%%%%%
%
%  Sample file for
%  Wiley Book Style, Design No.: SD 001B, 7x10
%  Wiley Book Style, Design No.: SD 004B, 6x9
%
%
%  Prepared by Amy Hendrickson, TeXnology Inc.
%  http://www.texnology.com
%%%%%%%%%%%%%%%%%%%%%%%%%%%%%%%%%%%%%%%%%%%%%%%%%%%%%%%%%%%%%%%%

%%%%%%%%%%%%%
% 7x10
%\documentclass{wileySev}

% 6x9
\documentclass{wileySix}

\usepackage{graphicx}
\usepackage{listings}

\usepackage{color}
 
\definecolor{codegreen}{rgb}{0,0.6,0}
\definecolor{codegray}{rgb}{0.5,0.5,0.5}
\definecolor{codepurple}{rgb}{0.58,0,0.82}
\definecolor{backcolour}{rgb}{0.95,0.95,0.92}
 
\lstdefinestyle{mystyle}{
    backgroundcolor=\color{backcolour},   
    commentstyle=\color{codegreen},
    keywordstyle=\color{magenta},
    numberstyle=\tiny\color{codegray},
    stringstyle=\color{codepurple},
    basicstyle=\footnotesize,
    breakatwhitespace=false,         
    breaklines=true,                 
    captionpos=b,                    
    keepspaces=true,                 
    numbers=left,                    
    numbersep=5pt,                  
    showspaces=false,                
    showstringspaces=false,
    showtabs=false,                  
    tabsize=2,
    language=sh
}
 
\lstset{style=mystyle}

%%%%%%%
%% for times math: However, this package disables bold math (!)
%% \mathbf{x} will still work, but you will not have bold math
%% in section heads or chapter titles. If you don't use math
%% in those environments, mathptmx might be a good choice.

% \usepackage{mathptmx}

% For PostScript text
\usepackage{w-bookps}

%%%%%%%%%%%%%%%%%%%%%%%%%%%%%%%%%%%%%%%%%%%%%%%%%%%%%%%%%%%%%%%%
%% Other packages you might want to use:

% for chapter bibliography made with BibTeX
% \usepackage{chapterbib}

% for multiple indices
% \usepackage{multind}

% for answers to problems
% \usepackage{answers}

%%%%%%%%%%%%%%%%%%%%%%%%%%%%%%
%% Change options here if you want:
%%
%% How many levels of section head would you like numbered?
%% 0= no section numbers, 1= section, 2= subsection, 3= subsubsection
%%==>>
\setcounter{secnumdepth}{3}

%% How many levels of section head would you like to appear in the
%% Table of Contents?
%% 0= chapter titles, 1= section titles, 2= subsection titles, 
%% 3= subsubsection titles.
%%==>>
\setcounter{tocdepth}{2}

%% Cropmarks? good for final page makeup
%% \docropmarks

%%%%%%%%%%%%%%%%%%%%%%%%%%%%%%
%
% DRAFT
%
% Uncomment to get double spacing between lines, current date and time
% printed at bottom of page.
% \draft
% (If you want to keep tables from becoming double spaced also uncomment
% this):
% \renewcommand{\arraystretch}{0.6}
%%%%%%%%%%%%%%%%%%%%%%%%%%%%%%

%%%%%%% Demo of section head containing sample macro:
%% To get a macro to expand correctly in a section head, with upper and
%% lower case math, put the definition and set the box 
%% before \begin{document}, so that when it appears in the 
%% table of contents it will also work:

\newcommand{\VT}[1]{\ensuremath{{V_{T#1}}}}

%% use a box to expand the macro before we put it into the section head:

\newbox\sectsavebox
\setbox\sectsavebox=\hbox{\boldmath\VT{xyz}}

%%%%%%%%%%%%%%%%% End Demo


\begin{document}


\booktitle{Cerdas Menguasai Python}
\subtitle{Dalam 24 Jam}

\authors{Rolly M. Awangga\\
\affil{Informatics Research Center}
%Floyd J. Fowler, Jr.\\
%\affil{University of New Mexico}
}

\offprintinfo{Cerdas Menguasai Python, First Edition}{Rolly M. Awangga}

%% Can use \\ if title, and edition are too wide, ie,
%% \offprintinfo{Survey Methodology,\\ Second Edition}{Robert M. Groves}

%%%%%%%%%%%%%%%%%%%%%%%%%%%%%%
%% 
\halftitlepage

\titlepage


\begin{copyrightpage}{2019}
%Survey Methodology / Robert M. Groves . . . [et al.].
%\       p. cm.---(Wiley series in survey methodology)
%\    ``Wiley-Interscience."
%\    Includes bibliographical references and index.
%\    ISBN 0-471-48348-6 (pbk.)
%\    1. Surveys---Methodology.  2. Social 
%\  sciences---Research---Statistical methods.  I. Groves, Robert M.  II. %
%Series.\\
%
%HA31.2.S873 2007
%001.4'33---dc22                                             2004044064
\end{copyrightpage}

\dedication{`Jika Kamu tidak dapat menahan lelahnya belajar, 
Maka kamu harus sanggup menahan perihnya Kebodohan.'
~Imam Syafi'i~}

\begin{contributors}
\name{Rolly Maulana Awangga,} Informatics Research Center., Politeknik Pos Indonesia, Bandung,
Indonesia



\end{contributors}

\contentsinbrief
\tableofcontents
\listoffigures
\listoftables
\lstlistoflistings


\begin{foreword}
Sepatah kata dari Kaprodi, Kabag Kemahasiswaan dan Mahasiswa
\end{foreword}

\begin{preface}
Buku ini diciptakan bagi yang awam dengan git sekalipun.

\prefaceauthor{R. M. Awangga}
\where{Bandung, Jawa Barat\\
Februari, 2019}
\end{preface}


\begin{acknowledgments}
Terima kasih atas semua masukan dari para mahasiswa agar bisa membuat buku ini 
lebih baik dan lebih mudah dimengerti.

Terima kasih ini juga ditujukan khusus untuk team IRC yang 
telah fokus untuk belajar dan memahami bagaimana buku ini mendampingi proses 
Intership.
\authorinitials{R. M. A.}
\end{acknowledgments}

\begin{acronyms}
\acro{ACGIH}{American Conference of Governmental Industrial Hygienists}
\acro{AEC}{Atomic Energy Commission}
\acro{OSHA}{Occupational Health and Safety Commission}
\acro{SAMA}{Scientific Apparatus Makers Association}
\end{acronyms}

\begin{glossary}
\term{git}Merupakan manajemen sumber kode yang dibuat oleh linus torvald.

\term{bash}Merupakan bahasa sistem operasi berbasiskan *NIX.

\term{linux}Sistem operasi berbasis sumber kode terbuka yang dibuat oleh Linus Torvald
\end{glossary}

\begin{symbols}
\term{A}Amplitude

\term{\hbox{\&}}Propositional logic symbol 

\term{a}Filter Coefficient

\bigskip

\term{\mathcal{B}}Number of Beats
\end{symbols}

\begin{introduction}

%% optional, but if you want to list author:

\introauthor{Rolly Maulana Awangga, S.T., M.T.}
{Informatics Research Center\\
Bandung, Jawa Barat, Indonesia}

Pada era disruptif  \index{disruptif}\index{disruptif!modern} 
saat ini. git merupakan sebuah kebutuhan dalam sebuah organisasi pengembangan perangkat lunak.
Buku ini diharapkan bisa menjadi penghantar para programmer, analis, IT Operation dan Project Manajer.
Dalam melakukan implementasi git pada diri dan organisasinya.

Rumusnya cuman sebagai contoh aja biar keren\cite{awangga2018sampeu}.

\begin{equation}
ABC {\cal DEF} \alpha\beta\Gamma\Delta\sum^{abc}_{def}
\end{equation}

\end{introduction}

%%%%%%%%%%%%%%%%%%Isi Buku_

\chapter{Judul Bagian Pertama}

\section{Resume}
\subsection{Resume Sejarah Python}
\begin{flushleft}
\qquad Bahasa pemrograman Python dirilis pertama kali oleh Guido van Rossum di tahun 1991, yang sudah dikembangkan sejak tahun 1989. Awal pemilihan nama Python tidak secara langsung berasal dari nama ular piton, tapi sebuah acara humor di BBC pada era 1980an dengan judul “Monty Python’s Flying Circus“. Monty Python adalah kelompok lawak yang membawakan acara tersebut. Kebetulan Guido van Rossum adalah penggemar dari acara ini. Pada tahun 1994, Python 1.0 dirilis, yang diikuti dengan Python 2.0 pada tahun 2000. Python 3.0 keluar pada tahun 2008.
\end{flushleft}
\subsection{Perbedaan Python 2 dan Python 3}
\subsubsection{Python 2}
\paragraph{}
Dipublikasikan pada akhir tahun 2000, Python 2 dinilai lebih transparan dan inklusif untuk pengembangan software ketimbang versi sebelumnya. Hal ini didukung dengan adanya PEP – Python Enhancement Proposal, sebuah spesifikasi teknis yang menjadi tuntunan informasi untuk penggunanya dan menggambarkan fitur baru pada Python itu sendiri. Sebagai tambahan, Python 2 dilengkapi dengan berbagai fitur programatikal seperti cycle-detecting garbage collector untuk mengotomasi manajemen memori, peningkatan dukungan untuk Unicode, list comprehension untuk membuat sebuah list berdasarkan list yang sudah ada. Unifikasi pada tipe data Python dan class ke satu hirarki terjadi pada rilis Python 2.2
\subsubsection{Python 3}
\paragraph{}
Python 3 diharapkan sebagai masa depan Python dan merupakan versi yang saat tulisan ini dibuat masih aktif dikembangkan. Python 3 sendiri adalah versi dengan banyak perubahan yang dirilis akhir tahun 2008. Fokus dari Python 3 itu sendiri adalah untuk melakukan perapian pada codebase dan menghapuskan duplikasi (redundancy). Perubahan terbesar pada Python 3 termasuk memasukkan statemen print ke dalam built-in function. Awalnya, Python 3 mengalami hambatan pada pengadopsiannya. Itu akibat dari tidak adanya backwards compatibility dengan Python 2. Hal ini membuat pengguna Python sangat berat hati untuk pindah ke versi 3 ini. Tambahannya, banyak sekali library yang hanya tersedia untuk Python 2., tapi setelah tim pengembangan di balik Python 3 telah berulang kali menjelaskan bahwa dukungan terhadap Python 2 akan segera dihentikan, dan semakin banyak libary disalin ke Python 3, maka penerapan Python 3 semakin lama semakin meningkat.
\subsection{Implementasi dan penggunaan Python pada Perusahaan}
daftar berikut adalah beberapa perusahaan yang menggunakan Python, diantaranya:
\begin{enumerate}
\item
Google adalah perusahaan besar yang menggunakan banyak kode Python di dalam mesin pencarinya. Dan mesin pencari google adalah yang paling terkenal di dunia.
\item
Youtube, situs video terbesar dan terpopuler di dunia, sebagian besar kodenya ditulis dalam bahasa Python.
\item
Facebook, media sosial terbesar di dunia, menggunakan Tornado, sebuah framework Python untuk menampilkan timeline.
\item
Instagram, siapa yang tidak kenal. Instagram menggunakan Django, framework python sebagai mesin pengolah sisi server dari aplikasinya.
\item
Pinterest, banyak menggunakan python untuk membangun aplikasinya.
\item
Dropbox, barangkali Anda adalah salah seorang pengguna layanan ini. Dropbox menggunakan python baik di sisi server maupun di sisi pengguna layanannya.
\item
Quora, salah satu situs tanya jawab terbesar di dunia, dibangun menggunakan Python.
\item
NASA, badan antariksa Amerika ini menggunakan Python untuk bidang sainsnya.
\item
NSA, badan mata – mata Amerika banyak menggunakan Python untuk analisa kriptografi dan intelijen.
\item
Blender, Maya, software pembuat animasi 3D terkenal, menggunakan Python sebagai salah satu bahasa skrip pemrogramannya.
\item
Raspberry Pi, komputer mini yang banyak digunakan sebagai mikrokontroller, menggunakan Python sebagai bahasa utamanya.
\end{enumerate}


\section{Instalasi}
\subsection{Cara Pemakaian Script dan interpreter python}
\subsection{Cara Pemakaian spyder termasuk variable explorer}

\section{Mencoba Python}
Untuk memulai suatu pemrograman, kita akan awali dengan membuat sebuah hello world. Di Python, cukup mudah untuk membuat sebuah hello world. Silahkan buat sebuah file dengan nama helloworld.py kemudian buat kode berikut di dalam file tersebut:
\paragraph{}
print "Hello world..."
\paragraph{}
Sekarang mari kita eksekusi file tersebut di konsol dengan perintah berikut:
python helloworld.py
\paragraph{}
Hello world...

\section{Identasi}
Ketika menulis kode program Python perlu memperhatikan indentasi, karena kode program Python distrukturkan berdasarkan indentasi. Kode program yang berada pada sisi kiri yang sama maka dibaca sebagai satu blok, untuk membuat sub blok maka cukup dengan memberikan jarak spasi atau tab ke kanan.
Soal indentasi ini akan lebih jelas ketika pembahasan tentang pencabangan, perulangan, fungsi, class, dan materi yang lain yang membutuhkan penulisan kode program bersarang.
Contohnya adalah sebagai berikut:
import sys
\paragraph{}
if len(sys.argv) < 2:
\paragraph{}
    print("Harap memasukkan argumen.")
    \paragraph{}
    sys.exit(1)
	
	\section{AlvanAlvanzah/1174077}
\subsection{Background}
Python adalah bahasa pemrograman interpretatif multiguna dengan filosofi perancangan yang berfokus pada tingkat keterbacaan kode. Python diklaim dijadikan bahasa yang menggabungkan kapabilitas, kesanggupan, dengan sintaksis kode yang sangat jelas, dan dilengkapi dengan fungsionalitas pustaka standar yang besar serta komprehensif.
\par
Python adalah bahasa pemrograman yang bersifat open source. Bahasa pemrograman ini dioptimalisasikan untuk software quality, developer productivity, program portability, dan component integration. Python telah digunakan untuk mengembangkan berbagai macam perangkat lunak, seperti internet scripting, systems programming, user interfaces, product customization, numberic programming dll. Python saat ini telah menduduki posisi 4 atau 5 bahasa pemrograman paling sering digunakan di seluruh dunia. Menggunakan alat pihak ketiga, kode Python dapat dikemas ke dalam program yang dapat dieksekusi mandiri. Penerjemah python tersedia untuk banyak sistem operasi.
\subsection{Problems}
\begin{itemize}
\item Bagaimana cara agar memahami bahasa pemrograman python
\end{itemize}
\subsection{Objective and Contribution}
\subsubsection{Objective}
\begin{itemize}
\item Dapat memahami bahasa pemrograman Python
\end{itemize}
\subsubsection{Contribution}
\begin{itemize}
\item Dapat mengimplementasikan bahasa pemrograman python
\end{itemize}

\subsection{Scoop and Environtment}
\begin{itemize}
\item Mempelajari tentang bahasa pemrograman python
\end{itemize}


\chapter{Chapter 2}
\section{Chapter 2 | D. Irga B. Naufal Fakhri D4 TI 2C}
\subsection{Teori Praktikum}
\begin{enumerate}
\item Jenis-jenis variabel pada python dan cara penggunaannya:

\begin{enumerate}
\item Boolean
\lstinputlisting[caption=Contoh kode variable Boolean., firstline=8, lastline=10]{src/1174066.py}

\item String
\lstinputlisting[caption=Contoh kode variable String., firstline=12, lastline=14]{src/1174066.py}

\item Integer
\lstinputlisting[caption=Contoh kode variable Integer., firstline=16, lastline=18]{src/1174066.py}

\item Float
\lstinputlisting[caption=Contoh kode variable Float., firstline=20, lastline=22]{src/1174066.py}

\item Hexadecimal
\lstinputlisting[caption=Contoh kode variable Hexadecimal., firstline=24, lastline=26]{src/1174066.py}

\item Complex
\lstinputlisting[caption=Contoh kode variable Complex., firstline=28, lastline=30]{src/1174066.py}

\item List
\lstinputlisting[caption=Contoh kode variable List., firstline=32, lastline=35]{src/1174066.py}

\item Tuple
\lstinputlisting[caption=Contoh kode variable Tuple., firstline=37, lastline=40]{src/1174066.py}

\item Set
\lstinputlisting[caption=Contoh kode variable Set., firstline=42, lastline=44]{src/1174066.py}

\item Dictionary
\lstinputlisting[caption=Contoh kode variable Dictionary., firstline=46, lastline=49]{src/1174066.py}

\end{enumerate}

\item Permintaan Input dari user dan Outputnya
\lstinputlisting[caption=Contoh kode input dan outputnya., firstline=51, lastline=53]{src/1174066.py}

\item Operator dasar aritmatika dan perubahan tipe data variable

Operator dasar aritmatika
\begin{enumerate}
\item Perjumlahan (+)
Operator ini berfungsi untuk melakukan operasi perjumlahan.
\lstinputlisting[caption=Contoh kode operasi pertambahan., firstline=51, lastline=60]{src/1174066.py}
\item Pengurangan (-)
Operator ini berfungsi untuk melakukan operasi pengurangan.
\lstinputlisting[caption=Contoh kode operasi pengurangan., firstline=62, lastline=66]{src/1174066.py}
\item Perkalian (*)
Operator ini dipergunakan untuk melakukan operasi perkalian.
\lstinputlisting[caption=Contoh kode operasi perkalian., firstline=68, lastline=72]{src/1174066.py}
\item Pembagian (/)
Operator ini dipergunakan untuk melakukan operasi pembagian.
\lstinputlisting[caption=Contoh kode operasi pembagian., firstline=74, lastline=78]{src/1174066.py}
\item Modulus (%)
Operator ini dipergunakan untuk melakukan operasi modulus.
\lstinputlisting[caption=Contoh kode operasi modulus., firstline=80, lastline=84]{src/1174066.py}
\item Perpangkatan (**)
Operator ini dipergunakan untuk melakukan operasi perpangkatan.
\lstinputlisting[caption=Contoh kode operasi perpangkatan., firstline=86, lastline=90]{src/1174066.py}
\item Pembulatan Kebawah Pada Hasil Pembagian (//)
Operator ini dipergunakan untuk melakukan operasi pembulatan hasil bagi.
\lstinputlisting[caption=Contoh kode operasi pembulatan hasil pembagian kebawah., firstline=92, lastline=96]{src/1174066.py}
\end{enumerate}

Perubahan tipe data variable
\begin{enumerate}
\item String menjadi Integer
\lstinputlisting[caption=Contoh kode variable string menjadi integer., firstline=99, lastline=102]{src/1174066.py}
\item Integer menjadi String
\lstinputlisting[caption=Contoh kode variable integer menjadi string., firstline=104, lastline=107]{src/1174066.py}
\end{enumerate}


\item Sintak perulangan (looping), jenis-jenisnya, dan penggunaannya.
\begin{enumerate}
\item While Loop
While Loop adalah perulangan yang mengeksekusi statement terus menerus selama kondisi bernilai True.
\lstinputlisting[caption=Contoh kode penggunaan while loop., firstline=111, lastline=115]{src/1174066.py}

\item For Loop
For Loop  adalah pengulangan berdasarkan kondisi yang telah ditentukan biasanya kondisi pertambahan seperti 1 sampai 5
\lstinputlisting[caption=Contoh kode penggunaan for loop., firstline=117, lastline=120]{src/1174066.py}

\item Nested Loop
Nested Loop merupakan pengulangan yang ada di dalam pengulangan
\lstinputlisting[caption=Contoh kode penggunaan nested loop., firstline=122, lastline=129]{src/1174066.py}

\end{enumerate}

\item Sintak kondisi dan penggunaannya.
\begin{enumerate}
\item If
Kondisi ini digunakan untuk mengecek apabila kondisi tersebut dipenuhi akan mengeksekusi kode didalamnya.
\lstinputlisting[caption=Contoh kode penggunaan if., firstline=132, lastline=136]{src/1174066.py}

\item If Else
Kondisi ini digunakan untuk mengecek apabila kondisi tersebut dipenuhi akan mengeksekusi kode didalamnya dan didalamnya memiliki dua kondisi.
\lstinputlisting[caption=Contoh kode penggunaan if else., firstline=138, lastline=143]{src/1174066.py}

\item Elif
Kondisi ini digunakan untuk mengecek apabila kondisi tersebut dipenuhi akan mengeksekusi kode didalamnya dan didalamnya memiliki dua kondisi atau lebih.
\lstinputlisting[caption=Contoh kode penggunaan elif., firstline=145, lastline=152]{src/1174066.py}

\item Kondisi di dalam kondisi
Kondisi ini digunakan saat kondisi memerlukan kondisi lagi didalamnya
\lstinputlisting[caption=Contoh kode penggunaan kondisi di dalam kondisi., firstline=155, lastline=165]{src/1174066.py}

\end{enumerate}

\item Jenis-jenis error pada python dan cara mengatasinya.
\begin{itemize}
\item Syntax Errors
Syntax Errors adalah kesalahan pada penulisan syntax atau kode. Solusinya adalah memperbaiki penulisan syntax atau kode

\item Zero Division Error
ZeroDivisonError adalah exceptions yang terjadi saat eksekusi program menghasilkan perhitungan matematika pembagian dengan angka nol (0). Solusinya adalah tidak membagi suatu yang hasilnya nol.

\item Name Error
NameError adalah exception saat kode melakukan eksekusi terhadap local name atau global name yang tidak terdefinisi atau tidak ada. Solusinya adalah memastikan variabel atau function yang akan dipanggil ada didalam program atau tidak salah mengetikannya.

\item Type Error
TypeError adalah exception saat melakukan eksekusi terhadap suatu operasi atau fungsi dengan type object yang tidak sesuai. Solusinya adalah mengkoversi varibelnya sesuai dengan tipe data sesuai dengan yang akan digunakan.

\end{itemize}

\item Cara pemakaian Try Except.
\lstinputlisting[caption=Contoh kode penggunaan try except., firstline=168, lastline=174]{src/1174066.py}

\end{enumerate}
\hfill \break

\subsection{Ketrampilan Pemrograman}

\begin{enumerate}
\item Jawaban Soal 1
\lstinputlisting[firstline=117, lastline=200]{src/1174066.py}

\item Jawaban Soal 2
\lstinputlisting[firstline=203, lastline=209]{src/1174066.py}

\item Jawaban Soal 3
\lstinputlisting[firstline=212, lastline=219]{src/1174066.py}

\item Jawaban Soal 4
\lstinputlisting[firstline=222, lastline=225]{src/1174066.py}

\item Jawaban Soal 5
\lstinputlisting[firstline=228, lastline=242]{src/1174066.py}

\item Jawaban Soal 6
\lstinputlisting[firstline=245, lastline=247]{src/1174066.py}

\item Jawaban Soal 7
\lstinputlisting[firstline=250, lastline=252]{src/1174066.py}

\item Jawaban Soal 8
\lstinputlisting[firstline=255, lastline=258]{src/1174066.py}

\item Jawaban Soal 9
\lstinputlisting[firstline=261, lastline=268]{src/1174066.py}

\item Jawaban Soal 10
\lstinputlisting[firstline=271, lastline=277]{src/1174066.py}

\item Jawaban Soal 11
\lstinputlisting[firstline=280, lastline=288]{src/1174066.py}

\end{enumerate}
\hfill \break

\subsection{Ketrampilan Penanganan Error}
\begin{enumerate}
\item Jawaban Soal No. 1
\begin{itemize}
\item Syntax Errors
Syntax Errors adalah kesalahan pada penulisan syntax atau kode. Solusinya adalah memperbaiki penulisan syntax atau kode

\item Zero Division Error
ZeroDivisonError adalah exceptions yang terjadi saat eksekusi program menghasilkan perhitungan matematika pembagian dengan angka nol (0). Solusinya adalah tidak membagi suatu yang hasilnya nol.

\item Name Error
NameError adalah exception saat kode melakukan eksekusi terhadap local name atau global name yang tidak terdefinisi atau tidak ada. Solusinya adalah memastikan variabel atau function yang akan dipanggil ada didalam program atau tidak salah mengetikannya.

\item Type Error
TypeError adalah exception saat melakukan eksekusi terhadap suatu operasi atau fungsi dengan type object yang tidak sesuai. Solusinya adalah mengkoversi varibelnya sesuai dengan tipe data sesuai dengan yang akan digunakan.

\end{itemize}

\item Jawaban Soal No. 2																			
\lstinputlisting[firstline=1, lastline=7]{src/2err_1174066.py}
\end{enumerate}



\bibliographystyle{IEEEtran} 
%\def\bibfont{\normalsize}
\bibliography{references}


%%%%%%%%%%%%%%%
%%  The default LaTeX Index
%%  Don't need to add any commands before \begin{document}
\printindex

%%%% Making an index
%% 
%% 1. Make index entries, don't leave any spaces so that they
%% will be sorted correctly.
%% 
%% \index{term}
%% \index{term!subterm}
%% \index{term!subterm!subsubterm}
%% 
%% 2. Run LaTeX several times to produce <filename>.idx
%% 
%% 3. On command line, type  makeindx <filename> which
%% will produce <filename>.ind 
%% 
%% 4. Type \printindex to make the index appear in your book.
%% 
%% 5. If you would like to edit <filename>.ind 
%% you may do so. See docs.pdf for more information.
%% 
%%%%%%%%%%%%%%%%%%%%%%%%%%%%%%

%%%%%%%%%%%%%% Making Multiple Indices %%%%%%%%%%%%%%%%
%% 1. 
%% \usepackage{multind}
%% \makeindex{book}
%% \makeindex{authors}
%% \begin{document}
%% 
%% 2.
%% % add index terms to your book, ie,
%% \index{book}{A term to go to the topic index}
%% \index{authors}{Put this author in the author index}
%% 
%% \index{book}{Cows}
%% \index{book}{Cows!Jersey}
%% \index{book}{Cows!Jersey!Brown}
%% 
%% \index{author}{Douglas Adams}
%% \index{author}{Boethius}
%% \index{author}{Mark Twain}
%% 
%% 3. On command line type 
%% makeindex topic 
%% makeindex authors
%% 
%% 4.
%% this is a Wiley command to make the indices print:
%% \multiprintindex{book}{Topic index}
%% \multiprintindex{authors}{Author index}

\end{document}

