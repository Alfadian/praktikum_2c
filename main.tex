%%%%%%%%%%%%%%
%% Run LaTeX on this file several times to get Table of Contents,
%% cross-references, and citations.

%% If you have font problems, you may edit the w-bookps.sty file
%% to customize the font names to match those on your system.

%% w-bksamp.tex. Current Version: Feb 16, 2012
%%%%%%%%%%%%%%%%%%%%%%%%%%%%%%%%%%%%%%%%%%%%%%%%%%%%%%%%%%%%%%%%
%
%  Sample file for
%  Wiley Book Style, Design No.: SD 001B, 7x10
%  Wiley Book Style, Design No.: SD 004B, 6x9
%
%
%  Prepared by Amy Hendrickson, TeXnology Inc.
%  http://www.texnology.com
%%%%%%%%%%%%%%%%%%%%%%%%%%%%%%%%%%%%%%%%%%%%%%%%%%%%%%%%%%%%%%%%

%%%%%%%%%%%%%
% 7x10
%\documentclass{wileySev}

% 6x9
\documentclass{wileySix}

\usepackage{graphicx}
\usepackage{listings}

\usepackage{color}
 
\definecolor{codegreen}{rgb}{0,0.6,0}
\definecolor{codegray}{rgb}{0.5,0.5,0.5}
\definecolor{codepurple}{rgb}{0.58,0,0.82}
\definecolor{backcolour}{rgb}{0.95,0.95,0.92}
 
\lstdefinestyle{mystyle}{
    backgroundcolor=\color{backcolour},   
    commentstyle=\color{codegreen},
    keywordstyle=\color{magenta},
    numberstyle=\tiny\color{codegray},
    stringstyle=\color{codepurple},
    basicstyle=\footnotesize,
    breakatwhitespace=false,         
    breaklines=true,                 
    captionpos=b,                    
    keepspaces=true,                 
    numbers=left,                    
    numbersep=5pt,                  
    showspaces=false,                
    showstringspaces=false,
    showtabs=false,                  
    tabsize=2,
    language=sh
}
 
\lstset{style=mystyle}

%%%%%%%
%% for times math: However, this package disables bold math (!)
%% \mathbf{x} will still work, but you will not have bold math
%% in section heads or chapter titles. If you don't use math
%% in those environments, mathptmx might be a good choice.

% \usepackage{mathptmx}

% For PostScript text
\usepackage{w-bookps}

%%%%%%%%%%%%%%%%%%%%%%%%%%%%%%%%%%%%%%%%%%%%%%%%%%%%%%%%%%%%%%%%
%% Other packages you might want to use:

% for chapter bibliography made with BibTeX
% \usepackage{chapterbib}

% for multiple indices
% \usepackage{multind}

% for answers to problems
% \usepackage{answers}

%%%%%%%%%%%%%%%%%%%%%%%%%%%%%%
%% Change options here if you want:
%%
%% How many levels of section head would you like numbered?
%% 0= no section numbers, 1= section, 2= subsection, 3= subsubsection
%%==>>
\setcounter{secnumdepth}{3}

%% How many levels of section head would you like to appear in the
%% Table of Contents?
%% 0= chapter titles, 1= section titles, 2= subsection titles, 
%% 3= subsubsection titles.
%%==>>
\setcounter{tocdepth}{2}

%% Cropmarks? good for final page makeup
%% \docropmarks

%%%%%%%%%%%%%%%%%%%%%%%%%%%%%%
%
% DRAFT
%
% Uncomment to get double spacing between lines, current date and time
% printed at bottom of page.
% \draft
% (If you want to keep tables from becoming double spaced also uncomment
% this):
% \renewcommand{\arraystretch}{0.6}
%%%%%%%%%%%%%%%%%%%%%%%%%%%%%%

%%%%%%% Demo of section head containing sample macro:
%% To get a macro to expand correctly in a section head, with upper and
%% lower case math, put the definition and set the box 
%% before \begin{document}, so that when it appears in the 
%% table of contents it will also work:

\newcommand{\VT}[1]{\ensuremath{{V_{T#1}}}}

%% use a box to expand the macro before we put it into the section head:

\newbox\sectsavebox
\setbox\sectsavebox=\hbox{\boldmath\VT{xyz}}

%%%%%%%%%%%%%%%%% End Demo


\begin{document}


\booktitle{Cerdas Menguasai Python}
\subtitle{Dalam 24 Jam}

\authors{Rolly M. Awangga\\
\affil{Informatics Research Center}
%Floyd J. Fowler, Jr.\\
%\affil{University of New Mexico}
}

\offprintinfo{Cerdas Menguasai Python, First Edition}{Rolly M. Awangga}

%% Can use \\ if title, and edition are too wide, ie,
%% \offprintinfo{Survey Methodology,\\ Second Edition}{Robert M. Groves}

%%%%%%%%%%%%%%%%%%%%%%%%%%%%%%
%% 
\halftitlepage

\titlepage


\begin{copyrightpage}{2019}
%Survey Methodology / Robert M. Groves . . . [et al.].
%\       p. cm.---(Wiley series in survey methodology)
%\    ``Wiley-Interscience."
%\    Includes bibliographical references and index.
%\    ISBN 0-471-48348-6 (pbk.)
%\    1. Surveys---Methodology.  2. Social 
%\  sciences---Research---Statistical methods.  I. Groves, Robert M.  II. %
%Series.\\
%
%HA31.2.S873 2007
%001.4'33---dc22                                             2004044064
\end{copyrightpage}

\dedication{`Jika Kamu tidak dapat menahan lelahnya belajar, 
Maka kamu harus sanggup menahan perihnya Kebodohan.'
~Imam Syafi'i~}

\begin{contributors}
\name{Rolly Maulana Awangga,} Informatics Research Center., Politeknik Pos Indonesia, Bandung,
Indonesia



\end{contributors}

\contentsinbrief
\tableofcontents
\listoffigures
\listoftables
\lstlistoflistings


\begin{foreword}
Sepatah kata dari Kaprodi, Kabag Kemahasiswaan dan Mahasiswa
\end{foreword}

\begin{preface}
Buku ini diciptakan bagi yang awam dengan git sekalipun.

\prefaceauthor{R. M. Awangga}
\where{Bandung, Jawa Barat\\
Februari, 2019}
\end{preface}


\begin{acknowledgments}
Terima kasih atas semua masukan dari para mahasiswa agar bisa membuat buku ini 
lebih baik dan lebih mudah dimengerti.

Terima kasih ini juga ditujukan khusus untuk team IRC yang 
telah fokus untuk belajar dan memahami bagaimana buku ini mendampingi proses 
Intership.
\authorinitials{R. M. A.}
\end{acknowledgments}

\begin{acronyms}
\acro{ACGIH}{American Conference of Governmental Industrial Hygienists}
\acro{AEC}{Atomic Energy Commission}
\acro{OSHA}{Occupational Health and Safety Commission}
\acro{SAMA}{Scientific Apparatus Makers Association}
\end{acronyms}

\begin{glossary}
\term{git}Merupakan manajemen sumber kode yang dibuat oleh linus torvald.

\term{bash}Merupakan bahasa sistem operasi berbasiskan *NIX.

\term{linux}Sistem operasi berbasis sumber kode terbuka yang dibuat oleh Linus Torvald
\end{glossary}

\begin{symbols}
\term{A}Amplitude

\term{\hbox{\&}}Propositional logic symbol 

\term{a}Filter Coefficient

\bigskip

\term{\mathcal{B}}Number of Beats
\end{symbols}

\begin{introduction}

%% optional, but if you want to list author:

\introauthor{Rolly Maulana Awangga, S.T., M.T.}
{Informatics Research Center\\
Bandung, Jawa Barat, Indonesia}

Pada era disruptif  \index{disruptif}\index{disruptif!modern} 
saat ini. git merupakan sebuah kebutuhan dalam sebuah organisasi pengembangan perangkat lunak.
Buku ini diharapkan bisa menjadi penghantar para programmer, analis, IT Operation dan Project Manajer.
Dalam melakukan implementasi git pada diri dan organisasinya.

Rumusnya cuman sebagai contoh aja biar keren\cite{awangga2018sampeu}.

\begin{equation}
ABC {\cal DEF} \alpha\beta\Gamma\Delta\sum^{abc}_{def}
\end{equation}

\end{introduction}

%%%%%%%%%%%%%%%%%%Isi Buku_

\chapter{SEJARAH DAN KARAKTERISTIK  PYTHON}

\section{Resume}
\subsection{Resume Sejarah Python}
\begin{flushleft}
\qquad Bahasa pemrograman Python dirilis pertama kali oleh Guido van Rossum di tahun 1991, yang sudah dikembangkan sejak tahun 1989. Awal pemilihan nama Python tidak secara langsung berasal dari nama ular piton, tapi sebuah acara humor di BBC pada era 1980an dengan judul “Monty Python’s Flying Circus“. Monty Python adalah kelompok lawak yang membawakan acara tersebut. Kebetulan Guido van Rossum adalah penggemar dari acara ini. Pada tahun 1994, Python 1.0 dirilis, yang diikuti dengan Python 2.0 pada tahun 2000. Python 3.0 keluar pada tahun 2008.
\end{flushleft}
\subsection{Perbedaan Python 2 dan Python 3}
\subsubsection{Python 2}
\paragraph{}
Dipublikasikan pada akhir tahun 2000, Python 2 dinilai lebih transparan dan inklusif untuk pengembangan software ketimbang versi sebelumnya. Hal ini didukung dengan adanya PEP – Python Enhancement Proposal, sebuah spesifikasi teknis yang menjadi tuntunan informasi untuk penggunanya dan menggambarkan fitur baru pada Python itu sendiri. Sebagai tambahan, Python 2 dilengkapi dengan berbagai fitur programatikal seperti cycle-detecting garbage collector untuk mengotomasi manajemen memori, peningkatan dukungan untuk Unicode, list comprehension untuk membuat sebuah list berdasarkan list yang sudah ada. Unifikasi pada tipe data Python dan class ke satu hirarki terjadi pada rilis Python 2.2
\subsubsection{Python 3}
\paragraph{}
Python 3 diharapkan sebagai masa depan Python dan merupakan versi yang saat tulisan ini dibuat masih aktif dikembangkan. Python 3 sendiri adalah versi dengan banyak perubahan yang dirilis akhir tahun 2008. Fokus dari Python 3 itu sendiri adalah untuk melakukan perapian pada codebase dan menghapuskan duplikasi (redundancy). Perubahan terbesar pada Python 3 termasuk memasukkan statemen print ke dalam built-in function. Awalnya, Python 3 mengalami hambatan pada pengadopsiannya. Itu akibat dari tidak adanya backwards compatibility dengan Python 2. Hal ini membuat pengguna Python sangat berat hati untuk pindah ke versi 3 ini. Tambahannya, banyak sekali library yang hanya tersedia untuk Python 2., tapi setelah tim pengembangan di balik Python 3 telah berulang kali menjelaskan bahwa dukungan terhadap Python 2 akan segera dihentikan, dan semakin banyak libary disalin ke Python 3, maka penerapan Python 3 semakin lama semakin meningkat.
\subsection{Implementasi dan penggunaan Python pada Perusahaan}
daftar berikut adalah beberapa perusahaan yang menggunakan Python, diantaranya:
\begin{enumerate}
\item
Google adalah perusahaan besar yang menggunakan banyak kode Python di dalam mesin pencarinya. Dan mesin pencari google adalah yang paling terkenal di dunia.
\item
Youtube, situs video terbesar dan terpopuler di dunia, sebagian besar kodenya ditulis dalam bahasa Python.
\item
Facebook, media sosial terbesar di dunia, menggunakan Tornado, sebuah framework Python untuk menampilkan timeline.
\item
Instagram, siapa yang tidak kenal. Instagram menggunakan Django, framework python sebagai mesin pengolah sisi server dari aplikasinya.
\item
Pinterest, banyak menggunakan python untuk membangun aplikasinya.
\item
Dropbox, barangkali Anda adalah salah seorang pengguna layanan ini. Dropbox menggunakan python baik di sisi server maupun di sisi pengguna layanannya.
\item
Quora, salah satu situs tanya jawab terbesar di dunia, dibangun menggunakan Python.
\item
NASA, badan antariksa Amerika ini menggunakan Python untuk bidang sainsnya.
\item
NSA, badan mata – mata Amerika banyak menggunakan Python untuk analisa kriptografi dan intelijen.
\item
Blender, Maya, software pembuat animasi 3D terkenal, menggunakan Python sebagai salah satu bahasa skrip pemrogramannya.
\item
Raspberry Pi, komputer mini yang banyak digunakan sebagai mikrokontroller, menggunakan Python sebagai bahasa utamanya.
\end{enumerate}


\section{Instalasi}
\subsection{Cara Pemakaian Script dan interpreter python}
\subsection{Cara Pemakaian spyder termasuk variable explorer}

\section{Mencoba Python}
Untuk memulai suatu pemrograman, kita akan awali dengan membuat sebuah hello world. Di Python, cukup mudah untuk membuat sebuah hello world. Silahkan buat sebuah file dengan nama helloworld.py kemudian buat kode berikut di dalam file tersebut:
\paragraph{}
print "Hello world..."
\paragraph{}
Sekarang mari kita eksekusi file tersebut di konsol dengan perintah berikut:
python helloworld.py
\paragraph{}
Hello world...

\section{Identasi}
Ketika menulis kode program Python perlu memperhatikan indentasi, karena kode program Python distrukturkan berdasarkan indentasi. Kode program yang berada pada sisi kiri yang sama maka dibaca sebagai satu blok, untuk membuat sub blok maka cukup dengan memberikan jarak spasi atau tab ke kanan.
Soal indentasi ini akan lebih jelas ketika pembahasan tentang pencabangan, perulangan, fungsi, class, dan materi yang lain yang membutuhkan penulisan kode program bersarang.
Contohnya adalah sebagai berikut:
import sys
\paragraph{}
if len(sys.argv) < 2:
\paragraph{}
    print("Harap memasukkan argumen.")
    \paragraph{}
    sys.exit(1)
	
	\section{AlvanAlvanzah/1174077}
\subsection{Background}
Python adalah bahasa pemrograman interpretatif multiguna dengan filosofi perancangan yang berfokus pada tingkat keterbacaan kode. Python diklaim dijadikan bahasa yang menggabungkan kapabilitas, kesanggupan, dengan sintaksis kode yang sangat jelas, dan dilengkapi dengan fungsionalitas pustaka standar yang besar serta komprehensif.
\par
Python adalah bahasa pemrograman yang bersifat open source. Bahasa pemrograman ini dioptimalisasikan untuk software quality, developer productivity, program portability, dan component integration. Python telah digunakan untuk mengembangkan berbagai macam perangkat lunak, seperti internet scripting, systems programming, user interfaces, product customization, numberic programming dll. Python saat ini telah menduduki posisi 4 atau 5 bahasa pemrograman paling sering digunakan di seluruh dunia. Menggunakan alat pihak ketiga, kode Python dapat dikemas ke dalam program yang dapat dieksekusi mandiri. Penerjemah python tersedia untuk banyak sistem operasi.
\subsection{Problems}
\begin{itemize}
\item Bagaimana cara agar memahami bahasa pemrograman python
\end{itemize}
\subsection{Objective and Contribution}
\subsubsection{Objective}
\begin{itemize}
\item Dapat memahami bahasa pemrograman Python
\end{itemize}
\subsubsection{Contribution}
\begin{itemize}
\item Dapat mengimplementasikan bahasa pemrograman python
\end{itemize}

\subsection{Scoop and Environtment}
\begin{itemize}
\item Mempelajari tentang bahasa pemrograman python
\end{itemize}


\chapter{Judul Bagian Kedua}
\section{Perintah Navigasi}
Perintah navigasi direktori


\chapter{Judul Bagian Ketiga}
\section {Sekar }

\documentclass[10pt]{article}

\title{Fungsi}

\begin{document}
Pada contoh dibawah , sebuah fungsi dengan nama perkalian(), memiliki dua buah argumen yaitu a dan b. Isi dari fungsi tersebut adalah melakukan perhitungan perkalian yang diambil dari nilai a dan b, yang di simpan ke dalam variabel c. Nilai dari c lah yang akan dikembalikan oleh fungsi dari hasil pemanggilan fungsi melalui statemen perkalian(5,10)\\
\include 
\begin{equation}
Contoh:\\
def perkalian(a,b):
	c = a*b
return c
	#Program Utama
print( perkalian(5,10))

>>> def nama():
	gelar = 'Mr'
	aksi = (lambda x: gelar + ' ' + x)
	return aksi

>>> act = nama()
>>> act('Namjoon')
'Sir Namjoon'
\end{equation}

\begin{Scope Variabel}
cakupan variabel merupakan suatu keadaan dimana pendeklarasian sebuah variabel di tentukan , Dalam scope variabel dikenal dua istilah yaitu local dan global.
Contoh penggunaan scope variabel() :
x = 12
y = 3
	print "Sebelum memanggil fungsi, x bernilai", x
	print "Sebelum memanggil fungsi, y bernilai", y
swap(x,y)
	print "Setelah memanggil fungsi, x bernilai", x
	print "Setelah memanggil fungsi, y bernilai", y
	
\begin{Fungsi Rekursif}
untuk menyederhanakan penulisan program dan menggantikan bentuk iterasi. Dengan rekursi, program akan lebih mudah dilihat.
# Fungsi Rekursif faktorial
	def faktorial(nilai):
		if nilai <= 1:
	return 1
		else:
	return nilai * faktorial(nilai - 1)
#Program utama
	for i in range(11):
	print "%2d ! = %d" % (i, faktorial(i))
	
\begin{Melewatkan Argumen dengan Kata Kunci}
Jika fungsi perkalian kita panggil dengan memberi pernyataan perkalian(10,8), maka nilai 10 akan disalin ke variabel x dan nilai 8 ke variabel y.\\
def perkalian(a, b):
	"Mengalikan dua bilangan"
	z = x * y
		print "Nilai a =",a
		print "Nilai b =",b
		print "a* b =",c
# program utama mulai di sini
	perkalian(5,3)
		print perkalian(b=4,a=2)
Hasilnya:
Nilai a = 5
Nilai b = 3
a*b = 15
Nilai a = 2

Jadi nilai default hanya boleh diberikan kepada deretan akhir parameter. Setelah pemberian nilai default, semua parameter di belakangnya juga harus diberi nilai default. Satu catatan, nilai awal argumen akan dievaluasi pada saat dideklarasikan. Perhatikan contoh berikut :
usernm="admin"
passwd="aa"
def login(username=usernm, password=passwd):
	print "Your username ",username
	print "Your password ",password
	print 
	usernm="tamu" 
	passwd="cc"
login()
Untuk memanggil fungsi dengan deklarasi seperti ini, kita harus menyebutkan daftar argumen beserta kata-kuncinya. 
Contoh ():
	def cetak1():
print ‘Hello World’
	def cetak2(n):
print n
	cetak1()
hallo world
	cetak2(123)
123
	cetak2('apa kabar?')
apa kabar
	def cetak3(x,y,z):
print x,y,z
	def cetak4(x,y,z=4):
print x,y,z
	cetak3(1,2,3)
1 2 3
	cetak4(1,2)
1 2 4
	cetak4(1,2,3)
1 2 3

\class Ngitung:
  def __init

\end{document}

\begin{Kelas}
Class adalah salah satu cara bagaimana kita membuat sebuah kode yang mempunyai behaviour tertentu dan lebih mudah dalam mengorganisasi berbagai fungsi dan state-nya. Dalam sebuah class kamu dapat menyimpan sebuah state tanpa harus membuat banyak state bila tidak menggunakan class.\\
Contoh :\\
class Product:
    __vendor_message = "Ini adalah rahasia"
    name = ""
    price = ""
    size = ""
    unit = ""
    
    def __init__(self, name):
        print "Ini adalah constructor"
        self.name = name
        self.unit = "ml"
        self.size = 350
        
    def get_vendor_message(self):
        print self.__vendor_message
        
	def set_price(self, price):
        self.price = price
        
p = Product("Banana Milk")
p.set_price(5500)

print "%s dengan ukuran %s %s harganya Rp. %d" % (p.name, p.size, p.unit, p.price)
# print p.__vendor_message

p.get_vendor_message()

p1 = Product("UltraMilk")
p1.set_price(3000)

print "%s dengan ukuran %s %s harganya Rp. %d" % (p.name, p.size, p.unit, p.price)

print p == p
print p1 == p1
print p == p1

\begin{Pemahanan Teori}
1.void(fungsi tanpa nilai balik)
	Fungsi yang void sering disebut juga prosedur. Disebut void karena fungsi tersebut tidak mengembalikan suatu nilai keluaran yang didapat dari hasil proses tersebut.
	Ciri-ciri dari jenis fungsi Void , yaitu :
	1. tidak adanya keyword return
	2. tidak adanya tipe data di dalam deklarasi fungsi
	3. menggunakan keyword void
	4. tidak memiliki nilai kembalian fungsi
	5. Keyword void juga digunakan jika suatu function tidak 				   mengandung suatu parameter apapun.
Contohnya:\\
	void menampilkan_jumlah(int a, int b)}
		in jumlah;
		jumlah = a + b;
		cout << jumlah;
	}
2.Non void (fungsi dengan nilai balik
	Fungsi non-void disebut juga function . disebut non-void karena mengembalikan nilai kembalian yang berasal dari keluaran hasil proses function tersebut.
	Ciri-ciri dari jenis fungsi Non-Void , yaitu :
	1. Ada keyword return
	2. ada tipe data yang mengawali fungsi
	3. tidak ada keyword void
	4. memiliki nilai keyword
	5. Non-void : int jumlah (int a, int b)

3. Prototype Function
	Sebuah program C++ dapat terdiri dari banyak fungsi. Salah satu fungsi tersebut harus bernama main(). Jika fungsi yang lain dituliskan setelah fungsi main(), sebelum fungsi main ditambahkan.
	Contohnya:\\
#include <stdio.h>
\\prototype function
	int hitung(int angka, int bilangan);
	int tulis(char);
	int tampil(int angka[],char huruf);
//fungsi main
	int main(){
		int array[3]={1,2,3};
		char huruf="D";
		//memanggil fungsi
		hitung (2,3);
		tulis("A");
		tampil(array,huruf);
}

4. Fungsi Rekursif
	Fungsi yang memanggil dirinya sendiri. Artinya , fungsi tersebut dipanggil di dalam tubuh fungsi itu sendiri. Parameter yang dilewatkan berubah sebanyak fungsi itu dipanggil.
	
\begin{Apa itu paket dan cara pemanggilan paket atau library dengan contoh kode program lainnya}
	<?php if ( ! defined('BASEPATH'))
		exit('No direct script access allowed');
	class Blog extends ci_controller {
	function __construct()
	{
		parent ::__construct();
	}
	function index()
	{
		echo "Hallo.. saya min yoongi adalah contoh dari boyband BTS 				yang mendunia";
	}
	
	
}

\begin{Jelaskan Apa itu kelas, apa itu objek, apa itu atribut, apa itu method dan contoh kode program lainnya masing-masing.}
gambaran umum tentang sebuah benda. Di dalam pemrograman nantinya, contoh class seperti: koneksi_database dan profile_user. penulisan class diawali dengan keyword class, kemudian diikuti dengan nama dari class. Aturan penulisan nama class sama seperti aturan penulisan variabel
Contoh :\\
<?php
	class laptop {
   		// isi dari class laptop...
}
?>

\end{document}

\begin{Jelaskan cara pemanggikan library kelas dari instansiasi dan pemakaiannya dengan contoh program lainnya.}\\
1. Membuat sebuah objek atau sebuah instance pada sebuah kelas disebut instansiasi atau instantiation.
Contoh:\\
	String str = new String("Hello");
	String str2 = "OOP Yes";
Komputer a = new Komputer();
Komputer b = new Komputer();

2. Atribut suatu class harus didefinisikan sebagai instance
variable.\\
Contoh:\\
	public class Time {
		private int hour;
		private int minute;
		private int second;
	//penulisan kode selanjutnya
}

\begin{Jelaskan dengan contoh pemakaian paket dengan perintah from kalkulator import Penambahan disertai dengan contoh kode lainnya.}\\
Paket dengan perintah from kalkulator import import penambahan pertama , yaitu tentukan nama fungsi , variabel dan inputnya. setiap penulisan harus menggunakan () dan : dan identasi.
Contoh  :\\
	def penambahan (a+b):\\
	r=(a+b)\\
	return\\
	a=5\\
	b=6\\
	anu=penambahan(a,b)

\begin{6.Jelaskan dengan contoh kodenya, pemakaian paket fungsi apabila file library ada di dalam folder.}\\
// Meletakkan kelas ke paket
package bangun.datar;
 
// Mendefinisikan kelas Segi3ABC
public class Segi3ABC {
 
   // Metoda hitungKeliling
   // Untuk mencari keliling segi tiga
   public static double hitungKeliling(double sisiAB, double sisiBC, double sisiCA) {
 
      double keliling;
      keliling = sisiAB + sisiBC + sisiCA;
      return keliling;
   }
 
   // Metoda hitungLuas
   // Untuk mencari luas segi tiga
   public static double hitungLuas(double sisiAB) {
 
      // Deklarasi variabel
      double luas;
 
      // Mencari tinggi segi tiga
      double tinggi = Math.sqrt(Math.pow(sisiAB, 2) - Math.pow((0.5 * sisiAB), 2));
 
      // Mencari luas segi tiga
      luas = sisiAB * tinggi;
      return luas;
   }
}
\end {enumerate}
	\subsection{Keterampilan Pemrograman}
\begin{itemize}

\end{itemize}
	\item 
		\lstinputlisting {firstline = 58, lastline 64} {src/1174075}
	\item 
        \lstinputlisting {firstline = 67, lastline 76} {src/1174075}
	\item 
        \lstinputlisting {firstline = 79, lastline 84} {src/1174075}
	\item 
        \lstinputlisting {firstline = 87, lastline 91} {src/1174075}
	\item 
        \lstinputlisting {firstline = 94, lastline 100} {src/1174075}
	\item 
        \lstinputlisting {firstline = 103, lastline 109} {src/1174075}
	\item 
        \lstinputlisting {firstline = 112, lastline 117} {src/1174075}
	\item 
	    \lstinputlisting {firstline = 120, lastline 125} {src/1174075}
	\item 
        \lstinputlisting {firstline = 128, lastline 141} {src/1174075}

\end {itemize}

\bibliographystyle{IEEEtran} 
%\def\bibfont{\normalsize}
\bibliography{references}


%%%%%%%%%%%%%%%
%%  The default LaTeX Index
%%  Don't need to add any commands before \begin{document}
\printindex

%%%% Making an index
%% 
%% 1. Make index entries, don't leave any spaces so that they
%% will be sorted correctly.
%% 
%% \index{term}
%% \index{term!subterm}
%% \index{term!subterm!subsubterm}
%% 
%% 2. Run LaTeX several times to produce <filename>.idx
%% 
%% 3. On command line, type  makeindx <filename> which
%% will produce <filename>.ind 
%% 
%% 4. Type \printindex to make the index appear in your book.
%% 
%% 5. If you would like to edit <filename>.ind 
%% you may do so. See docs.pdf for more information.
%% 
%%%%%%%%%%%%%%%%%%%%%%%%%%%%%%

%%%%%%%%%%%%%% Making Multiple Indices %%%%%%%%%%%%%%%%
%% 1. 
%% \usepackage{multind}
%% \makeindex{book}
%% \makeindex{authors}
%% \begin{document}
%% 
%% 2.
%% % add index terms to your book, ie,
%% \index{book}{A term to go to the topic index}
%% \index{authors}{Put this author in the author index}
%% 
%% \index{book}{Cows}
%% \index{book}{Cows!Jersey}
%% \index{book}{Cows!Jersey!Brown}
%% 
%% \index{author}{Douglas Adams}
%% \index{author}{Boethius}
%% \index{author}{Mark Twain}
%% 
%% 3. On command line type 
%% makeindex topic 
%% makeindex authors
%% 
%% 4.
%% this is a Wiley command to make the indices print:
%% \multiprintindex{book}{Topic index}
%% \multiprintindex{authors}{Author index}

\end{document}

